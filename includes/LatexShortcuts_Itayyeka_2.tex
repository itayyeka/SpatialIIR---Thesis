%general:
%Box and color definitions:
%--------------------------
%General definitions:
%-------------------
\newcommand{\etal}{{\em {et al.}}}
% \newcommand{\B}[1]{\mathbf{#1}}
\newcommand{\df}{\triangleq}
\newcommand{\norm}[1]{\left\Vert#1\right\Vert}
\newcommand{\abs}[1]{\left\vert#1\right\vert}
\newcommand{\RE}{\operatorname{Re}}
\newcommand{\IM}{\operatorname{Im}}
\newcommand{\sgma}[3]{\sum\limits_{{#1}={#2}}^{#3}}
\newcommand{\Brace}[1]{\left\{{#1}\right\}} %Braces
\newcommand{\Brack}[1]{\left({#1}\right)} %Brackets
\newcommand{\sBrack}[1]{\left[{#1}\right]} %square Brackets

%\newcommand{\ip}[2]{{\langle{#1},{#2}\rangle}} %inner-product
\newcommand{\ipLW}[3]{{\langle{#1},{#2}\rangle}_{{#3}}} %weighted inner-product

\newcommand{\Tr}[1]{Tr\Brack{#1}}
\newcommand{\Mtr}[2] %short notation for 2x1 Matrix.
{\begin{bmatrix}
  #1 \\
  #2
\end{bmatrix}}
\newcommand{\cMtr}[2] %short notation for 2x1 Matrix with curves.
{\left(
\begin{array}{c}
    {#1} \\
    {#2} \\
\end{array}
\right)}
\newcommand{\Mtrs}[2] %short notation for 2x1 Matrix star (adjoint)
{\begin{bmatrix}
  #1 &
  #2
\end{bmatrix}}
\newcommand{\Mtrt}[3] %short notation for 3x1 Matrix.
{\begin{bmatrix}
  #1 \\
  #2 \\
  #3
\end{bmatrix}}

\newcommand{\Cases}[4]{
\left\{
\begin{tabular}{lcl}
    $#1$ & $=$ & $#2$\\
    $#3$     & $=$ & $#4$
\end{tabular}
\right. }

\newcommand{\und}{\underline} %How lazy can I get?
\newcommand{\ovr}{\overline}
\newcommand{\conj}[1]{{#1}^\ast} %Conjugation


\newcommand{\er}[1]{{(\ref{#1})}} %equation reference

\newtheorem{Lemma}{Lemma}{}
\newtheorem{Prop}{Proposition}{}
% \newtheorem{theorem}{Theorem}{}


\newenvironment{alg}[5]
{
\begin{figure}[htbp]
\begin{center}
\fbox{
  \begin{ColorBoxedminipage}{13cm}
%    \leftline{\color{Black}\bf {#1}}
    {#4}
   \end{ColorBoxedminipage}
   }
\end{center}
  \bcaptionff{#1}{#2}{}{#3}
  \label{#5}
\end{figure}
}{}

%Just body, caption and label.
\newenvironment{algo}[3]
{
\begin{figure}[htbp]
\begin{center}
\fbox{
  \begin{ColorBoxedminipage}{7.5cm}
%    \leftline{\color{Black}\bf {#1}}
    {#1}
   \end{ColorBoxedminipage}
   }
\end{center}
  \caption{#2}
  \label{#3}
\end{figure}
}{}

\newenvironment{BOX}[1]
{
\begin{center}
\fbox{
  \begin{ColorBoxedminipage}{16cm}
%    \leftline{\color{Black}\bf {#1}}
    {#1}
   \end{ColorBoxedminipage}
   }
\end{center}
}{}

\newcommand\vecnot[1]{\boldsymbol{#1}}
\newcommand\optvecnot[1]{\vecnot{#1}_{opt}}

\usepackage{amsmath}
\DeclareMathOperator*{\argmax}{arg\,max}
\DeclareMathOperator*{\argmin}{arg\,min}