In the process of simplifying \eqref{eq_D} in order to find the expression for the directivity $\mathcal{D}$, we noticed the results were rational.
Following this observation, we numerically swept various values of $N,r$ pairs, trying find a common expression that fits all results which will enable the formulation of the directivity improvement when using the feedback beamformer.
In the following, we elaborate on the technicalities of this sweeping process.
\par The first step was to formulate the symbolic expression for $H$.
To this end, we used MATLAB\textsuperscript{\textregistered}'s symbolic toolbox, generating the symbolic response $H$.
Afterwards, the 2D sweep range was set, where as mentioned before, $r$ was set to be of rational values.
In each sweep step, a numerical integration was executed with specific values for $N$ and $r$.
Each result was fitted with it's next closest rational number with the help of MATLAB\textsuperscript{\textregistered}'s \emph{rats} auxiliary command.
After observing the results, we found that all of them obey the same expression as in \eqref{eq_D_result} for any tolerance, which increased the confidence in the suggested expression.
\par The printout of the code is presented as Table.~\ref{table_D} and the text below.
\begin{table}[h!]
    \caption{Performances of Classical ULA and the Proposed Feedback-Beamforming Architecture, with a Gain Mismatch $r$.}
    \centering
    %\resizebox{1\linewidth}{!}
    {
        \begin{tabular}{||c c c c c c c c c c||}
            \hline
            $r \textbackslash N$ & 2 & 3 & 4 & 5 & 6 & 7 & 8 & 9 & 10 \\ [0.5ex] 
            \hline\hline
            1/10     & 19/9  & 29/9  & 13/3  & 49/9  & 59/9  & 23/3  & 79/9  & 89/9  & 11   \\ [1ex] 
            1/5      & 9/4   & 7/2   & 19/4  & 6     & 29/4  & 17/2  & 39/4  & 11    & 49/4 \\ [1ex] 
            1/3      & 5/2   & 4     & 11/2  & 7     & 17/2  & 10    & 23/2  & 13    & 29/2 \\ [1ex] 
            1/2      & 3     & 5     & 7     & 9     & 11    & 13    & 15    & 17    & 19   \\ [1ex] 
            5/8      & 11/3  & 19/3  & 9     & 35/3  & 43/3  & 17    & 59/3  & 67/3  & 25   \\ [1ex] 
            3/4      & 5     & 9     & 13    & 17    & 21    & 25    & 29    & 33    & 37   \\ [1ex] 
            4/5      & 6     & 11    & 16    & 21    & 26    & 31    & 36    & 41    & 46   \\ [1ex] 
            \hline
         \end{tabular}
     }
    \label{table_D}
\end{table}
\section*{MATLAB\textsuperscript{\textregistered} Code}
\lstinputlisting[style=mcode]{Modules/directivityAnalysis.m}
