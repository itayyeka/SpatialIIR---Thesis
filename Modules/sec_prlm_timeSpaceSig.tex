% The incoming signals are spatially sampled by the array microphones, then the samples are processed to attenuate signals from undesired directions and extract the signal from a desired direction. 
% A spatial response of the microphone array is obtained via a beam (main-lobe) directed to the desired signal while nulls are directed to the undesired signals.
As mentioned before, the measurements of the sensor array's elements are fused together, where spatial information resides in the relations between samples taken in the same time $t$.
A basic sensor array based entity is the beamformer, which is designed to enhance signals from certain directions while suppressing other signals from unwanted directions, heuristically forming a beam directed to the enhanced direction.  
\begin{figure}[ht!]
    \begin{center}
        \begin{overpic}[width=0.5\linewidth, 
        %grid, 
        tics=10,trim=0 0 0 0]{./Media/fig_ULA_imping.png}
            % \put (50, 62.5) {\footnotesize{$r=0$}}
        \end{overpic}
    \end{center}
     \caption{A simple beamformer design.}
    \label{fig_ULA_imping}
\end{figure}
Fig.~\ref{fig_ULA_imping} illustrates a beamformer design based on a ULA of Fig.~\ref{fig_ULA}.
% Let $\vecnot{f}\rBrace{t,\vecnot{x}}$ denote the set of signals sampled by the microphone array at time $t$, expressed by
A temporal snapshot measured by the array at time $t$ is
\begin{equation}
\vecnot{f}\rBrace{t,\vecnot{x}} = \vBrace{f\rBrace{t,x_{0}},\dots,f\rBrace{t,x_{M-1}}}^{T},
\end{equation}
where $x_{m}$ denotes the microphone position, $t$ denotes the continuous-time variable and $^{T}$ denotes the transpose operation. 
Assuming simple, weight-and-sum beamformer, the output ($y\rBrace{t}$) is expressed by:
\begin{equation}
y\rBrace{t} = \sum_{m=0}^{M-1}{f\rBrace{t,x_{m}}w_{m}^{*}},
\end{equation}
where $w_{m}$ is a complex weight used for the $m$'th sensor's output, as shown in Fig.~\ref{fig_ULA_imping} and $^{*}$ denotes complex conjugation. 
Consider $s\rBrace{t} = e^{j\omega{t}}$ as a plane wave propagating at angular frequency $w$, and $\theta \in \vBrace{-\frac{\pi}{2},\frac{\pi}{2}}$ as the direction of arrival (DOA) angle measured with respect to the broadside of the linear array, as shown in Fig.~\ref{fig_ULA_imping}.
The wave signals spatially sampled by the sensor array inputs are
\begin{equation}
\label{eq_prlm_timeSpace_outputVec}
\vecnot{f}\rBrace{t,x} = \vBrace{s\rBrace{t-\tau_{0}}, \dots, s\rBrace{t-\tau_{M-1}}}^{T}, \tau_{m} = \frac{\sin{\theta}\cdot{}x_{m}}{c},
\end{equation}
where $\tau_{m}$ is the propagation delay for the incoming signal and $c$ is the wave's velocity in the medium.
% A value of $\tau_{m}=0, \forall{m}$ implies a DOA of $\theta = 0$, i.e. a plane wave parallel to the array, propagating perpendicularly to the array.
Let $\kappa=\frac{\omega}{c}\sin{\theta}=\frac{2\pi}{\lambda}\sin{\theta}$ denote the wavenumber for plane waves in a locally homogeneous medium, where $\lambda$ denotes the wavelength corresponding to the angular frequency $\vecnot{d}\rBrace{\kappa}$ denote the \emph{steering vector}, featuring all of the array's spatial characteristics.
Based on \eqref{eq_prlm_timeSpace_outputVec}, and the definition of $\kappa$ above, the steering vector can be expressed as
\begin{equation}
\vecnot{d}\rBrace{\kappa}=\vBrace{e^{-j\kappa{}x_{0}},\dots,e^{-j\kappa{}x_{M-1}}}^{T}.
\end{equation}