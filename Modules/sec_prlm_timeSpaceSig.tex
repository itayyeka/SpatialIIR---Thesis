As mentioned before, the measurements of the sensor array's elements are fused together, where spatial information resides in the relations between samples taken the same time $t$.
A basic application of the sensor array is the beamformer, designed to enhance signals from certain directions while suppressing other signals from unwanted directions, forming a beam directed to the enhanced direction.  
\begin{figure}[ht!]
    \begin{center}
        \begin{overpic}[width=0.6\linewidth, 
        %grid, 
        tics=10,trim=0 0 0 0]{./Media/arrayProc.png}
        \put(20,65){\rotatebox{-50}{\tiny{Impinging wavefront}}}
        \put(49,33.25){\rotatebox{0}{$\theta$}}
        \put(15,25){\rotatebox{0}{\tiny{0}}}
        \put(28,25){\rotatebox{0}{\tiny{1}}}
        \put(41,25){\rotatebox{0}{\tiny{2}}}
        \put(80,25){\rotatebox{0}{\tiny{$N-1$}}}
        \put(16,20){\rotatebox{0}{\tiny{$\omega_{0}$}}}
        \put(29.5,20){\rotatebox{0}{\tiny{$\omega_{1}$}}}
        \put(42,20){\rotatebox{0}{\tiny{$\omega_{2}$}}}
        \put(81,20){\rotatebox{0}{\tiny{$\omega_{N-1}$}}}
        \put(50.5,3.5){\rotatebox{0}{\tiny{$z\rBrace{t}$}}}
        \end{overpic}
    \end{center}
     \caption{ULA based coherent processing of simultaneous spatially sampled signal, measuring a wavefront impinging from DOA $\theta$}
    \label{fig_ULA_imping}
\end{figure}
In complimentary to Fig.~\ref{fig_ULA_sketch}, Fig.~\ref{fig_ULA_imping} adds the processor layer of the beamformer.
\par A temporal snapshot measured by the array at time $t$ is
\begin{equation}
\vecnot{f}\rBrace{t,\vecnot{p}} = \vBrace{f\rBrace{t,p_{0}},\dots,f\rBrace{t,p_{N-1}}}^{T},
\end{equation}
where $p_{n}$ denotes the $n$-th sensor position, $t$ denotes the continuous-time variable and $^{T}$ denotes the transpose operation. 
Considering the \emph{conventional beamformer} \cite{van2004optimum}, the output ($z\rBrace{t}$) is expressed by:
\begin{equation}
z\rBrace{t} = \sum_{n=0}^{N-1}{f\rBrace{t,p_{n}}\omega_{n}^{*}},
\end{equation}
where $\omega_{n}$ is a complex weight used for the $n$'th sensor's output, as shown in Fig.~\ref{fig_ULA_imping} and $^{*}$ denotes complex conjugation. 
Consider $s\rBrace{t} = e^{j\omega{t}}$ as a plane wave propagating at temporal frequency $\omega$, and $\theta \in \vBrace{-\frac{\pi}{2},\frac{\pi}{2}}$ as the direction of arrival (DOA) angle measured with respect to the broadside of the linear array, as shown in Fig.~\ref{fig_ULA_imping}.
The wave signal, spatially sampled by the sensor array inputs is
\begin{equation}
\label{eq_prlm_timeSpace_outputVec}
\vecnot{f}\rBrace{t,x} = \vBrace{s\rBrace{t},s\rBrace{t-\tau}\dots, s\rBrace{t-\rBrace{N-1}\tau}}^{T},
\end{equation}
where $\tau = \frac{d\cdot{}\cos{\theta}}{c}$ is the propagation delay between consecutive sensors and $c$ is the wave's velocity in the medium.
Considering narrowband signals, the time delay $\tau$ translates to a mere phase shift and the \emph{steering vector} takes the form of
\begin{equation}
\vecnot{d}\rBrace{\theta}=\vBrace{1,e^{-j\omega\tau},\dots,e^{-j\omega\tau\rBrace{N-1}}}^{T}.
\end{equation}
As every impinging signal will cause the excitation of the array elements in the form of a specific steering vector (up to a common additive phase), we define the \emph{array manifold} to be the set of all steering vectors of $\theta\in\left[0,2\pi\right)$, hence it also spans the received signal sub space in the noiseless scenario.