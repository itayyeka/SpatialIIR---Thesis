As a preliminary, we start with simple temporal simulations of the feedback beamformer.
% In this section, we start by identifying the unknown parameters in the experimental scenario and formulate their related error terms.
% Using those error terms, some basic simulation scenarios are defined, raging from ideal to fully noised cases, which enable rigorous presentation of our findings.
% After the error terms have been defined, 
We first address the temporal stability of the system and discuss its correspondence with the system's response which was found in \eqref{eqn:GeneralFeedbackTransferFunction}.
\subsection{Gain mismatch}
The analogy of the FB to the temporal IIR architecture raises some fundamental issues that should be addressed.
In the following, supported by simulations, we answer those questions.
\par The first question, related to the FB's analogy to the temporal IIR architecture, is stability related.
To begin with, we should first ask what does ``stability`` means.
% Interestingly, the system's response of \eqref{eqn:GeneralFeedbackTransferFunction} is not frequency dependent, for it is exclusively governed by spatial parameters.
% Therefore, 
In the absence of any previous related work on spatial feedback, we find that the commonly used Bounded-Input-Bounded-Output (BIBO) stability is appropriate. 
% As such, assuming the \emph{Perfect alignment} scenario, we identify $r$ in \eqref{eq:SF_CB} to be the most influential for determining the system's stability for it controls the response's denominator - e.g. for $r<1$ the denominator will never be nullified.
To this end, the gain mismatch is
\begin{equation}
    r=g/\hat{g}.
\end{equation}
And plugging \eqref{eq:alpha_beta_opt} into \eqref{eqn:GeneralFeedbackTransferFunction}, gives rise to
\begin{equation}\label{eq:SF_CB_gainMismatch}
    %\resizebox{.894\linewidth}{!}
    {
        \begin{split}
            H_{\vecnot{\beta}_{\text{CB,opt}},\vecnot{\alpha}_{\text{CB,opt}}}
            = &\frac{
            g\vecnot{\beta}_{\text{CB,opt}}^{H}\vecnot{d}\exp{\rBrace{-j\phi}}
            }{
            1 - g\vecnot{\alpha}_{\text{CB,opt}}^{H}\vecnot{d}\exp{\rBrace{-j\phi}}
            }\\
            % = &\frac{    
            % \frac{r}{N}\vecnot{d}^{H}\vecnot{d}
            % }{
            % 1 - \frac{r}{N}\vecnot{d}^{H}\vecnot{d}
            % }\\
            = & \frac{r}{1-r} \triangleq H_{0}.
        \end{split}
    }
\end{equation}
As anticipated from \eqref{eq:SF_CB_gainMismatch} and confirmed with the simulations presented in Fig.\ref{fig_stability}, for $r\geq1$ the system is not stable, for the received amplitude may increase (in absolute value) to infinity.
In Fig.~\ref{fig_stabilityVal}, we plot the final amplitude value after 100 iterations of the signal being retransmitted between the array and the target of interest and find that the final values substantially increase as $r\to{1}$.
This issue will have to be taken into consideration when one designs such an array for practical applications but currently this is outside the scope of this work.
As a final observation, we address the ``discrete`` behaviour of $z$, i.e. the value (most noticeable in the $r=1.1$ plot in Fig.~\ref{fig_stability}) seems to be piecewise constant.
This phenomenon corresponds with the propagation latency of the signal's round trip to the target and back.
In each iteration the mainlobe sharpens, as can be seen in Fig.~\ref{fig_beamThinning}.
The reader may notice in Fig.~\ref{fig_beamThinning} that the $t=1\cdot{}\tau_{pd}$ plot is actually the conventional beampattern, for the feedback signal has not yet re-impinged the array.
\begin{figure}[t!]
    \begin{center}
        \begin{overpic}[width=0.55\linewidth, 
        %grid, 
        tics=10,trim=0 0 0 0]{./Media/fig_beamThinning.png}
            \put (50, -5){\rotatebox{0}{$\theta_{d}/\pi$}}
            \put (-6, 38){\rotatebox{90}{$H_{r=0.9}$}}
            \put (100, 49){\rotatebox{0}{$t=1\cdot\tau_{pd}$}}
            \put (100, 39){\rotatebox{0}{$t=2\cdot\tau_{pd}$}}
            \put (100, 22.5){\rotatebox{0}{$t=5\cdot\tau_{pd}$}}
            \put (100, 12){\rotatebox{0}{$t=10\cdot\tau_{pd}$}}
            \put (53.5, 27.5){\rotatebox{-75}{$t=50\cdot\tau_{pd}$}}
            \put (4.5, 80){\rotatebox{0}{dB}}
        \end{overpic}
    \end{center}
    \caption{
    Simulation of the same setup as in Fig.\ref{fig_stability} with $r=0.9$.
    Several simulations are conducted, where each is simulated with different duration.
    It actually illustrates the forming of the final beampattern in time as our analysis assumes infinite duration.
    This convergence is analogous to the settling time of the IIR filters' temporal response, where at early stages, the response is not yet in its steady-state.  
    }
    \label{fig_beamThinning}
\end{figure}
\begin{figure}[t!]
    \begin{center}
        \begin{overpic}[width=0.55\linewidth, 
        %grid, 
        tics=10,trim=0 0 0 0]{./Media/fig_stabilization.png}
            \put (46, -5){\rotatebox{0}{$t/\tau_{pd}$}}
            \put (-6, 38){\rotatebox{90}{$\abs{z\rBrace{t}}$}}
            \put (46, 18.5){\rotatebox{0}{$r=0.6$}}
            \put (46, 35){\rotatebox{0}{$r=0.9$}}
            \put (46, 54){\rotatebox{0}{$r=1$}}
            \put (18, 60){\rotatebox{0}{$r=1.1$}}
            \put (0.75, 80){\rotatebox{0}{dB}}
        \end{overpic}
    \end{center}
    \caption{
    Simulating a 3 element ULA based FB, steered to a target which resides in the direction $\theta_{s} = \pi/2$ and plotting temporal response (dB) for multiple $r$ values.
    The horizontal axis ($t/\tau_{pd}$) is the time, normalized to the signal's round-trip duration to the target and back.
    }
    \label{fig_stability}
\end{figure}
\begin{figure}[t!]
    \begin{center}
        \begin{overpic}[width=0.55\linewidth, 
        %grid, 
        tics=10,trim=0 0 0 0]{./Media/fig_stabilizationVal.png}
            \put (0, 82){\rotatebox{0}{dB}}
            \put (-6, 30){\rotatebox{90}{$\lim_{t\to\infty}\abs{z\rBrace{t}}$}}
            \put (51, -5){\rotatebox{0}{$r$}}
        \end{overpic}
    \end{center}
    \caption{
    After simulating the same setup as in Fig.~\ref{fig_stability}, we plot the final received amplitude in dB with different choices of $r$.
    }
    \label{fig_stabilityVal}
\end{figure}
\par The second question to be asked regards the settling time of the system.
For example, in Fig.~\ref{fig_stability} we easily observe that the settling time for $r=0.6$ ($\sim{}10$ signal round trips) is substantially shorter than its matching value ($\sim{}50$) when $r=0.9$.
To better understand the phenomenon, we plot for each $r$ value, its corresponding settling time in Fig.~\ref{fig_stabilityDur}.
Although outside of this work's scope, this issue will obviously be of great importance when considering dynamic targets, as will be discussed in the ``future research`` part of the concluding Chapter.~\ref{chap:conclusion}.
\begin{figure}[t!]
    \begin{center}
        \begin{overpic}[width=0.55\linewidth, 
        %grid, 
        tics=10,trim=0 0 0 0]{./Media/fig_stabilizationDur.png}
            \put (-3, 35){\rotatebox{90}{$t/\tau_{\text{pd}}$}}
            \put (51, -5){\rotatebox{0}{$r$}}
        \end{overpic}
    \end{center}
    \caption{
    Simulation of same setup as in Fig.~\ref{fig_stability}.
    For each $r$ (the horizontal axis), we plot the time (the vertical axis) where the received amplitude entered the $1\%$ sleeve around its final value.
    The plotted time is also normalized in units of $\tau_{\text{pd}}$ as in Fig.~\ref{fig_stability}.
    }
    \label{fig_stabilityDur}
\end{figure}
% \par To conclude this section, we choose to add one more note, regarding the influence of the array's number of elements.
% Although in spatial processing literature, enlarging the number of elements in the array commonly increases the spatial performance, both examining \eqref{eq:SF_CB_gainMismatch} and simulating the experimental scenarios show that it (increase of elements) does not affect the temporal performance of the system - i.e. the settling time remains the same as $N$ increases.
\subsection{Phase mismatch}
\label{subsec_error_trms}
% Using the same weights $\vecnot{\beta}_{\text{CB,opt}},\vecnot{\alpha}_{\text{CB,opt}}$, 
We denote $\hat{\phi},\hat{\theta}$ to be the range and DOA related phase estimates respectively.
As we intend to compare the results to \cite{van2004optimum}, we consider a ULA, introducing its estimated steering vector
\begin{equation}\label{eq_est_d_ULA}
\hat{\vecnot{d}} = \vBrace{1, \exp{\rBrace{-\hat{\theta}}}, \hdots, \exp{\rBrace{-\rBrace{N-1}\hat{\theta}}}}^{T}.
\end{equation}
Observing that
\begin{equation*}
    \begin{split}
        % \hat{\vecnot{d}}^{H}\vecnot{d} =&\ \vecnot{d}^{H}\hat{\vecnot{d}}\\
        \hat{\vecnot{d}}^{H}\vecnot{d} =&\ \sum_{n=0}^{N-1}\exp{\rBrace{jn\Delta\theta}} \\
        =&\ \frac{\exp{\rBrace{jN\Delta\theta}}-1}{\exp{\rBrace{j\Delta\theta}}-1}\\
        =&\ \frac{\exp{\rBrace{jN\Delta\theta/2}}}{\exp{\rBrace{j\Delta\theta/2}}}\cdot
        \frac{\exp{\rBrace{jN\Delta\theta/2}}-\exp{\rBrace{-jN\Delta\theta/2}}}
        {\exp{\rBrace{j\Delta\theta/2}}-\exp{\rBrace{-j\Delta\theta/2}}}\\
        =&\ \exp{\rBrace{j\rBrace{N-1}\Delta\theta/2}}\cdot
        \frac{\sin{\rBrace{N\Delta\theta/2}}}
        {\sin{\rBrace{\Delta\theta/2}}}
    \end{split}
\end{equation*}
and using the definition of the normalized Dirichlet kernel (illustrated in Fig.~\ref{fig_DirichletKernelNorm})
\[
\D{x}{N}
% =\frac{1}{N}D_{\test{orig}}\rBrace{x,N}
=\frac{1}{N}\frac{\sin\rBrace{Nx}}{\sin\rBrace{x}},
\]
we find that for ULA, \eqref{eqn:GeneralFeedbackTransferFunction} may be expressed as
\begin{equation}\label{eq:alpha_beta_hat}
    \begin{split}
        H_{\vecnot{\beta}_{\text{CB}},\vecnot{\alpha}_{\text{CB}}}
        = &\frac{r\D{\dTheta/2}{N}\exp\cBrace{-j\vBrace{\dPhi+\rBrace{N-1}\dTheta/2}}}{1-r\D{\dTheta/2}{N}\exp\cBrace{-j\vBrace{\dPhi+\rBrace{N-1}\dTheta/2}}}.
    \end{split}
\end{equation}
% \begin{figure}[t!]
%     \begin{center}
%         \begin{overpic}[width=0.65\linewidth, 
%         %grid, 
%         tics=10,trim=0 0 0 0]{./Media/DirichletKernels.png}
%             \put (3, 26){\rotatebox{90}{$D_{\test{orig}}\rBrace{x,N}$}}
%             \put (91, 32){\rotatebox{0}{\scriptsize{$D_{\test{orig}}\rBrace{x,2}$}}}
%             \put (91, 53){\rotatebox{0}{\scriptsize{$D_{\test{orig}}\rBrace{x,5}$}}}
%             \put (91, 59){\rotatebox{0}{\scriptsize{$D_{\test{orig}}\rBrace{x,7}$}}}
%             \put (91, 8){\rotatebox{0}{\scriptsize{$D_{\test{orig}}\rBrace{x,10}$}}}
%             \put (50, 2.5){\rotatebox{0}{$x$}}
%             \put (12, 5){\rotatebox{0}{$-\pi$}}
%             \put (29, 5){\rotatebox{0}{$-\frac{\pi}{2}$}}
%             \put (70, 5){\rotatebox{0}{$\frac{\pi}{2}$}}
%             \put (89, 5){\rotatebox{0}{$\pi$}}
%         \end{overpic}
%     \end{center}
%     \caption{
%     Few illustrations of the Dirichlet kernel, for $N$ values of 2,5,7 and 10.
%     }
%     \label{fig_DirichletKernel}
% \end{figure}
\begin{figure}[t!]
    \begin{center}
        \begin{overpic}[width=0.65\linewidth, 
        %grid, 
        tics=10,trim=0 0 0 0]{./Media/DirichletKernelsNorm.png}
            \put (3, 32){\rotatebox{90}{$D\rBrace{x,N}$}}
            \put (86.5, 47){\rotatebox{80}{\tiny{$D\rBrace{x,7}$}}}
            \put (81.5, 47){\rotatebox{80}{\tiny{$D\rBrace{x,5}$}}}
            \put (84.5, 30){\rotatebox{-86}{\tiny{$D\rBrace{x,10}$}}}
            \put (75.5, 30){\rotatebox{-70}{\tiny{$D\rBrace{x,2}$}}}
            \put (50, 2.5){\rotatebox{0}{$x$}}
            \put (12, 5){\rotatebox{0}{$-\pi$}}
            \put (29, 5){\rotatebox{0}{$-\frac{\pi}{2}$}}
            \put (70, 5){\rotatebox{0}{$\frac{\pi}{2}$}}
            \put (89, 5){\rotatebox{0}{$\pi$}}
        \end{overpic}
    \end{center}
    \caption{
    Few illustrations of the normalized Dirichlet kernel, for $N$ values of 2,5,7 and 10.
    }
    \label{fig_DirichletKernelNorm}
\end{figure}
In the following, four fundamental scenarios are considered:
\begin{itemize}
    \item{\makebox[.35\linewidth]{Perfect alignment \hfill} $\rBrace{\dTheta=0\ , \dPhi=0},$}
    \item{\makebox[.35\linewidth]{Steering error \hfill} $\rBrace{\abs{\dTheta}>0\ , \dPhi=0},$}
    \item{\makebox[.35\linewidth]{Range error \hfill} $\rBrace{\dTheta=0\ , \abs{\dPhi}>0},$}
    \item{\makebox[.35\linewidth]{General \hfill} $\rBrace{\abs{\dTheta}>0\ , \abs{\dPhi}>0}.$}
\end{itemize}