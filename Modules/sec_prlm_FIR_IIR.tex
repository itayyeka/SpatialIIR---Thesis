A basic design entity in every signal processing scheme is the digital filter, which is the modern evolution of analog filters.
The design of the digital filters is very well established and thoroughly studied research field.
As stated in \cite{oppenheim1975digital} (and numerous other literature), two key filter types are Finite Impulse Response (FIR) and Infinite Impulse Response (IIR) filters.
This section will briefly overview the design methods and the analogy of ULA based beamforming to the temporal FIR filtering, as implied in Ch.~\ref{chap:intro}.
\begin{itemize}
    \item \textbf{FIR design}\\
    The FIR filter is a merely a delay-and-sum mechanism, therefore its memory is finite and equal to the number of taps in the final filter's configuration as in Fig.~\ref{fig_FIR_arch}.
    \item \textbf{IIR design}\\
    IIR filters may be viewed as a generalization of the FIR delay-and-sum architecture, where delayed instances of the array's output are fed back to the summation as in Fig.~\ref{fig_IIR_arch}, theoretically generating an infinite loop where each output sample is affected by all past input samples. 
\end{itemize}
\begin{figure}[h!]
    \begin{center}
        \begin{overpic}[width=0.7\linewidth, 
        %grid, 
        tics=10,trim=0 0 0 0]{./Media/BASIC_FIR_FILTER_ARCH.png}
            \put (13, 36){\footnotesize{$s$}}
            \put (22, 22){\footnotesize{$\beta_{0}$}}
            \put (38.75, 22){\footnotesize{$\beta_{1}$}}
            \put (55.5, 22){\footnotesize{$\beta_{2}$}}
            \put (80.75, 22){\footnotesize{$\beta_{N-1}$}}
            \put (85, 10){\footnotesize{$z$}}
        \end{overpic}
    \end{center}
    \caption{An FIR based filtering of signal $s$ with the $\vecnot{\beta}$ as the coefficients, generating a filtered signal $z$.}
    \label{fig_FIR_arch}
\end{figure}
\begin{figure}[h!]
    \begin{center}
        \begin{overpic}[width=0.7\linewidth, 
        % grid, 
        tics=10,trim=0 0 0 0]{./Media/BASIC_IIR_FILTER_ARCH.png}
            \put (60, 50){\footnotesize{$\beta_{0}$}}
            \put (60, 30){\footnotesize{$\beta_{1}$}}
            \put (60, 10){\footnotesize{$\beta_{2}$}}
            \put (36, 30){\footnotesize{$\alpha_{1}$}}
            \put (36, 10){\footnotesize{$\alpha_{2}$}}
            \put (10, 50){\footnotesize{$s$}}
            \put (85, 50){\footnotesize{$z$}}
            \put (47.5, 34.5){\footnotesize{$z^{-1}$}}
            \put (47.5, 14.5){\footnotesize{$z^{-1}$}}
        \end{overpic}
    \end{center}
    \caption{Direct form II $2^{nd}$ order IIR architecture, where the FIR part is the $\vecnot{\beta}$ coefficients and the recursive part is implemented via the $\vecnot{\alpha}$ set.}
    \label{fig_IIR_arch}
\end{figure}
\par In the following, we revisit the basic interpretation of the ULA as the spatial equivalent to the temporal FIR filter \cite{van1988beamforming} where DOA is shown to be the matching spatial entity to the temporal frequency, hence the \emph{conventional beamformer} \cite{van2004optimum} is merely the spatial version of temporal domain FIR filtering. To this end, a brief overview on ULA beamforming and its resemblance to FIR filter design are presented next.
\par Consider an $N$-element ULA with inter-element spacing $d$, where its $n$'th sensor is positioned at $\vpi{n},\;\text{for}\; n=0,\ldots,N-1$. We set $\vpi{0}$ as the axis reference point and assume that the target of interest is positioned at $\vpt$.
Focusing on a far field localization problem, DOA and target range are to be estimated. Without loss of generality, aiming to simplify the exposition, we assume $2\text{D}$ planar problem, where DOA is described by a single angle $\thetaD$, measured from the array's broadside.
For simplicity, we assume an anechoic environment, an array of identical omni-directional sensors and a stationary target of interest.
Inspired by radar based applications, we also place a transmitter at $\vpi{0}$, assuming the transmitted signal $s$, is reflected back from the target and re-impinges the array, with a total time delay of $\tau_{\text{pd}}=2R/c$ seconds, where $R = \norm{\vpt-\vpi{0}}$ is the target's range and $c$ represents the propagation velocity of the signal in the medium.
\par Let $x_{n}(t)$ be the measured signal at the $n$'th sensor
\begin{equation}
x_{n}(t) = gs\Brack{t-\tau_{pd}-\tau_{n}},
\label{eqn:noFeedbackULA_singleSensor_temporal}
\end{equation}
where $\tau_{n}=nd\cos\Brack{\thetaD}/c$ represents the time difference of arrival between the $n$'th sensor and the reference sensor and $g$, being a scalar in an anechoic environment, is the channel's gain, related to both propagation and the target's radar cross section (RCS).
Defining $\vecnot{x}\rBrace{t}\triangleq\vBrace{x_{0}\rBrace{t}\hdots{}x_{N-1}\rBrace{t}}^{T}$ and its Fourier transform, $\vecnot{\F{x}}\rBrace{\omega}\triangleq\vBrace{X_{0}\rBrace{\omega},\hdots,X_{N-1}\rBrace{\omega}}^{T}$, one may write 
\[
\vecnot{\F{x}}\rBrace{\omega,\theta_{d}}=g\vecnot{d}\rBrace{\omega}\F{s}\rBrace{\omega}\exp{\rBrace{-j\omega\tau_{pd}\rBrace{\theta}}}
\]
where $S\rBrace{\omega}$ is the Fourier transform of $s\rBrace{t}$ and $\vdO$ denotes the steering vector whose $n$'th element is
\begin{equation}
    \label{eq:d}
    d_{n}\rBrace{\omega,\theta_{d}} = \exp{\rBrace{-j\omega\tau_{n}\rBrace{\theta}}}.
\end{equation}
Denoting the beamformer's weights as $\vBeta\rBrace{\omega}$ and the beamformer's output as $z$, we express the latter in the frequency domain
\begin{equation}
    \label{eq_Z}
    Z\rBrace{\omega,\theta_{d}} = g\vecnot{\beta}^{H}\rBrace{\omega}\vecnot{d}\rBrace{\omega}\F{s}\rBrace{\omega}\exp{\rBrace{-j\omega\tau_{pd}}}.
\end{equation}
Defining the electric phase to be
\begin{equation}\label{eq:thetaULA}
\theta=\omega{d\cos\Brack{\thetaD}}/{c},
\end{equation}
we rewrite \eqref{eq_Z} as 
\[
Z\rBrace{\omega,\theta} = g\F{s}\rBrace{\omega}\exp{\rBrace{-j\omega\tau_{pd}}}\sum_{n=0}^{N-1}\beta^{*}_{n}\rBrace{\omega}\exp\Brack{-jn\theta},
\]
hence in the ULA case, aiming for a desired spatial response, the weights vector $\vecnot{\beta}\rBrace{\omega}$ configuration is mathematically equivalent to an FIR filter design~\cite{van1988beamforming,benesty2018}. 
Assuming narrowband stimuli signals, we suppress $\omega$ dependency in the notation throughout the rest of this paper, where possible.
\par It is well known \cite{rabiner1974some} that IIR filters have several appealing advantages over their matching (in terms of number of taps) FIR counterparts.
The main advantage of IIR filters is that for a given set of frequency response specifications, the IIR filter will (in most cases) require substantially less coefficients, which may even reach few order of magnitude \cite{rabiner1974some}.
In the spatial processing context, this implies improved spatial performance, using the same or smaller number of sensors.
However, a well known \cite{oppenheim1975digital} fact in digital signal processing is that the filter latency may be frequency dependant, namely linearly dependent in the phase response derivative of the specific frequency.
Therefore, linear phase implies that the phase response's derivative (i.e. signal latency) is constant which implies all input frequencies will reside in the processor for the exact same time, preventing dispersion which practically causes pulse widening.
Due to this important property, many applications which are sensitive to signal distortions, especially communication based systems, chose the FIR architecture, for it is possible to design it as linear phase filter.
Acquainted with the analogy between frequency and DOA, it seems that when localization applications are considered, where the DOA resolution criteria is more important than the consistency of detection latency between separate DOAs, the linear phase property is less relevant.
% \par
% Concluding that spatial linear phase response in DOA domain is not a necessity, combined with the known efficiency of the IIR architecture and the need for smaller and cheaper arrays, the motivation for finding the spatial version of the temporal IIR is complete.
\par In standard radar signal processing schemes, a waveform is transmitted to, and reflected from the target of interest. Then, the reflected signal is processed by the radar reception array in order to estimate the target's dynamics (e.g., DOA, range, velocity etc.).
As opposed to the standard scheme, we suggest in Chapter.~\ref{chap:firstchap} a continuous re-transmission of the signal and its echoes back to the platform, generating a spatial feedback loop between the array and the target.
Another deviation from traditional radar processing, used to simplify the exposition, is using continuous-wave (CW) stimuli, rather than using pulse based signals.