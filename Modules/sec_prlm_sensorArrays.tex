Microphone arrays consist of sets of microphones positioned in a way to function as a directional acoustic antenna. 
Microphone arrays are utilized for filtering signals in a space-time field. 
Filtering is enabled by exploitation of the incoming signals spatial characteristics. 
A desirable spatial filtering, i.e., beamforming, should result in enhancement of signals of interest, originated in a specific direction, while forcing suppression of undesired signals originated in other directions.
Microphone arrays are utilized for solving many signal processing problems: dereverberation, localization of a single source, noise reduction and source separation \cite{cohen2004multichannel,habets2006dual,pavlidi2013real}.
\par 
Numerous factors must be taken into consideration when designing a microphone array configuration.
Initially, the geometry of the microphone array plays an important role in the formulation of the processing algorithms, as it forces fundamental constraints on the array’s operation. 
In most cases, the array geometry is the first consideration in array design due to practical and physical constraints of the design. 
Therefore, the degree of freedom in choosing the array geometry is limited. 
Nevertheless, in some other crucial problems such as noise reduction or source separation, the geometry of the array may have little importance. 
For instance, Uniform Linear Arrays (ULA) can only handle one source direction at a time resulting in uncertainties and possibly direction ambiguities.
Therefore, ULA configurations must be ruled out when handling several signal sources.
Other array geometries such as non-uniform linear arrays and circular arrays, have been studied in the field \cite{liu2008design,van2004optimum}.
This work is based on ULA and therefore we elaborate on this geometry.
A conventional beamformer \cite{van2004optimum} is composed of a ULA of microphones. 
That is, the microphones are positioned on an axis with a uniform spacing between the microphones.
The general expression of the microphone positioning in a ULA is given by:
\begin{equation}
x_{m}=m\cdot{d}, m=0, 1, \dots, M-1
\end{equation}
where $m$ denotes the microphone index; $x_{m}$ denotes the position of the microphone having index $m$, where $x_{m} = 0$ corresponds to the left-hand side of the array, i.e., the location where the microphone indexed by $m = 0$ is positioned; $d$ denotes the spacing between microphones, and $M$ is the array size, i.e., the number of microphones in the array, as demonstrated in Fig.~\ref{fig_ULA}.
\begin{figure}[h!]
    \begin{center}
        \begin{overpic}[width=0.5\linewidth, 
        %grid, 
        tics=10,trim=0 0 0 0]{./Media/fig_ULA.png}
            % \put (50, 62.5) {\footnotesize{$r=0$}}
        \end{overpic}
    \end{center}
     \caption{A uniform linear array of size M and spacing d.}
    \label{fig_ULA}
\end{figure}