Delay-and-sum beamformers, as shown in Fig.~\ref{XXX}, utilize a single weight for each microphone. 
This makes them ineffective when dealing with wideband signals, the signals of interest in speech and audio processing, as each of the signals is composed of various frequency components. 
In order to design a beamformer for wideband signals, the weight values in \eqref{XXX} must be altered for different frequencies in order to obtain the desired beamformer output. 
That is, the weights should be frequency dependent, i.e., in the form of $\vecnot{w}\rBrace{\omega}=\vBrace{w_{0}\rBrace{\omega},\dots,w_{M-1}\rBrace{\omega}}^{T}$. 
This can be acheived by introducing discrete FIR filters \cite{yan2005design,liu2010wideband}.
FIR filters perform temporal filtering in order to compensate for the phase differences of the input wideband signals’ various frequency components. 
Fig.\ref{XXX} shows the realization of frequency dependent weights by FIR filters connected to the microphone array.
Let $y\rBrace{t}$ denote the output of the FIR-based beamformer. 
It is expressed by:
\begin{equation}
y\rBrace{t}=\sum_{m=0}^{M-1}\sum_{n=0}^{N-1}f\rBrace{t-nT_{s},x_{m}}\cdot{}w_{m,n}^{*}
\end{equation}
where $N$ denotes the number of FIR filter coefficients connected to each of the $M$ sensors, $w_{m,n}^{*}$ is the $n$-th coefficient of the FIR filter connected to the microphone indexed by $m$, and $T_{s}$ denotes the delay between adjacent filter elements. 
Given a complex plane wave signal, the beamformer output is given by:
\begin{equation}
y\rBrace{t}=e^{j\omega{}t}P\rBrace{\omega,\theta}=e^{j\omega{}t}\sum_{m=0}^{M-1}\sum_{n=0}^{N-1}e^{-j\omega\rBrace{\tau_{m}+nT_{s}}}\cdot{}w_{m,n}^{*}
\end{equation}
Let $\vecnot{v}_{s}\rBrace{\omega,\theta}$ denote a \emph{stacked array manifold vector} of dimension $M\cdot{}N$, where each subvector of dimension $M$ epresents the array manifold vector associated with a specific FIR filter coefficient in \eqref{XXX}, i.e. the first subvector, $\vBrace{e^{-j\omega\tau_{0}},\dots,e^{-j\omega\tau_{M-1}}}^{T}$, is associated with the coefficient indexed by $n=0$ and all of the array microphones, indexed by $m=0,\dots,M-1$.
Thus, it is denoted by $\vecnot{v}_{0}\rBrace{\omega,\theta}$.
The second subvector, $\vBrace{e^{=j\omega\rBrace{\tau_{0}+T_{s}}},\dots,e^{=j\omega\rBrace{\tau_{M-1}+T_{s}}}}^{T}$, is associated with the coefficient indexed by $n=1$ and all of the array microphones.
Thus, it is denoted by $\vecnot{v}_{1}\rBrace{\omega,\theta}$, and so on.
This form of expression is called vector stacking, and $\vecnot{v}_{s}\rBrace{\omega,\theta}$ is given by:
\begin{align}
\vecnot{v}_{s}\rBrace{\omega,\theta} &= 
\begin{bmatrix}
   \vecnot{v}_{0}\rBrace{\omega,\theta} \\
   \vecnot{v}_{1}\rBrace{\omega,\theta} \\
   \vdots \\
   \vecnot{v}_{N-1}\rBrace{\omega,\theta}
\end{bmatrix}.
\end{align}
Given a ULA of $M$ sensors, sensor spacing $d$, and $M$ FIR filters, each composed of $N$ coefficients, and connected to a respective microphone. 
Then, from \eqref{XXX} and
\eqref{XXX}, the beamformer response can be expressed as:
\begin{equation}
    P\rBrace{\omega,\theta}=    \sum_{m=0}^{M-1}e^{-j\omega\tau_{m}}\sum_{n=0}^{N-1}e^{jn\omega{}T_{s}}\cdot{}w_{m,n}^{*}=\vecnot{w}_{s}^{H}\vecnot{v}_{s}\rBrace{\omega,\theta},
\end{equation}
where $\vecnot{w}_{s}$ denotes the composite stacked weight vector of dimension $M\cdot{}N$ created by vector stacking, having $\vecnot{w}_{0}=\vBrace{w_{0,0},w_{1,0},\dots,w_{M-1,0}}^{T}$ as its first subvector, $\vecnot{w}_{1}=\vBrace{w_{0,1},w_{1,1},\dots,w_{M-1,1}}^{T}$ as its second subvector, and so on.
The design specification of the FIR filters will be addressed in Chapter \ref{XXX}. Fig.\ref{XXX} illustrates an FIR beamformer architecture.