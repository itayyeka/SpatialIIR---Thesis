\section{Range Error Sensitivity}
\label{sec_sim}
In this section, we investigate $\HrTPr$ for the general case, where the range misalignment phase term $\dPhi$ may also be non zero. In Fig.~\ref{fig_hDUDTContour}, we plot $\abs{\HrTPr}$ in logarithmic scale, with respect to both steer and range misalignments.
Close inspection of the range error related beampattern behaviour sheds light to some important points.
First, we notice that although setting $r\to1$ (i.e., close to a perfect gain match), sharpens the beampattern's main lobe  (i.e., higher spatial selectivity), it also amplifies the range error ($\dPhi$) related sensitivity as the main lobe's support over the $\dPhi/\pi$ axis shrinks. 
Next, as evident from \eqref{eq_narmalized_pattern}, the range error related sensitivity is $2\pi$-periodic with respect to $\dPhi$ (see Fig.~\ref{fig_hDUDTContour_mutliPeak}).
To establish our final observation, we first recall that
\[
\phi\triangleq\omega\tau_{pd}=\frac{2\pi R_{\text{rt}}}{\lambda},
\]
where $R_{\text{rt}}=2R$ is the round-trip distance between the array and the target of interest and $\lambda$ is the wavelength. 
Define
\[
\Delta R_{\text{rt}}=\frac{\dPhi\lambda}{2\pi} 
\]
to be the range estimation error.
Fig.~\ref{fig_rangError} shows that even minor range errors of $\Delta R_{\text{rt}}\sim0.1\lambda$ significantly distort the beampattern.
\begin{figure}[t!]
    \begin{center}
        \begin{overpic}[width=.9\linewidth, 
        % grid, 
        tics=10,
        % trim={<left> <lower> <right> <upper>}
        trim={0cm 0cm 1.5cm 0cm}
        ]{./Media/spatialIIR_amb_N3_all_r.eps}
            \put (91, 74) {\footnotesize{dB}}
            \put (-2, 22) {$\frac{\dPhi}{\pi}$}
            \put (5, 74) {\footnotesize{$r=0.4$}}
            \put (51, 74) {\footnotesize{$r=0.6$}}
            \put (5, 38) {\footnotesize{$r=0.7$}}
            \put (51, 38) {\footnotesize{$r=0.8$}}
            \put (19, 3) {\footnotesize{$\dTheta/\pi$}}
        \end{overpic}
    \end{center}
    \caption{Evaluation of $10\log_{10}\abs{\Hr_{\dTheta,\dPhi,r}}^2$, considering both steer ($\dTheta$) and range related ($\dPhi$) errors. Centered in each plot, is the 3dB main lobe (white color fill), exemplifying that as the gain mismatch $r$ is set closer to one, we observe an increase of the spatial selectivity (regarding  both $\dTheta$ and $\dPhi$).}
  \label{fig_hDUDTContour}
\end{figure}
\begin{figure}[t!]
    \begin{center}
        \begin{overpic}[width=0.6\linewidth, 
        % grid, 
        tics=10,
        % trim={<left> <lower> <right> <upper>}
        trim={0 0 0 0}
        ]{./Media/spatialIIR_amb_N3r04_multiPeak.eps}
            \put (83, 71) {\tiny{dB}}
            \put (42, -0.5) {\scriptsize{$\dTheta/\pi$}}
            \put (0, 37) {$\frac{\dPhi}{\pi}$}
        \end{overpic}
    \end{center}
    \caption{Evaluation of $10\log_{10}\abs{\Hr_{\dTheta,\dPhi,r=0.4}}^2$ for $-3\pi\leq\dPhi\leq 3\pi$. The response is $2\pi$ periodic.}
  \label{fig_hDUDTContour_mutliPeak}
\end{figure}
\begin{figure}[t!]
    \begin{center}
        \begin{overpic}[width=0.6\linewidth, 
        % grid, 
        tics=10,
        % trim={<left> <lower> <right> <upper>}
        trim={0 0 0 0}
        ]{./Media/rangeError_r06.eps}
            \put (-1, 74) {\footnotesize{$10\log_{10}\abs{\Hr_{\dTheta,\dPhi,r}}^2$}}
            \put (46, -1) {\footnotesize{$\dTheta/\pi$}}
            \put (0, 37) {\footnotesize{dB}}
            \put (92, 40) {\footnotesize{$\Delta R_{\text{rt}}=0.1\lambda$}}
            \put (92, 50) {\footnotesize{$\Delta R_{\text{rt}}=0.3\lambda$}}
            \put (92, 30) {\footnotesize{$\Delta R_{\text{rt}}=0$}}
        \end{overpic}
    \end{center}
    \caption{Evaluation of the array response (where $r=0.4$) for several values of range error $\Delta R_{\text{rt}}$ . Even minor range errors significantly distort the beampattern.}
  \label{fig_rangError}
\end{figure}
\par At first glance, this sensitivity to range errors renders the system being too sensitive for any practical use.
This leads us to seek robust implementations, as elaborated in Sec.~\ref{sec_app}. 