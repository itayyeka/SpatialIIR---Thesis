In this chapter, we revisit the most basic DOA estimation approach, i.e. beamforming, and present our novel feedback based architecture.
The feedback based beamforming concept is introduced in Sec.~\ref{sec_introduceFeedback}.
Then, we apply information theory related reasoning for the array setting in Sec.~\ref{sec_FIM}, namely we use the Fisher Information Matrix (FIM).
The resultant beamformer's temporal response and spatial performance are analysed in Sec.~\ref{sec_stability} and Sec.~\ref{sec_Performance} respectively, assuming ideal noiseless scenario.
Considering noisy setups, the beamformer's noise sensitivity is exemplified in Sec.\ref{sec_sim}, followed by a practical and robust dual-frequency beamformer which is presented in Sec.~\ref{sec_app}.

\section{Notations and problem setup}
\label{sec_introduceFeedback}
In this section, a feedback-based architecture is proposed for spatial signal processing.
Inspired by time domain ``Direct form II'' IIR filter design (see Fig.~\ref{fig_IIR_arch}), 
we propose to use the same concept in the spatial domain.
The suggested feedback beamformer (FB) architecture, where the output signal ($z$) is synthesized using weights $\vBeta$ and the weights $\vAlpha$ synthesize the feedback transmission ($\Tx$), is presented in Fig.~\ref{fig:Proposed_spatialIIR_ARCH}. The beamformer's output and the feedback signal are synthesized using two independently configured beamformers, $s$ is the system's stimulus and an additive noise (n) is assumed at the array's output.
Also, the FB block is marked (dashed line) for later use.
Note that setting $\vAlpha=\vecnot{0}$ (i.e., cancelling the feedback) degenerates the system to a plain delay-and-sum (DS) beamformer.
\begin{figure}[t!]
    \begin{center}
        \begin{overpic}[width=0.5\linewidth, 
        % grid, 
        tics=10,trim={0 0 0 0}]{./Media/SpatialIIR-diagram/SpatialIIR_VER9.png}
            \put (14, 39){\footnotesize{$\sum_{n=0}^{N-1}\beta^{*}_{n}x_{n}$}}
            \put (54, 39){\footnotesize{$\sum_{n=0}^{N-1}\alpha^{*}_{n}x_{n}$}}
            \put (24.25, 50){\footnotesize{$x_{0}$}}
            \put (35, 50){\footnotesize{$x_{1}$}}
            \put (55, 50){\footnotesize{$x_{N-1}$}}
            \put (30.5, 66){\footnotesize{$\delta$}}
            \put (89, 96){\footnotesize{$\vpt$}}
            \put (58, 58){\footnotesize{$\vpi{N-1}$}}
            \put (37, 58){\footnotesize{$\vpi{1}$}}
            \put (26.5, 58){\footnotesize{$\vpi{0}$}}
            \put (41.5, 64.5){\footnotesize{$\theta_{d}$}}
            \put (19, 11.25){\large{$+$}}
            \put (61.25, 27.25){\large{$+$}}
            \put (47, 11.75){$s$}
            \put (35.25, 11.75){$n$}
            \put (21,4){$z$}
            \put (63.5,14){$\Tx$}
            \put (1, 51){$\text{FB}_{\vAlpha,\vBeta}$}
        \end{overpic}
    \end{center}
    \caption{
    The proposed feedback beamformer.
    The spatial feedback is obtained by continuous re-transmission of $\Tx$ to the target at $\vpt$.
    We designate the feedback beamformer block (dashed line) for later use.
    }
    \label{fig:Proposed_spatialIIR_ARCH}
\end{figure}
\subsection{Obtained spatial response}
As in Sec.~\ref{sec:prlm_FIR_IIR}, an $N$-element ULA with inter-element spacing $d$ is considered.
Its $n$'th sensor is positioned at $\vpi{n},\;\text{for}\; n=0,\ldots,N-1$ and we set $\vpi{0}$ as the axis reference point while the target of interest is positioned at $\vpt$.
We also place a transmitter at $\vpi{0}$, where transmitted signal $s$, is reflected back from the target and re-impinges the array, with a total time delay of $\tau_{\text{pd}}=2R/c$ seconds and $R = \norm{\vpt-\vpi{0}}, c$ are the target's range propagation velocity of the signal in the medium respectively.
Time domain analysis of the proposed feedback based architecture, considering both propagation delay and attenuation, gives rise to
\begin{equation}
    \label{eqn:SingleSensorTemporalEquality}
    % \resizebox{.91\linewidth}{!}{
        \begin{split}
            x_{n}(t) = g\vBrace{s\rBrace{t-\tau_{pd}-\tau_{n}}
            +\sum_{m=0}^{N-1}{\alpha^{*}_{m}x_{m}\rBrace{t-\tau_{pd}-\tau_{n}}}},
        \end{split}
    % }
\end{equation}
where the first term on the right-hand-side (RHS) represents the contribution of the transmitted waveform $s(t)$ to the $n$'th array element and the second term represents the feedback contribution of the re-transmitted array signal to this same element.
Expressing the Fourier transform of \eqref{eqn:SingleSensorTemporalEquality},
\begin{equation}
    \label{eqn_singleSensorFourier}
    % \resizebox{.91\linewidth}{!}{
            X_{n} =
            g\cBrace{\F{s}
            \exp\vBrace{-j\omega\rBrace{\tau_{pd}+\tau_{n}}}
            +\sum_{m=0}^{N-1}
            {
            \alpha^{*}_{m}\F{x}_{m}
            \exp\vBrace{-j\omega\rBrace{\tau_{pd}+\tau_{n}}}
            }},
    % }
\end{equation}
and its vector from,
$$
\F{\vx} = g\rBrace{\F{s}+\vAlphaH \F{\vx}}\vd\exp{\rBrace{-j\omega\tau_{pd}}},
$$
we find that it can be simplified to
$$
\F{\vx} =g\rBrace{I-g\vd\vAlphaH{}e^{-j\omega\tau_{pd}}}^{-1}\vd\F{s}\exp{\rBrace{-j\omega\tau_{pd}}}.
$$
Then, denoting
\[
\phi\triangleq\omega\tau_{pd}
\]
as the round-trip signal propagation related electrical phase.
We use the Sherman-Morrison formula \cite{sherman1950adjustment}, considered to be a result of the Woodbury matrix identity \cite{woodbury1950inverting}, stating that
\begin{equation*}
    \rBrace{I+\vecnot{u}\vecnot{v}^{T}}^{-1}=I-\frac{\vecnot{u}\vecnot{v}^{T}}{1+\vecnot{v}^{T}\vecnot{u}},
\end{equation*}
where $\vecnot{u},\vecnot{v}$ are two $N\times1$ vectors and $I$ is the identity matrix.   
Setting $\vecnot{u}=-g\vecnot{d}\exp{\rBrace{-j\omega\tau}}$ and $\vecnot{v}=\vecnot{\alpha}^{*}$ gives rise to
\begin{equation*}
    \begin{split}
        \rBrace{I-g\vecnot{d}\vecnot{\alpha}^{H}\exp{\rBrace{-j\omega\tau}}}^{-1} 
        &= I+\frac{g\vecnot{d}\vecnot{\alpha}^{H}\exp{\rBrace{-j\omega\tau}}}{1-g\vecnot{\alpha}^{H}\vecnot{d}\exp{\rBrace{-j\omega\tau}}}
        \\
        &= \frac{\rBrace{I-g\vecnot{\alpha}^{H}\vecnot{d}\exp{\vBrace{-j\omega\tau}}+g\vecnot{d}\vecnot{\alpha}^{H}\exp{\vBrace{-j\omega\tau}}}\vecnot{d}}{1-g\vecnot{\alpha}^{H}\vecnot{d}\exp{\rBrace{-j\omega\tau}}}
        \\
        &= \frac{
        \vBrace{I+\rBrace{\vecnot{d}\vecnot{\alpha}^{H}-\vecnot{\alpha}^{H}\vecnot{d}}g\exp{\rBrace{-j\omega\tau}}}\vecnot{d}
        }{
        1-g\vecnot{\alpha}^{H}\vecnot{d}\exp{\rBrace{-j\omega\tau}}
        }
        \\
        &= \frac{
        \vecnot{d}+\rBrace{\vecnot{d}\vecnot{\alpha}^{H}\vecnot{d}-\vecnot{\alpha}^{H}\vecnot{d}\vecnot{d}}g\exp{\rBrace{-j\omega\tau}}
        }{
        1-g\vecnot{\alpha}^{H}\vecnot{d}\exp{\rBrace{-j\omega\tau}}
        }
        \\
        &= \frac{
        \vecnot{d}+\vecnot{\alpha}^{H}\vecnot{d}\rBrace{\vecnot{d}-\vecnot{d}}g\exp{\rBrace{-j\omega\tau}}
        }{
        1-g\vecnot{\alpha}^{H}\vecnot{d}\exp{\rBrace{-j\omega\tau}}
        }
        \\
        &= \frac{
        \vecnot{d}
        }{
        1-g\vecnot{\alpha}^{H}\vecnot{d}\exp{\rBrace{-j\omega\tau}}
        }
    \end{split}
\end{equation*}
leading to
$$
\F{\vx}
=
\frac{    
g\vd\exp{\rBrace{-j\phi}}
}{
1 - g\aHd{}\exp{\rBrace{-j\phi}}
}\F{s}.
$$
Let $z=\vBetaH{}\vecnot{x}+\text{n}$ be the beamformer's output (see Fig.~\ref{fig:Proposed_spatialIIR_ARCH}), with Fourier transform $Z$. Considering the noiseless case $\rBrace{\text{i.e., n}=0}$, the frequency response of the FB is 
\begin{equation}
\label{eqn:GeneralFeedbackTransferFunction}
\Hba
\triangleq
\frac{\F{z}}{\F{s}} 
=
\frac{    
g\bHd{}\exp\rBrace{-j\phi}
}{
1 - g\aHd{}\exp\rBrace{-j\phi}
}.
\end{equation}
\par Note that this architecture achieves a controllable (via setting of $\vBeta$ and $\vAlpha$) and recursive (non-trivial denominator) spatial response.
As will be shown, high directivity and narrow beamwidth are obtainable by a proper selection of the weights. Compared to traditional beamformers (i.e., without feedback), the performance improvement will be expressed in terms of increased aperture, narrower beamwidth and improved sidelobe attenuation.
One may observe that opposed to traditional beamformers, the array response, $\Hba,$ is not only influenced by the impinging signal DOA, since it is also range selective due to its $\phi$ dependency.
As demonstrated in Fig.~\ref{fig_rangeAzimuthSelectivity}, the combination of both angular and range selectivity enables the designer to enhance signals arriving from specific locations (grey area) rather than only specific directions.
\begin{figure}[t!]
    \begin{center}
        \begin{overpic}[width=0.55\linewidth, 
        % grid, 
        tics=10,trim=0 0 0 0]{./Media/azimuthRangSelectivity.png}
            \put (20, 23){\rotatebox{0}{\footnotesize{Angular response}}}
            \put (32, 47){\rotatebox{0}{\footnotesize{Enhanced radial slice}}}
        \end{overpic}
    \end{center}
    \caption{
    % A visualization of the spatial area selectivity concept.
    Combining both radial selectivity and DOA-based selectivity allows to localize the target.
    }
    \label{fig_rangeAzimuthSelectivity}
\end{figure}
\section{Fisher Information Matrix}
\label{sec_FIM}
A possible evaluation for the contribution of the presented feedback mechanism is to measure the additional information in the system.
To this end, the FIM (see App.~\ref{apdx:FIM}), denoted by $\vecnot{J}$, will now be calculated with respect to the DOA $\rBrace{\thetaD}$ and range $\rBrace{\phi}$ parameters. 
As the feedback-based transfer function (\ref{eqn:GeneralFeedbackTransferFunction}) is expressed in the frequency domain, we rely on \cite{zeira1990frequency} to express the frequency domain FIM as well. 
\par The $\vBrace{k,l}$'th FIM element, may be expressed as
\begin{equation}\label{eq_FIM_kl_full}
    %\resizebox{.9\linewidth}{!}
    {
        \begin{split}
            J_{k,l}\rBrace{\vEta} 
            =&
            \Re\cBrace{
            \frac{1}{2\pi}
            \int_{-\omega_{s}/2}^{\omega_{s}/2}
            {
            \frac{1}{\Phi\rBrace{\omega}}
            \mathfrak{F}^{*}\left\{
            \frac{\partial z(t)}{\partial\eta_{k}}
            \right\}
            \mathfrak{F}\left\{
            \frac{\partial z(t)}{\partial\eta_{l}}
            \right\}
            d\omega
            }}
            \\ &+
            \frac{T}{4\pi}
            \int_{-\omega_{s}/2}^{\omega_{s}/2}
            \frac{1}{\Phi^{2}\rBrace{\omega}}
            \rBrace{\frac{\partial\Phi\rBrace{\omega}}{\partial\eta_{k}}}^{\ast}
            \frac{\partial\Phi\rBrace{\omega}}{\partial\eta_{l}}
            d\omega
        \end{split}
    }
\end{equation}
where $ \vEta = [\thetaD,\phi]^{T} $ is the parameters vector, $\Re$ stands for the real-part extraction operator, $k,l \in\cBrace{1,2}$, $\Phi\rBrace{\omega}$ is the noise spectrum, $\mathfrak{F}$ is the Fourier transform operator, $T$ is the measurement observation interval and $\omega_{s}$ is the signal bandwidth. 
For simplicity, $\text{n}\rBrace{t}$ is assumed to be white Gaussian with some constant power spectral density $\Phi(\omega)=\sigma^2$ and independent of the estimated parameters $\vEta$. Hence, the second term in the RHS of \eqref{eq_FIM_kl_full} vanishes. 
Assuming continuously differentiable functions, where order alteration of the Fourier transform and the differentiation operations is allowed, \eqref{eq_FIM_kl_full} simplifies to
\begin{equation}
    \label{eq_beamPatternFreqDomain_FIM}
    % \resizebox{1\linewidth}{!}{
        \begin{split}
            J_{kl}\rBrace{\vEta} = 
            \Re\cBrace{
            \frac{1}{2\pi\sigma^2}
            \int_{-\omega_{s}/2}^{\omega_{s}/2}
            {
            \rBrace{\frac{\partial{}\F{z}\rBrace{\omega}}{\partial\eta_{k}}}^{\ast}
            \frac{\partial{}\F{z}\rBrace{\omega}}{\partial\eta_{l}}
            d\omega
            }}
        \end{split}.
    % }
\end{equation}
Expressing the steering vector derivative with respect to $\thetaD$, results in
\begin{equation}\label{eq_vdDiff}
\frac{\partial\vd}{\partial\thetaD}=\vecnot{A}\vd
\end{equation}
where $\vecnot{A}$ is an $N\times{}N$ diagonal matrix with
\[
A_{ii}=-j\omega\frac{\partial \tau_{i}}{\partial{\thetaD}}\ \  \forall{i\in\cBrace{0\hdots{}N-1}}.
\]
It is worth mentioning that \eqref{eq_vdDiff} is relevant even for arbitrary arrays (not necessarily ULA) when smooth and slowly changing radiation patterns are assumed.
In App.~\ref{apdx_clacFim} we compute the FIM terms, concluding that
\begin{equation}
    \label{eqn_FIMelements}
    % \resizebox{.91\linewidth}{!}
    {
        \begin{split}
            &J_{11}=J_{\thetaD\thetaD}
            =
            \frac{1}{2\pi\sigma^{2}}\int_{-\omega_{s}/2}^{\omega_{s}/2}{\frac{
            \lBrace{g\vBetaH{}\vecnot{A}\vd-g^{2}\vBetaH{}B\vAlphaC\ePhi{-}}^{2}
            }{
            \lBrace{1-g\aHd\ePhi{-}}^{4}
            }\lBrace{\F{s}\rBrace{\omega}}^{2}d\omega}
            \\
            &J_{22}=J_{\phi\phi}
            =
            \frac{1}{2\pi\sigma^{2}}\int_{-\omega_{s}/2}^{\omega_{s}/2}{\frac{
            \lBrace{g\bHd}^{2}
            }{
            \lBrace{1-g\aHd\ePhi{-}}^{4}
            }\lBrace{\F{s}\rBrace{\omega}}^{2}d\omega}.
        \end{split}
    }
\end{equation}
where $\vecnot{B}\triangleq\vd\vdT{}\vecnot{A}-\vecnot{A}\vd\vdT.$ 
% Generalizing the conventional beamformer (CB) \cite{van2004optimum}, such that the feedback weights are
% \begin{equation}\label{eq:alpha_beta_opt}
% \vAlpha_{\text{CB,opt}}=\frac{\vdC\exp\rBrace{j\phi}}{\hat{g}\norm{\vd}^2}
% \end{equation}
% where $\hat{g}$ is the channel gain estimate, both maximizes the FIM diagonal elements (via denominator minimization) and nulls it's cross-terms (see App.~\ref{apdx_clacFim}), thus significantly increases the available information.
% Assuming perfect knowledge of the target's location, while
Aiming to maximize the FIM diagonal elements via denominator (i.e., $\lBrace{1-g\aHd\ePhi{-}}$) minimization, the optimal feedback weights are 
\begin{equation}\label{eq:alpha_beta_opt}
\vAlphaC_{\text{CB,opt}}=\frac{\vdC\exp\rBrace{j\phi}}{\hat{g}\norm{\vd}^2},
\end{equation}
where $\hat{g}$ is the channel gain estimate.
This choice of weights may be interpreted as a generalized version of the \emph{conventional beamformer} (CB) \cite{van2004optimum}. 
Furthermore, setting $\vBeta=\vBeta_{\text{CB,opt}}=\vAlpha_{\text{CB,opt}}$, is shown (see App.~\ref{apdx_clacFim}) to nullify the FIM cross terms, such that $J_{12}=J_{\thetaD,\phi}=J_{21}=J_{\phi\thetaD}=0.$
\par Note that setting the feedback weights as in \eqref{eq:alpha_beta_opt} requires perfect knowledge of the target's range, since $\phi$ is range dependant.
Also, the reader may notice that   assuming $\hat{g}=g$, this choice of optimal weights nullifies the denominator of  \eqref{eqn:GeneralFeedbackTransferFunction}. Thus, theoretically, the FIM becomes infinite when the transfer function \eqref{eqn:GeneralFeedbackTransferFunction} is unstable due to positive and coherent feedback between the beamformer and the target. 
In practice, though, there will be unavoidable errors, and perfect knowledge of target's location and the channel gain is usually unknown.
In Sec.~\ref{sec_Performance}, we quantify the effect of such errors and discuss its influence on the array performance. 
\section{Temporal Stability}
\label{sec_stability}
As a preliminary to the following performance discussion in Sec.~\ref{sec_Performance}, we start with simple temporal simulations of the feedback beamformer.
In this section, we start by identifying the unknown parameters in the experimental scenario and formulate their related error terms which are then used to define some basic simulation scenarios that will be thoroughly investigated throughout this chapter.  
After the error terms have been defined, we first address the temporal stability of the system and discuss its correspondence with the system's response which was found in \eqref{eqn:GeneralFeedbackTransferFunction}.
\subsection{Error terms}
In the absence of accurately known parameters, we denote $\hat{\phi},\hat{\theta}$ to be the range and DOA related phase estimates respectively.
Then, using the same weights as in \eqref{eq:alpha_beta_opt} for both the feedback and output synthesis gives rise to
\begin{equation}\label{eq:alpha_beta_hat}
\vBetaCFB=\vAlphaCFB=\frac{\vdHatC\exp\rBrace{j\hat{\phi}}}{\hat{g}\norm{\vdHat}^2},
\end{equation}
introducing the estimated steering vector 
\begin{equation}\label{eq:d_hat}
\vdHat=\vBrace{1,\exp(-\hat\theta),\ldots,\exp(-(N-1)\hat\theta)}^T.
\end{equation}
Plugging \eqref{eq:alpha_beta_hat} into \eqref{eqn:GeneralFeedbackTransferFunction}, results in
\begin{equation}\label{eq:SF_CB}
    %\resizebox{.894\linewidth}{!}
    {
        \begin{split}
            \HbaFB=\frac{r\D{\dTheta/2}{N}\exp\rBrace{-j\rBrace{\dPhi+(N-1)\dTheta/2}}}{1-r\D{\dTheta/2}{N}\exp\rBrace{-j\rBrace{\dPhi+(N-1)\dTheta/2}}}
        \end{split}
    }
\end{equation}
where \[
\D{x}{N}\triangleq\frac{1}{N}\frac{\sin\rBrace{Nx}}{\sin\rBrace{x}}
\]
is the normalized Dirichlet kernel and
\[
\dTheta\triangleq\theta-\hat{\theta},\ \dPhi\triangleq\phi-\hat{\phi},\ 
r\triangleq g/\hat{g},
\]
are defined as the DOA, range and gain error terms respectively.
In the following, four fundamental scenarios are considered:
\begin{itemize}
    \item{\makebox[.35\linewidth]{Perfect alignment \hfill} $\rBrace{\dTheta=0\ , \dPhi=0},$}
    \item{\makebox[.35\linewidth]{Steering error \hfill} $\rBrace{\abs{\dTheta}>0\ , \dPhi=0},$}
    \item{\makebox[.35\linewidth]{Range error \hfill} $\rBrace{\dTheta=0\ , \abs{\dPhi}>0},$}
    \item{\makebox[.35\linewidth]{General \hfill} $\rBrace{\abs{\dTheta}>0\ , \abs{\dPhi}>0}.$}
\end{itemize}
\subsection{Temporal Simulations}
The analogy of the feedback beamformer to the temporal IIR architecture raises some fundamental questions that should be addressed.
In the following, supported by simulations, we answer those questions.
\par The first question, related to the feedback beamformer's analogy to the temporal IIR architecture, is stability related.
To begin with, we should first ask what does ``stability`` means.
Interestingly, the system's response of \eqref{eqn:GeneralFeedbackTransferFunction} is not frequency dependent, for it is exclusively governed by spatial parameters.
Therefore, in the absence of any previous related work on spatial feedback, we find that the commonly used Bounded-Input-Bounded-Output (BIBO) stability is appropriate. 
As such, assuming the \emph{Perfect alignment} scenario, we identify $r$ in \eqref{eq:SF_CB} to be the most influential for determining the system's stability for it controls the response's denominator - e.g. for $r<1$ the denominator will never be nullified.
As anticipated from \eqref{eq:SF_CB} and confirmed with the simulations presented in Fig.\ref{fig_stability}, for $r>1$ the system is not stable, for the received amplitude does not converge to a finite value.
Also noticeable is the increase in the final amplitude value as $r$ increases.
In Fig.~\ref{fig_stabilityVal}, we plot the final amplitude value with respect to $r$ and find that the increase is substantial as $r$ increases towards 1.
This issue will have to be taken into consideration when one designs such an array for practical applications but is outside the scope of this work.
\begin{figure}[t!]
    \begin{center}
        \begin{overpic}[width=0.55\linewidth, 
        %grid, 
        tics=10,trim=0 0 0 0]{./Media/fig_stabilization.png}
            \put (46, -5){\rotatebox{0}{$t/2\tau_{pd}$}}
            \put (-6, 38){\rotatebox{90}{$\abs{z\rBrace{t}}$}}
            \put (46, 18.5){\rotatebox{0}{$r=0.6$}}
            \put (46, 35){\rotatebox{0}{$r=0.9$}}
            \put (46, 54){\rotatebox{0}{$r=1$}}
            \put (18, 60){\rotatebox{0}{$r=1.1$}}
            \put (0.75, 80){\rotatebox{0}{dB}}
        \end{overpic}
    \end{center}
    \caption{
    Simulating a 3 element ULA based feedback beamformer, steered to a target which resides in the direction $\theta_{s} = \pi/2$ and plotting temporal response (dB) for multiple $r$ values.
    The horizontal axis ($t/2\tau_{pd}$) is the time, normalized to the signal's round-trip duration to the target and back.
    }
    \label{fig_stability}
\end{figure}
As a final observation, we address the ``discrete`` behaviour of $z$, i.e. the value (most noticeable in the $r=1.1$ plot) seems to be piecewise constant.
This phenomenon corresponds with the propagation latency of the signal's round trip to the target and back.
In each iteration the spatial focusing is increased, together with the received amplitude, as can be seen in Fig.~\ref{fig_beamThinning}.
The reader may notice that the $t=1\cdot{}\tau_{pd}$ plot is actually the conventional beampattern, for the feedback signal has not yet reimpinged the array.
\begin{figure}[t!]
    \begin{center}
        \begin{overpic}[width=0.55\linewidth, 
        %grid, 
        tics=10,trim=0 0 0 0]{./Media/fig_beamThinning.png}
            \put (50, -5){\rotatebox{0}{$\theta_{d}/\pi$}}
            \put (-6, 38){\rotatebox{90}{$H_{r=0.9}$}}
            \put (100, 49){\rotatebox{0}{$t=1\cdot\tau_{pd}$}}
            \put (100, 39){\rotatebox{0}{$t=2\cdot\tau_{pd}$}}
            \put (100, 26.5){\rotatebox{0}{$t=5\cdot\tau_{pd}$}}
            \put (100, 14){\rotatebox{0}{$t=10\cdot\tau_{pd}$}}
            \put (53.5, 27.5){\rotatebox{-75}{$t=50\cdot\tau_{pd}$}}
            \put (4.5, 80){\rotatebox{0}{dB}}
        \end{overpic}
    \end{center}
    \caption{
    Simulation of same setup as in Fig.\ref{fig_stability}.
    We set $r=0.9$ and plot the beampattern for various simulation times.
    }
    \label{fig_beamThinning}
\end{figure}
\begin{figure}[t!]
    \begin{center}
        \begin{overpic}[width=0.55\linewidth, 
        %grid, 
        tics=10,trim=0 0 0 0]{./Media/fig_stabilizationVal.png}
            \put (0, 82){\rotatebox{0}{dB}}
            \put (-6, 30){\rotatebox{90}{$\lim_{t\to\infty}\abs{z\rBrace{t}}$}}
            \put (51, -5){\rotatebox{0}{$r$}}
        \end{overpic}
    \end{center}
    \caption{
    Simulation of same setup as in Fig.\ref{fig_stability}.
    For each $r$ (the horizontal axis), we plot the final received amplitude in dB.
    }
    \label{fig_stabilityVal}
\end{figure}
\par The Second question to be asked regards the settling time of the system, for in Fig.~\ref{fig_stability} we easily observe that the settling time of $r=0.6$ (i.e. $\sim{}10$ signal round trips) is substantially shorter than its matching value ($\sim{}50$) when $r=0.9$.
To better understand the phenomenon, we plot the settling time's $r$ dependency in Fig.~\ref{fig_stabilityDur}.
Although outside of this work's scope, this issue will obviously be of great importance when considering dynamic targets, as will be discussed in the ``future research`` part of the concluding Chapter.~\ref{chap:futureResearch}.
\begin{figure}[t!]
    \begin{center}
        \begin{overpic}[width=0.55\linewidth, 
        %grid, 
        tics=10,trim=0 0 0 0]{./Media/fig_stabilizationDur.png}
            \put (-3, 35){\rotatebox{90}{$t/2\tau_{pd}$}}
            \put (51, -5){\rotatebox{0}{$r$}}
        \end{overpic}
    \end{center}
    \caption{
    Simulation of same setup as in Fig.\ref{fig_stability}.
    For each $r$ (the horizontal axis), we plot the time (the vertical axis) where the received amplitude entered the $1\%$ sleeve around its final value.
    The plotted time is also normalized as in Fig.\ref{fig_stability}.
    }
    \label{fig_stabilityDur}
\end{figure}
\par Although many interesting questions rise from the initial simulations, most of them are outside the scope of this work.
We choose to add one more note, regarding the influence of the array's number of elements.
Although in spatial processing literature, enlarging the number of elements in the array commonly increases the spatial performance, both examining \eqref{eq:SF_CB} and simulating the experimental scenarios show that it (increase of elements) does not affect the temporal performance of the system - i.e. the settling time remains the same as $N$ increases.
\section{Performance Analysis}
\label{sec_Performance}
In this section we analyze the suggested FB (see Fig.~\ref{fig:Proposed_spatialIIR_ARCH}), considering some fundamental properties which are commonly used to asses array performance and discussed in Sec.~\ref{sec:prlm_array_perf}: Beamwidth, peak-to-sidelobe level, and directivity. Each property is then compared to traditional passive ULAs, showing that significantly improved performances are obtainable with spatial feedback integration.
\subsection{The normalized beampattern}
\label{subsection_spatialIIR_normBP}
Applying the \emph{normalized beampattern} concept of Sec.~\ref{sec:prlm_array_perf} to this work, we set $\vecnot{\beta}_{\coefSetName,\text{opt}}=\vecnot{\alpha}_{\coefSetName,\text{opt}},$ giving rise to
\begin{equation}
    \label{eq_narmalized_pattern}
    %\resizebox{.89\linewidth}{!}{
    \begin{split}
        \HrTPr&\triangleq
        \frac{
        \HbaFB
        }{
        H_{\vecnot{\beta}_{\coefSetName,\text{opt}},\vecnot{\alpha}_{\coefSetName,\text{opt}}}
        }\\
        &=
        \frac{
        \HbaFB
        }{
        r/\rBrace{1-r}
        }\\
        &=
        \frac{\rBrace{1-r}\D{\dTheta/2}{N}}{\exp\rBrace{j\rBrace{\dPhi+(N-1)\dTheta/2}}-r\D{\dTheta/2}{N}}.
    \end{split}
\end{equation}
Note that the known \cite{van2004optimum} normalized response of standard ULA is obtained by setting $r=0$ and $\Delta\phi=0$
$$
\Hr_{\Delta\theta,\Delta\phi=0,r=0}=
             \D{\dTheta/2}{N}\exp\rBrace{-j\rBrace{(N-1)\dTheta/2}}.
$$
Considering the steering error scenario (i.e., $\dTheta\neq0,\ \dPhi=0$) first, where 
\begin{equation}\label{eq_Hdphi0}
\Hr_{\Delta\theta,\Delta\phi=0,r}=
             \frac{\rBrace{1-r}\D{\dTheta/2}{N}}{\exp\rBrace{j\rBrace{(N-1)\dTheta/2}}-r\D{\dTheta/2}{N}},
\end{equation}
we evaluate the FB's beamwidth, sidelobe level and directivity and compare them to those of the standard ULA.
\subsection{Half power beamwidth}
% The half power beamwidth (HPBW), denoted as $\dThetaHPBW$ quantifies the array's main lobe narrowness.
% It represents the DOA where the beampattern's energy reduces to half of its maximal value.
\par In App.~\ref{apdx_HPBW} we extend the known result of \eqref{eq_known_HBPW} for any $r\geq 0$. It turns out that for large $N$, the HPBW is obtained by solving for $x$ the equality
\begin{equation}\label{eq_HPBW}
    % \resizebox{1\linewidth}{!}{
        \begin{split}
            \rBrace{r^{2}-4r+2}\frac{\sin{\rBrace{x}}^{2}}{x^{2}}+r\frac{\sin{\rBrace{2x}}}{x}-1=0
        \end{split}
    % }
\end{equation}
where we define $x\triangleq{} N\dTheta_{HPBW}/2$. In Fig.~\ref{fig_feedbackULA_HPBW_Nx_vs_N_variousR} we plot the numerical solution of \eqref{eq_HPBW} for various values of $r$ and $N$, showing that $x$ reaches its limit around $N=20$. Also note that for $r=0$ we obtain the known result of standard ULA with the limiting factor of $1.4$.
Having the limiting factors for various values of the gain mismatch $r$, we investigate the feedback related improvement and express the HPBW by
\[
\dThetaHPBW/2\approx \frac{1.4}{f(r)N}.
\]
Note that $f(r)$ represents the array aperture improvement factor, compared to the standard ULA.
To find a suitable expression for $f\rBrace{r}$, we computed $\Hr_{\Delta\theta,\Delta\phi=0,r}$ using $N=100$ on a fine grid of $r$ values and looked for proper polynomial approximation (using MATLAB\textsuperscript{\textregistered}) for the HPBW.
It turns out that the 2-nd order polynomial fit satisfies
\begin{equation}
    \label{eq_Bapprox}
    f\rBrace{r}\approx\frac{1.4}{\rBrace{1-r}\rBrace{-0.4r+1.4}}.
\end{equation}
Note that for an accurate gain match (i.e., $r\to1$), the RHS of \eqref{eq_Bapprox} tends towards infinity, implying that the equivalent array has an infinite number  of elements ($f\rBrace{r}N$), hence obtaining perfect spatial selectivity (see Fig.~\ref{fig_feedbackULA_beamwidth_limit_r_dependent}).
\begin{figure}[t]
    \begin{center}
        \begin{overpic}[width=0.65\linewidth, 
        %grid, 
        tics=10,trim=0 0 0 0]{./Media/spatial_IIR_MATLAB/arrayParameters/HPBW_vs_N_various_r.eps}
            \put (4, 75){\footnotesize{$N\dThetaHPBW/2$}}
            \put (50, 62.5) {\footnotesize{$r=0$}}
            \put (50, 54) {\footnotesize{$r=0.1$}}
            \put (50, 39.5) {\footnotesize{$r=0.3$}}
            \put (50, 28.5) {\footnotesize{$r=0.5$}}
            \put (50, 19.75) {\footnotesize{$r=0.7$}}
            \put (50, 12.5) {\footnotesize{$r=0.9$}}
            \put (50, 2) {\footnotesize{$N$}}
        \end{overpic}
    \end{center}
     \caption{Plot of $x=N\dThetaHPBW/2$ vs. $N$, for various $r$ values, obtained by numerically solving \eqref{eq_HPBW}.}
    \label{fig_feedbackULA_HPBW_Nx_vs_N_variousR}
\end{figure}
\begin{figure}[t]
    \begin{center}
        \begin{overpic}[width=0.65\linewidth, 
        % grid, 
        tics=10,trim=0 0 0 0]{./Media/HPBW_limit_vs_r.eps}
            \put (39.5, 63.5) {\scriptsize{Numerical solution of \eqref{eq_HPBW}}}
            \put (39.5, 58.25) {\scriptsize{Polynomial fitting \eqref{eq_Bapprox}}}
            \put (39.5, 52.5) {\footnotesize{$\log_{10}f\rBrace{r}$}}
            \put (85, 75) {\footnotesize{$\log_{10}f\rBrace{r}$}}
            \put (4, 75){\footnotesize{$N\dThetaHPBW/2$}}
            \put (50, 2) {\footnotesize{$r$}}
        \end{overpic}
    \end{center}
    \caption{Evaluation of $N\dThetaHPBW/2$ for $N=100$ and its approximation $1.4/f\rBrace{r}$  (marked by red diamonds). $f\rBrace{r}$ is also presented, in logarithmic scale (dotted curve).} 
    \label{fig_feedbackULA_beamwidth_limit_r_dependent}
\end{figure}
\subsection{Sidelobes attenuation}
\ifdefined\showDev
    \fbox{
    \begin{minipage}{0.9\linewidth}
    \textbf{development specifics}\\
    Let $f\rBrace{\dTheta} \triangleq \D{\dTheta/2}{N}\exp\rBrace{-j\rBrace{(N-1)\dTheta/2}},$ such that $\Hr_{\Delta\theta,\Delta\phi=0,r=0} = \frac{f\rBrace{\dTheta}}{1-rf\rBrace{\dTheta}}.$ Then, we compute the beampattern's derivative and state that
    \begin{equation*}
    % \resizebox{0.9\linewidth}{!}{
        \begin{split}
            &\frac{\partial}{\partial\dTheta}\Hr_{\Delta\theta,\Delta\phi=0,r} &\\
            &=\frac{\Dp{\dTheta/2,N}{'}\rBrace{1-r\D{\dTheta/2}{N}}-\D{\dTheta/2}{N}\rBrace{-r\Dp{\dTheta/2,N}{'}}}{\rBrace{1-r\D{\dTheta/2}{N}}^{2}}
            \\
            &=\frac{\Dp{\dTheta/2,N}{'}}{\rBrace{1-r\D{\dTheta/2}{N}}^{2}}.
        \end{split}
        % }
    \end{equation*}
    It follows that the sidelobes of the feedback-based beampattern are located at the same angles as the in the ULA case.
    \end{minipage}
    }
\else
\fi
By taking a derivative of $\Hr_{\Delta\theta,\Delta\phi=0,r}$ with respect to $\dTheta$ it can be easily verified that the beampattern's extrema points are located exactly as in the standard ULA beampattern. Specifically, the sidelobes locations are
\begin{equation}
    \label{eqn_CB_sidelobesLocations}
    \Delta\theta_{\text{sidelobe}} = \frac{\rBrace{2m+1}\pi}{N}\ \forall m\in\cBrace{\pm 1,\pm 2,\hdots}.
\end{equation}
Our main interest is with the first sidelobe (i.e. $m=1$), therefore we evaluate \eqref{eq_Hdphi0} at $\Delta\theta = 3\pi/N$, which results in
\begin{equation}
    \abs{\Hr_{{3\pi}/{N},0,r}}^2
    =
    \frac{
    2\rBrace{1-r}^{2}
    }{
    \rBrace{N^{2}-2Nr}\rBrace{1-\cos{\rBrace{\frac{3\pi}{N}}}}+2r^{2}
    }
    \label{eq_HSidelobes}
\end{equation}
and for large $N$ values 
\begin{equation*}
    \lim_{N\rightarrow\infty}\abs{\Hr_{3\pi/N,0,r}}=\frac{
    2\rBrace{1-r}
    }{
    3\pi
    }.
\end{equation*}
\ifdefined\showDev
    \fbox{
    \begin{minipage}{0.9\linewidth}
    \textbf{development specifics}\\
    We know that \eqref{eq_Hdphi0} can be rewritten as
    \begin{equation*}
        \begin{split}
            \Hr_{\Delta\theta,\Delta\phi=0,r=0} =
            \frac{
            \rBrace{1-\cos{\rBrace{N\dTheta}}}\rBrace{1-r}^{2}
            }{
            \begin{split}
                N^{2}&\rBrace{1-\cos{\dTheta}}+r^{2}\rBrace{1-\cos{N\dTheta}}
                \\
                &+Nr\Bigg(1+\cos{\rBrace{\rBrace{N-1}\dTheta}}
                \\
                &-\cos{\rBrace{N\dTheta}-\cos{\rBrace{\dTheta}}}\Bigg)
            \end{split}
            }
        \end{split}
    \end{equation*}
    Using the symbolic toolbox in MATLAB, setting $\dTheta=3\pi/N$, results in \eqref{eq_HSidelobes}.
    \end{minipage}
    }
\else
\fi
\par For standard ULA, the gain of the first sidelobe is known to be $2/3\pi$ \cite{van2004optimum}, which implies that the first sidelobe is smaller by a factor of $1-r$ compared to standard ULA.
Specifically, in perfect gain match scenario (i.e., $r\to{}1$) , the sidelobes vanish. 
\subsection{Array directivity}
In the context of this work, the directivity is expressed as
\begin{equation}\label{eq_D}
    \mathcal{D}\rBrace{N,r} = \frac{\Hr_{\Delta\theta=0,\Delta\phi=0,r}}{\frac{1}{2\pi}\int_{0}^{2\pi}\Hr_{\Delta\theta,\Delta\phi=0,r}\ d\Delta\theta} = \frac{2\pi}{\int_{0}^{2\pi}\Hr_{\Delta\theta,\Delta\phi=0,r}\ d\Delta\theta},
\end{equation}
Plugging \eqref{eq_Hdphi0} within \eqref{eq_D} and by numerical evaluation (see App.~\ref{apndx_directivityFit} and the complementary Fig.~\ref{fig_directivity}) we suggest to approximate the directivity with 
\begin{equation}\label{eq_D_result}
    \mathcal{D}\rBrace{N,r} \approx \frac{N-r}{1-r},
\end{equation}
where the standard ULA's known result is obtained for $r=0$.
Also, for $N\geq2$, $\lim_{r\rightarrow 1}\mathcal{D}\rBrace{N,r}=\infty$, implying infinite directivity for the perfectly gain-matched FB. 
\par Finally, expressing the improvement in directivity compared to the standard ULA, assuming large $N$ values, gives rise to
\begin{equation}\label{eq_Dimprovement}
\lim_{N\to\infty}\frac{\mathcal{D}\rBrace{N,r}}{\mathcal{D}\rBrace{N,0}}
=\frac{N/\rBrace{1-r}}{N}=\frac{1}{1-r}.
\end{equation}
\begin{figure}[t]
    \begin{center}
        \begin{overpic}[width=0.75\linewidth, 
        %grid, 
        tics=10,trim=0 0 0 0]{./Media/directivity_expr.eps}
            \put (20, 10){\footnotesize{$N$}}
            \put (72, 6){\footnotesize{$r$}}
            \put (16.5, 70){\footnotesize{Numerical integration}}
            \put (16.5, 65){\footnotesize{$\rBrace{N-r}/\rBrace{1-r}$}}
            \put (-9, 45){\footnotesize{$\mathcal{D}\rBrace{N,r}$}}
        \end{overpic}
    \end{center}
     \caption{Plot of $\mathcal{D}\rBrace{N,r}$, computed using numerical integration (surface), shown to perfectly match the analytic expression (black diamonds) presented in \eqref{eq_D_result}.}
    \label{fig_directivity}
\end{figure}
\subsection{Summary}
To conclude this section, we summarize the feedback integration related performance improvements in Table.~\ref{table_arrayPerformance}.
\begin{table}[h!]
    \caption{Performances of Classical ULA and the Proposed Feedback-Beamforming Architecture, with a Gain Mismatch $r$.}
    \centering
    %\resizebox{1\linewidth}{!}
    {
        \begin{tabular}{||c c c c||}
            \hline
            & ULA & \thead{FEEDBACK\\BEAMFORMING} & IMPROVEMENT \\ [0.5ex] 
            \hline\hline
            HPBW & $ 1.4/N$ & $1.4/\rBrace{f(r)N}$ & Narrower by a factor of $f\rBrace{r}$\\ 
            \thead{FIRST\\SIDELOBE\\GAIN} & $2/3\pi$ & $2\rBrace{1-r}/3\pi$ & \thead{smaller by a factor of $1-r$\\for $N\gg{}1$} \\
            DIRECTIVITY & $N$ & $\rBrace{N-r}/\rBrace{1-r}$ & \thead{$1/\rBrace{1-r}$ times higher \\ for $N\gg{}1$}\\
            [1ex] 
            \hline
         \end{tabular}
     }
    \label{table_arrayPerformance}
\end{table}
Also, in Fig.~\ref{fig_perf} we demonstrate the HPBW and sidelobe attenuation for 3 elements ULA.
As the expressions for HPBW and sidelobe-attenuation are relevant for $\sim{}N>20$, we also simulate 20 elements ULA for in Fig.~\ref{fig_perf20}. 
In Table.~\ref{table_arrayPerfEmp} we show the coherency between the results and the performance related expressions of Table.~\ref{table_arrayPerformance}.
As predicted, the results of Fig.~\ref{fig_perf20} are consistent with the theoretical expressions.
\begin{figure}[t]
    \begin{center}
        \begin{overpic}[width=0.9\linewidth, 
        %grid, 
        tics=10,trim=0 0 0 0]{./Media/fig_performance.png}
            \put (9, 36.5){$dB$}
            \put (2, 20){$\abs{H}^{2}$}
            \put (27, 0){$\theta_{d}/\pi$}
            \put (34.5, 67){$r=0$}
            \put (29, 63){\rotatebox{-90}{\tiny{HPBW = $0.1\pi$}}}
            \put (40, 61){\tiny{$-19.08_{dB}$}}
            \put (77.5, 67){$r=0.3$}
            \put (73.14, 63){\rotatebox{-90}{\tiny{HPBW = $0.073\pi$}}}
            \put (83.5, 58.5){\tiny{$-23.43_{dB}$}}
            \put (34.5, 31.5){$r=0.6$}
            \put (29, 27){\rotatebox{-90}{\tiny{HPBW = $0.043\pi$}}}
            \put (40, 18.75){\tiny{$-31.09_{dB}$}}
            \put (77.5, 31.5){$r=0.8$}
            \put (73.14, 27){\rotatebox{-90}{\tiny{HPBW = $0.02\pi$}}}
            \put (84.25, 13.5){\tiny{$-41.6_{dB}$}}
        \end{overpic}
    \end{center}
     \caption{Simulating 3 elements ULA based feedback beamformer for $r$ values of 0, 0.3, 0.6 and 0.8.
     The HPBW is marked with vertical red dashed line, where an auxiliary horizontal line of $\abs{H}^{2} = 1/2$ is also provided.
     The sidelobe attenuation is cited in each plot in a textual fashion.}
    \label{fig_perf}
\end{figure}
\begin{figure}[t]
    \begin{center}
        \begin{overpic}[width=0.9\linewidth, 
        %grid, 
        tics=10,trim=0 0 0 0]{./Media/fig_performance20.png}
            \put (9, 36.5){$dB$}
            \put (-3, 20){$\abs{H}^{2}$}
            \put (27, 0){$\theta_{d}/\pi$}
            \put (34.5, 67){$r=0$}
            \put (29, 63){\rotatebox{-90}{\tiny{HPBW = $0.0605\pi$}}}
            \put (32.25, 56.5){\tiny{$-26.38_{dB}$}}
            \put (77.5, 67){$r=0.3$}
            \put (73, 63){\rotatebox{-90}{\tiny{HPBW = $0.0454\pi$}}}
            \put (79, 53.75){\tiny{$-32.06_{dB}$}}
            \put (34.5, 31.25){$r=0.6$}
            \put (29, 27){\rotatebox{-90}{\tiny{HPBW = $0.0259\pi$}}}
            \put (40, 14){\tiny{$-41.21_{dB}$}}
            \put (77.5, 31.25){$r=0.8$}
            \put (73, 27){\rotatebox{-90}{\tiny{HPBW = $0.0127\pi$}}}
            \put (84.25, 12.5){\tiny{$-52.55_{dB}$}}
        \end{overpic}
    \end{center}
     \caption{Simulating 20 elements ULA for the sake of HPBW analysis.
     As in Fig.~\ref{fig_perf}, $r$ values of 0, 0.3, 0.6 and 0.8 are simulated.
     The HPBW is marked in the same manner also.}
    \label{fig_perf20}
\end{figure}
\begin{table}[h!]
    \caption{Performances of conventional beamformer $r=0$ and the proposed Feedback-Beamforming Architecture, for several $r$ values.
    Also comparison to expected theoretical expressions is provided.}
    \centering
    %\resizebox{1\linewidth}{!}
    {
        \begin{tabular}{||c | c c c | c c c||}
            \hline
            & \multicolumn{3}{c|}{HPBW [RAD]} & \multicolumn{3}{c||}{SIDELOBE GAIN [dB]} \\ [0.5ex]
            \hline
            Expected & \multicolumn{3}{c|}{$\frac{\rBrace{1-r}\rBrace{-0.4r+1.4}}{1.4}$} & \multicolumn{3}{c||}{$20\log{}\rBrace{\abs{\frac{2\rBrace{1-r}}{3\pi}}^{2}}$} \\ [0.5ex]
            \hline
            & Result & Expected & Error & Result & Expected & Error \\ [0.5ex] 
            \hline\hline
            $r=0$ & 0.0605 & 0.07 & 0.0095 & -26.38 & -26.93 & 0.55 \\ [0.5ex]
            $r=0.3$ & 0.0454 & 0.0448 & 0.006 & -32.06 & -33.12 & 1.06 \\ [0.5ex]
            $r=0.6$ & 0.0259 & 0.0232 & 0.0027 & -41.21 & -42.84 & 1.63 \\ [0.5ex]
            $r=0.8$ & 0.0127 & 0.0108 & 0.0019 & -52.55 & -54.88 & 2.33 \\ [0.5ex]
            \hline
         \end{tabular}
     }
    \label{table_arrayPerfEmp}
\end{table}