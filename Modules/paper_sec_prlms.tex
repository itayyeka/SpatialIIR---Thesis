In this chapter, we revisit the most basic DOA estimation approach, i.e. beamforming, and present our novel feedback based architecture.
The feedback based beamforming concept is introduced in Sec.~\ref{sec_introduceFeedback}.
Then, we apply information theory related reasoning for the array setting in Sec.~\ref{sec_FIM}, namely we use the Fisher Information Matrix.
The resultant beamformer's temporal response and spatial performance are analysed in Sec.~\ref{sec_stability} and Sec.~\ref{sec_Performance} respectively, assuming ideal noiseless scenario.
Considering practical scenarios, the beamformer's noise sensitivity is exemplified in Sec.\ref{sec_sim}, followed by a practical and robust dual-frequency beamformer which is presented in Sec.~\ref{sec_app}.

\section{Notations and problem setup}
\label{sec_introduceFeedback}
In this section, a feedback-based architecture is proposed for spatial signal processing.
Inspired by time domain ``Direct form II'' IIR filter design (see Fig.~\ref{fig_IIR_arch}), 
we propose to use the same concept in the spatial domain.
The suggested feedback beamformer (FB) architecture, where the output signal ($z$) is synthesized using weights $\vBeta$ and the weights $\vAlpha$ synthesize the feedback transmission ($\Tx$), is presented in Fig.~\ref{fig:Proposed_spatialIIR_ARCH}. The beamformer's output and the feedback signal are synthesized using two independently configured beamformers, $s$ is the system's stimulus and an additive noise (n) is assumed at the array's output.
Also, the FB block is marked (dashed line) for later use.
Note that setting $\vAlpha=\vecnot{0}$ (i.e., cancelling the feedback) degenerates the system to a plain delay-and-sum (DS) beamformer.
\begin{figure}[t!]
    \begin{center}
        \begin{overpic}[width=0.5\linewidth, 
        % grid, 
        tics=10,trim={0 0 0 0}]{./Media/SpatialIIR-diagram/SpatialIIR_VER9.png}
            \put (14, 39){\footnotesize{$\sum_{n=0}^{N-1}\beta^{*}_{n}x_{n}$}}
            \put (54, 39){\footnotesize{$\sum_{n=0}^{N-1}\alpha^{*}_{n}x_{n}$}}
            \put (24.25, 50){\footnotesize{$x_{0}$}}
            \put (35, 50){\footnotesize{$x_{1}$}}
            \put (55, 50){\footnotesize{$x_{N-1}$}}
            \put (30.5, 66){\footnotesize{$\delta$}}
            \put (89, 96){\footnotesize{$\vpt$}}
            \put (58, 58){\footnotesize{$\vpi{N-1}$}}
            \put (37, 58){\footnotesize{$\vpi{1}$}}
            \put (26.5, 58){\footnotesize{$\vpi{0}$}}
            \put (41.5, 64.5){\footnotesize{$\theta_{d}$}}
            \put (19, 11.25){\large{$+$}}
            \put (61.25, 27.25){\large{$+$}}
            \put (47, 11.75){$s$}
            \put (35.25, 11.75){$n$}
            \put (21,4){$z$}
            \put (63.5,14){$\Tx$}
            \put (1, 51){$\text{FB}_{\vAlpha,\vBeta}$}
        \end{overpic}
    \end{center}
    \caption{
    The proposed feedback beamformer.
    The spatial feedback is obtained by continuous re-transmission of $\Tx$ to the target at $\vpt$.
    We designate the feedback beamformer block (dashed line) for later use.
    }
    \label{fig:Proposed_spatialIIR_ARCH}
\end{figure}
\subsection{Obtained spatial response}
As in Sec.~\ref{sec:prlm_FIR_IIR}, an $N$-element ULA with inter-element spacing $d$ is considered.
Its $n$'th sensor is positioned at $\vpi{n},\;\text{for}\; n=0,\ldots,N-1$ and we set $\vpi{0}$ as the axis reference point while the target of interest is positioned at $\vpt$.
We also place a transmitter at $\vpi{0}$, where transmitted signal $s$, is reflected back from the target and re-impinges the array, with a total time delay of $\tau_{\text{pd}}=2R/c$ seconds and $R = \norm{\vpt-\vpi{0}}, c$ are the target's range propagation velocity of the signal in the medium respectively.
Time domain analysis of the proposed feedback based architecture, considering both propagation delay and attenuation, gives rise to
\begin{equation}
    \label{eqn:SingleSensorTemporalEquality}
    % \resizebox{.91\linewidth}{!}{
        \begin{split}
            x_{n}(t) = g\vBrace{s\rBrace{t-\tau_{pd}-\tau_{n}}
            +\sum_{m=0}^{N-1}{\alpha^{*}_{m}x_{m}\rBrace{t-\tau_{pd}-\tau_{n}}}},
        \end{split}
    % }
\end{equation}
where the first term on the right-hand-side (RHS) represents the contribution of the transmitted waveform $s(t)$ to the $n$'th array element and the second term represents the feedback contribution of the re-transmitted array signal to this same element.
Expressing the Fourier transform of \eqref{eqn:SingleSensorTemporalEquality},
\begin{equation}
    \label{eqn_singleSensorFourier}
    % \resizebox{.91\linewidth}{!}{
            X_{n} =
            g\cBrace{\F{s}
            \exp\vBrace{-j\omega\rBrace{\tau_{pd}+\tau_{n}}}
            +\sum_{m=0}^{N-1}
            {
            \alpha^{*}_{m}\F{x}_{m}
            \exp\vBrace{-j\omega\rBrace{\tau_{pd}+\tau_{n}}}
            }},
    % }
\end{equation}
and its vector from,
$$
\F{\vx} = g\rBrace{\F{s}+\vAlphaH \F{\vx}}\vd\exp{\rBrace{-j\omega\tau_{pd}}},
$$
we find that it can be simplified to
$$
\F{\vx} =g\rBrace{I-g\vd\vAlphaH{}e^{-j\omega\tau_{pd}}}^{-1}\vd\F{s}\exp{\rBrace{-j\omega\tau_{pd}}}.
$$
Then, denoting
\[
\phi\triangleq\omega\tau_{pd}
\]
as the round-trip signal propagation related electrical phase.
We use the Sherman-Morrison formula \cite{sherman1950adjustment}, considered to be a result of the Woodbury matrix identity \cite{woodbury1950inverting}, stating that
\begin{equation*}
    \rBrace{I+\vecnot{u}\vecnot{v}^{T}}^{-1}=I-\frac{\vecnot{u}\vecnot{v}^{T}}{1+\vecnot{v}^{T}\vecnot{u}},
\end{equation*}
where $\vecnot{u},\vecnot{v}$ are two $N\times1$ vectors and $I$ is the identity matrix.   
Setting $\vecnot{u}=-g\vecnot{d}\exp{\rBrace{-j\omega\tau}}$ and $\vecnot{v}=\vecnot{\alpha}^{*}$ gives rise to
\begin{equation*}
    \begin{split}
        \rBrace{I-g\vecnot{d}\vecnot{\alpha}^{H}\exp{\rBrace{-j\omega\tau}}}^{-1} 
        &= I+\frac{g\vecnot{d}\vecnot{\alpha}^{H}\exp{\rBrace{-j\omega\tau}}}{1-g\vecnot{\alpha}^{H}\vecnot{d}\exp{\rBrace{-j\omega\tau}}}
        \\
        &= \frac{\rBrace{I-g\vecnot{\alpha}^{H}\vecnot{d}\exp{\vBrace{-j\omega\tau}}+g\vecnot{d}\vecnot{\alpha}^{H}\exp{\vBrace{-j\omega\tau}}}\vecnot{d}}{1-g\vecnot{\alpha}^{H}\vecnot{d}\exp{\rBrace{-j\omega\tau}}}
        \\
        &= \frac{
        \vBrace{I+\rBrace{\vecnot{d}\vecnot{\alpha}^{H}-\vecnot{\alpha}^{H}\vecnot{d}}g\exp{\rBrace{-j\omega\tau}}}\vecnot{d}
        }{
        1-g\vecnot{\alpha}^{H}\vecnot{d}\exp{\rBrace{-j\omega\tau}}
        }
        \\
        &= \frac{
        \vecnot{d}+\rBrace{\vecnot{d}\vecnot{\alpha}^{H}\vecnot{d}-\vecnot{\alpha}^{H}\vecnot{d}\vecnot{d}}g\exp{\rBrace{-j\omega\tau}}
        }{
        1-g\vecnot{\alpha}^{H}\vecnot{d}\exp{\rBrace{-j\omega\tau}}
        }
        \\
        &= \frac{
        \vecnot{d}+\vecnot{\alpha}^{H}\vecnot{d}\rBrace{\vecnot{d}-\vecnot{d}}g\exp{\rBrace{-j\omega\tau}}
        }{
        1-g\vecnot{\alpha}^{H}\vecnot{d}\exp{\rBrace{-j\omega\tau}}
        }
        \\
        &= \frac{
        \vecnot{d}
        }{
        1-g\vecnot{\alpha}^{H}\vecnot{d}\exp{\rBrace{-j\omega\tau}}
        }
    \end{split}
\end{equation*}
leading to
$$
\F{\vx}
=
\frac{    
g\vd\exp{\rBrace{-j\phi}}
}{
1 - g\aHd{}\exp{\rBrace{-j\phi}}
}\F{s}.
$$
Let $z=\vBetaH{}\vecnot{x}+\text{n}$ be the beamformer's output (see Fig.~\ref{fig:Proposed_spatialIIR_ARCH}), with Fourier transform $Z$. Considering the noiseless case $\rBrace{\text{i.e., n}=0}$, the frequency response of the FB is 
\begin{equation}
\label{eqn:GeneralFeedbackTransferFunction}
\Hba
\triangleq
\frac{\F{z}}{\F{s}} 
=
\frac{    
g\bHd{}\exp\rBrace{-j\phi}
}{
1 - g\aHd{}\exp\rBrace{-j\phi}
}.
\end{equation}
\par Note that this architecture achieves a controllable (via setting of $\vBeta$ and $\vAlpha$) and recursive (non-trivial denominator) spatial response.
As will be shown, high directivity and narrow beamwidth are obtainable by a proper selection of the weights. Compared to traditional beamformers (i.e., without feedback), the performance improvement will be expressed in terms of increased aperture, narrower beamwidth and improved sidelobe attenuation.
One may observe that opposed to traditional beamformers, the array response, $\Hba,$ is not only influenced by the impinging signal DOA, since it is also range selective due to its $\phi$ dependency.
As demonstrated in Fig.~\ref{fig_rangeAzimuthSelectivity}, the combination of both angular and range selectivity enables the designer to enhance signals arriving from specific locations (grey area) rather than only specific directions.
\begin{figure}[t!]
    \begin{center}
        \begin{overpic}[width=0.55\linewidth, 
        % grid, 
        tics=10,trim=0 0 0 0]{./Media/azimuthRangSelectivity.png}
            \put (20, 23){\rotatebox{0}{\footnotesize{Angular response}}}
            \put (32, 47){\rotatebox{0}{\footnotesize{Enhanced radial slice}}}
        \end{overpic}
    \end{center}
    \caption{
    % A visualization of the spatial area selectivity concept.
    Combining both radial selectivity and DOA-based selectivity allows to localize the target.
    }
    \label{fig_rangeAzimuthSelectivity}
\end{figure}
\section{Fisher Information Matrix}
\label{sec_FIM}
\input{./Modules/sec_FIM_and_CRLB.tex}
\section{Temporal Stability}
\label{sec_stability}
As a preliminary, we start with simple temporal simulations of the feedback beamformer.
% In this section, we start by identifying the unknown parameters in the experimental scenario and formulate their related error terms.
% Using those error terms, some basic simulation scenarios are defined, raging from ideal to fully noised cases, which enable rigorous presentation of our findings.
% After the error terms have been defined, 
We first address the temporal stability of the system and discuss its correspondence with the system's response which was found in \eqref{eqn:GeneralFeedbackTransferFunction}.
\subsection{Gain mismatch}
The analogy of the FB to the temporal IIR architecture raises some fundamental issues that should be addressed.
In the following, supported by simulations, we answer those questions.
\par The first question, related to the FB's analogy to the temporal IIR architecture, is stability related.
To begin with, we should first ask what does ``stability`` means.
% Interestingly, the system's response of \eqref{eqn:GeneralFeedbackTransferFunction} is not frequency dependent, for it is exclusively governed by spatial parameters.
% Therefore, 
In the absence of any previous related work on spatial feedback, we find that the commonly used Bounded-Input-Bounded-Output (BIBO) stability is appropriate. 
% As such, assuming the \emph{Perfect alignment} scenario, we identify $r$ in \eqref{eq:SF_CB} to be the most influential for determining the system's stability for it controls the response's denominator - e.g. for $r<1$ the denominator will never be nullified.
To this end, the gain mismatch is
\begin{equation}
    r=g/\hat{g}.
\end{equation}
And plugging \eqref{eq:alpha_beta_opt} into \eqref{eqn:GeneralFeedbackTransferFunction}, gives rise to
\begin{equation}\label{eq:SF_CB_gainMismatch}
    %\resizebox{.894\linewidth}{!}
    {
        \begin{split}
            H_{\vecnot{\beta}_{\text{CB,opt}},\vecnot{\alpha}_{\text{CB,opt}}}
            = &\frac{
            g\vecnot{\beta}_{\text{CB,opt}}^{H}\vecnot{d}\exp{\rBrace{-j\phi}}
            }{
            1 - g\vecnot{\alpha}_{\text{CB,opt}}^{H}\vecnot{d}\exp{\rBrace{-j\phi}}
            }\\
            % = &\frac{    
            % \frac{r}{N}\vecnot{d}^{H}\vecnot{d}
            % }{
            % 1 - \frac{r}{N}\vecnot{d}^{H}\vecnot{d}
            % }\\
            = & \frac{r}{1-r} \triangleq H_{0}.
        \end{split}
    }
\end{equation}
As anticipated from \eqref{eq:SF_CB_gainMismatch} and confirmed with the simulations presented in Fig.\ref{fig_stability}, for $r\geq1$ the system is not stable, for the received amplitude may increase (in absolute value) to infinite values.
In Fig.~\ref{fig_stabilityVal}, we plot the final amplitude value after 100 iterations of the signal being retransmitted between the array and the target of interest and find that the final values substantially increase as $r\to{1}$.
This issue will have to be taken into consideration when one designs such an array for practical applications but currently this is outside the scope of this work.
\begin{figure}[t!]
    \begin{center}
        \begin{overpic}[width=0.55\linewidth, 
        %grid, 
        tics=10,trim=0 0 0 0]{./Media/fig_stabilization.png}
            \put (46, -5){\rotatebox{0}{$t/\tau_{pd}$}}
            \put (-6, 38){\rotatebox{90}{$\abs{z\rBrace{t}}$}}
            \put (46, 18.5){\rotatebox{0}{$r=0.6$}}
            \put (46, 35){\rotatebox{0}{$r=0.9$}}
            \put (46, 54){\rotatebox{0}{$r=1$}}
            \put (18, 60){\rotatebox{0}{$r=1.1$}}
            \put (0.75, 80){\rotatebox{0}{dB}}
        \end{overpic}
    \end{center}
    \caption{
    Simulating a 3 element ULA based FB, steered to a target which resides in the direction $\theta_{s} = \pi/2$ and plotting temporal response (dB) for multiple $r$ values.
    The horizontal axis ($t/\tau_{pd}$) is the time, normalized to the signal's round-trip duration to the target and back.
    }
    \label{fig_stability}
\end{figure}
As a final observation, we address the ``discrete`` behaviour of $z$, i.e. the value (most noticeable in the $r=1.1$ plot) seems to be piecewise constant.
This phenomenon corresponds with the propagation latency of the signal's round trip to the target and back.
In each iteration the mainlobe sharpens, as can be seen in Fig.~\ref{fig_beamThinning}.
The reader may notice in Fig.~\ref{fig_beamThinning} that the $t=1\cdot{}\tau_{pd}$ plot is actually the conventional beampattern, for the feedback signal has not yet re-impinged the array.
\begin{figure}[t!]
    \begin{center}
        \begin{overpic}[width=0.55\linewidth, 
        %grid, 
        tics=10,trim=0 0 0 0]{./Media/fig_beamThinning.png}
            \put (50, -5){\rotatebox{0}{$\theta_{d}/\pi$}}
            \put (-6, 38){\rotatebox{90}{$H_{r=0.9}$}}
            \put (100, 49){\rotatebox{0}{$t=1\cdot\tau_{pd}$}}
            \put (100, 39){\rotatebox{0}{$t=2\cdot\tau_{pd}$}}
            \put (100, 22.5){\rotatebox{0}{$t=5\cdot\tau_{pd}$}}
            \put (100, 12){\rotatebox{0}{$t=10\cdot\tau_{pd}$}}
            \put (53.5, 27.5){\rotatebox{-75}{$t=50\cdot\tau_{pd}$}}
            \put (4.5, 80){\rotatebox{0}{dB}}
        \end{overpic}
    \end{center}
    \caption{
    Simulation of the same setup as in Fig.\ref{fig_stability} with $r=0.9$.
    Several simulations are conducted, where each is simulated with different duration.
    It actually illustrates the forming of the final beampattern in time as our analysis assumes infinite duration.
    This convergence is analogous to the settling time of the IIR filters' temporal response, where at early stages, the response is not yet in its steady-state.  
    }
    \label{fig_beamThinning}
\end{figure}
\begin{figure}[t!]
    \begin{center}
        \begin{overpic}[width=0.55\linewidth, 
        %grid, 
        tics=10,trim=0 0 0 0]{./Media/fig_stabilizationVal.png}
            \put (0, 82){\rotatebox{0}{dB}}
            \put (-6, 30){\rotatebox{90}{$\lim_{t\to\infty}\abs{z\rBrace{t}}$}}
            \put (51, -5){\rotatebox{0}{$r$}}
        \end{overpic}
    \end{center}
    \caption{
    After simulating the same setup as in Fig.\ref{fig_stability}, we plot the final received amplitude in dB with different choices of $r$.
    }
    \label{fig_stabilityVal}
\end{figure}
\par The second question to be asked regards the settling time of the system, for in Fig.~\ref{fig_stability} we easily observe that the settling time of $r=0.6$ (i.e. $\sim{}10$ signal round trips) is substantially shorter than its matching value ($\sim{}50$) when $r=0.9$.
To better understand the phenomenon in Fig.~\ref{fig_stabilityDur}, we plot for each $r$ value, its corresponding settling time.
Although outside of this work's scope, this issue will obviously be of great importance when considering dynamic targets, as will be discussed in the ``future research`` part of the concluding Chapter.~\ref{chap:conclusion}.
\begin{figure}[t!]
    \begin{center}
        \begin{overpic}[width=0.55\linewidth, 
        %grid, 
        tics=10,trim=0 0 0 0]{./Media/fig_stabilizationDur.png}
            \put (-3, 35){\rotatebox{90}{$t/\tau_{\text{pd}}$}}
            \put (51, -5){\rotatebox{0}{$r$}}
        \end{overpic}
    \end{center}
    \caption{
    Simulation of same setup as in Fig.\ref{fig_stability}.
    For each $r$ (the horizontal axis), we plot the time (the vertical axis) where the received amplitude entered the $1\%$ sleeve around its final value.
    The plotted time is also in units of $\tau_{\text{pd}}$ normalized as in Fig.\ref{fig_stability}.
    }
    \label{fig_stabilityDur}
\end{figure}
% \par To conclude this section, we choose to add one more note, regarding the influence of the array's number of elements.
% Although in spatial processing literature, enlarging the number of elements in the array commonly increases the spatial performance, both examining \eqref{eq:SF_CB_gainMismatch} and simulating the experimental scenarios show that it (increase of elements) does not affect the temporal performance of the system - i.e. the settling time remains the same as $N$ increases.
\subsection{Phase mismatch}
\label{subsec_error_trms}
% Using the same weights $\vecnot{\beta}_{\text{CB,opt}},\vecnot{\alpha}_{\text{CB,opt}}$, 
We denote $\hat{\phi},\hat{\theta}$ to be the range and DOA related phase estimates respectively.
As we intend to compare the results to \cite{van2004optimum}, we consider a ULA, introducing its estimated steering vector
\begin{equation}\label{eq_est_d_ULA}
\hat{\vecnot{d}} = \vBrace{1, \exp{\rBrace{-\hat{\theta}}}, \hdots, \exp{\rBrace{-\rBrace{N-1}\hat{\theta}}}}^{T}.
\end{equation}
Observing that
\begin{equation*}
    \begin{split}
        \hat{\vecnot{d}}^{H}\vecnot{d} =&\ \vecnot{d}^{H}\hat{\vecnot{d}}\\
        =&\ \sum_{n=0}^{N-1}\exp{\rBrace{jn\Delta\theta}} \\
        =&\ \frac{\exp{\rBrace{jN\Delta\theta}}-1}{\exp{\rBrace{j\Delta\theta}}-1}\\
        =&\ \frac{\exp{\rBrace{jN\Delta\theta/2}}}{\exp{\rBrace{j\Delta\theta/2}}}\cdot
        \frac{\exp{\rBrace{jN\Delta\theta/2}}-\exp{\rBrace{-jN\Delta\theta/2}}}
        {\exp{\rBrace{j\Delta\theta/2}}-\exp{\rBrace{-j\Delta\theta/2}}}\\
        =&\ \exp{\rBrace{j\rBrace{N-1}\Delta\theta/2}}\cdot
        \frac{\sin{\rBrace{N\Delta\theta/2}}}
        {\sin{\rBrace{\Delta\theta/2}}}
    \end{split}
\end{equation*}
and using the definition of the normalized Dirichlet kernel (illustrated in Fig.~\ref{fig_DirichletKernelNorm})
\[
\D{x}{N}
% =\frac{1}{N}D_{\test{orig}}\rBrace{x,N}
=\frac{1}{N}\frac{\sin\rBrace{Nx}}{\sin\rBrace{x}},
\]
we find that for ULA, \eqref{eqn:GeneralFeedbackTransferFunction} may be expressed as
\begin{equation}\label{eq:alpha_beta_hat}
    \begin{split}
        H_{\vecnot{\beta}_{\text{CB}},\vecnot{\alpha}_{\text{CB}}}
        = &\frac{r\D{\dTheta/2}{N}\exp\cBrace{-j\vBrace{\dPhi+\rBrace{N-1}\dTheta/2}}}{1-r\D{\dTheta/2}{N}\exp\cBrace{-j\vBrace{\dPhi+\rBrace{N-1}\dTheta/2}}}.
    \end{split}
\end{equation}
% \begin{figure}[t!]
%     \begin{center}
%         \begin{overpic}[width=0.65\linewidth, 
%         %grid, 
%         tics=10,trim=0 0 0 0]{./Media/DirichletKernels.png}
%             \put (3, 26){\rotatebox{90}{$D_{\test{orig}}\rBrace{x,N}$}}
%             \put (91, 32){\rotatebox{0}{\scriptsize{$D_{\test{orig}}\rBrace{x,2}$}}}
%             \put (91, 53){\rotatebox{0}{\scriptsize{$D_{\test{orig}}\rBrace{x,5}$}}}
%             \put (91, 59){\rotatebox{0}{\scriptsize{$D_{\test{orig}}\rBrace{x,7}$}}}
%             \put (91, 8){\rotatebox{0}{\scriptsize{$D_{\test{orig}}\rBrace{x,10}$}}}
%             \put (50, 2.5){\rotatebox{0}{$x$}}
%             \put (12, 5){\rotatebox{0}{$-\pi$}}
%             \put (29, 5){\rotatebox{0}{$-\frac{\pi}{2}$}}
%             \put (70, 5){\rotatebox{0}{$\frac{\pi}{2}$}}
%             \put (89, 5){\rotatebox{0}{$\pi$}}
%         \end{overpic}
%     \end{center}
%     \caption{
%     Few illustrations of the Dirichlet kernel, for $N$ values of 2,5,7 and 10.
%     }
%     \label{fig_DirichletKernel}
% \end{figure}
\begin{figure}[t!]
    \begin{center}
        \begin{overpic}[width=0.65\linewidth, 
        %grid, 
        tics=10,trim=0 0 0 0]{./Media/DirichletKernelsNorm.png}
            \put (3, 32){\rotatebox{90}{$D\rBrace{x,N}$}}
            \put (86.5, 47){\rotatebox{80}{\tiny{$D\rBrace{x,7}$}}}
            \put (81.5, 47){\rotatebox{80}{\tiny{$D\rBrace{x,5}$}}}
            \put (84.5, 30){\rotatebox{-86}{\tiny{$D\rBrace{x,10}$}}}
            \put (75.5, 30){\rotatebox{-70}{\tiny{$D\rBrace{x,2}$}}}
            \put (50, 2.5){\rotatebox{0}{$x$}}
            \put (12, 5){\rotatebox{0}{$-\pi$}}
            \put (29, 5){\rotatebox{0}{$-\frac{\pi}{2}$}}
            \put (70, 5){\rotatebox{0}{$\frac{\pi}{2}$}}
            \put (89, 5){\rotatebox{0}{$\pi$}}
        \end{overpic}
    \end{center}
    \caption{
    Few illustrations of the normalized Dirichlet kernel, for $N$ values of 2,5,7 and 10.
    }
    \label{fig_DirichletKernelNorm}
\end{figure}
In the following, four fundamental scenarios are considered:
\begin{itemize}
    \item{\makebox[.35\linewidth]{Perfect alignment \hfill} $\rBrace{\dTheta=0\ , \dPhi=0},$}
    \item{\makebox[.35\linewidth]{Steering error \hfill} $\rBrace{\abs{\dTheta}>0\ , \dPhi=0},$}
    \item{\makebox[.35\linewidth]{Range error \hfill} $\rBrace{\dTheta=0\ , \abs{\dPhi}>0},$}
    \item{\makebox[.35\linewidth]{General \hfill} $\rBrace{\abs{\dTheta}>0\ , \abs{\dPhi}>0}.$}
\end{itemize}
\section{Performance Analysis}
\label{sec_Performance}
In this section we analyze the suggested FB (see Fig.~\ref{fig:Proposed_spatialIIR_ARCH}), considering some fundamental properties which are commonly used to asses array performance and discussed in Sec.~\ref{sec:prlm_array_perf}: Beamwidth, peak-to-sidelobe level, and directivity. Each property is then compared to traditional passive ULAs, showing that significantly improved performances are obtainable with spatial feedback integration.
\subsection{The normalized beampattern}
\label{subsection_spatialIIR_normBP}
Applying the \emph{normalized beampattern} concept of Sec.~\ref{sec:prlm_array_perf} to this work, we set $\vecnot{\beta}_{\coefSetName,\text{opt}}=\vecnot{\alpha}_{\coefSetName,\text{opt}},$ giving rise to
\begin{equation}
    \label{eq_narmalized_pattern}
    %\resizebox{.89\linewidth}{!}{
    \begin{split}
        \HrTPr&\triangleq
        \frac{
        \HbaFB
        }{
        H_{\vecnot{\beta}_{\coefSetName,\text{opt}},\vecnot{\alpha}_{\coefSetName,\text{opt}}}
        }
         =
        \frac{
        \HbaFB
        }{
        r/\rBrace{1-r}
        },
        % \\
        % &=\frac{\rBrace{1-r}^{2}\Dp{\dTheta/2,N}{2}}{1+r^{2}\Dp{\dTheta/2,N}{2}-2r\D{\dTheta/2}{N}\cos{\rBrace{\dPhi+\frac{N-1}{2}\dTheta}}},
    \end{split}
    %}
\end{equation}
where the $\Delta\theta,\Delta\phi,r$ subscripts express the DOA, range and gain errors dependency, respectively.
%Also, $\abs{\HrTPr}$ is defined to be the beampattern.
Plugging \eqref{eq:SF_CB} into \eqref{eq_narmalized_pattern} yields 
\begin{equation}\label{eq_generalH}
    %\resizebox{0.89\linewidth}{!}
    {
        \begin{split}
             \HrTPr=
             \frac{\rBrace{1-r}\D{\dTheta/2}{N}}{\exp\rBrace{j\rBrace{\dPhi+(N-1)\dTheta/2}}-r\D{\dTheta/2}{N}}.
        \end{split}
        }
\end{equation}
Note that the known \cite{van2004optimum} normalized response of standard ULA is obtained by setting $r=0$ and $\Delta\phi=0$
$$
\Hr_{\Delta\theta,\Delta\phi=0,r=0}=
             \D{\dTheta/2}{N}\exp\rBrace{-j\rBrace{(N-1)\dTheta/2}}.
$$
Considering the steering error scenario (i.e., $\dTheta\neq0,\ \dPhi=0$) first, where 
\begin{equation}\label{eq_Hdphi0}
\Hr_{\Delta\theta,\Delta\phi=0,r}=
             \frac{\rBrace{1-r}\D{\dTheta/2}{N}}{\exp\rBrace{j\rBrace{(N-1)\dTheta/2}}-r\D{\dTheta/2}{N}},
\end{equation}
we evaluate the FB's beamwidth, sidelobe level and directivity and compare them to those of the standard ULA.
\subsection{Half power beamwidth}
\input{./Modules/arrayPerformance_beamwidth_LOCAL.tex}
\subsection{Sidelobes attenuation}
\ifdefined\showDev
    \fbox{
    \begin{minipage}{0.9\linewidth}
    \textbf{development specifics}\\
    Let $f\rBrace{\dTheta} \triangleq \D{\dTheta/2}{N}\exp\rBrace{-j\rBrace{(N-1)\dTheta/2}},$ such that $\Hr_{\Delta\theta,\Delta\phi=0,r=0} = \frac{f\rBrace{\dTheta}}{1-rf\rBrace{\dTheta}}.$ Then, we compute the beampattern's derivative and state that
    \begin{equation*}
    % \resizebox{0.9\linewidth}{!}{
        \begin{split}
            &\frac{\partial}{\partial\dTheta}\Hr_{\Delta\theta,\Delta\phi=0,r} &\\
            &=\frac{\Dp{\dTheta/2,N}{'}\rBrace{1-r\D{\dTheta/2}{N}}-\D{\dTheta/2}{N}\rBrace{-r\Dp{\dTheta/2,N}{'}}}{\rBrace{1-r\D{\dTheta/2}{N}}^{2}}
            \\
            &=\frac{\Dp{\dTheta/2,N}{'}}{\rBrace{1-r\D{\dTheta/2}{N}}^{2}}.
        \end{split}
        % }
    \end{equation*}
    It follows that the sidelobes of the feedback-based beampattern are located at the same angles as the in the ULA case.
    \end{minipage}
    }
\else
\fi
By taking a derivative of $\Hr_{\Delta\theta,\Delta\phi=0,r}$ with respect to $\dTheta$ it can be easily verified that the beampattern's extrema points are located exactly as in the standard ULA beampattern. Specifically, the sidelobes locations are
\begin{equation}
    \label{eqn_CB_sidelobesLocations}
    \Delta\theta_{\text{sidelobe}} = \frac{\rBrace{2m+1}\pi}{N}\ \forall m\in\cBrace{\pm 1,\pm 2,\hdots}.
\end{equation}
Our main interest is with the first sidelobe (i.e. $m=1$), therefore we evaluate \eqref{eq_Hdphi0} at $\Delta\theta = 3\pi/N$, which results in
\begin{equation}
    \abs{\Hr_{{3\pi}/{N},0,r}}^2
    =
    \frac{
    2\rBrace{1-r}^{2}
    }{
    \rBrace{N^{2}-2Nr}\rBrace{1-\cos{\rBrace{\frac{3\pi}{N}}}}+2r^{2}
    }
    \label{eq_HSidelobes}
\end{equation}
and for large $N$ values 
\begin{equation*}
    \lim_{N\rightarrow\infty}\abs{\Hr_{3\pi/N,0,r}}=\frac{
    2\rBrace{1-r}
    }{
    3\pi
    }.
\end{equation*}
\ifdefined\showDev
    \fbox{
    \begin{minipage}{0.9\linewidth}
    \textbf{development specifics}\\
    We know that \eqref{eq_Hdphi0} can be rewritten as
    \begin{equation*}
        \begin{split}
            \Hr_{\Delta\theta,\Delta\phi=0,r=0} =
            \frac{
            \rBrace{1-\cos{\rBrace{N\dTheta}}}\rBrace{1-r}^{2}
            }{
            \begin{split}
                N^{2}&\rBrace{1-\cos{\dTheta}}+r^{2}\rBrace{1-\cos{N\dTheta}}
                \\
                &+Nr\Bigg(1+\cos{\rBrace{\rBrace{N-1}\dTheta}}
                \\
                &-\cos{\rBrace{N\dTheta}-\cos{\rBrace{\dTheta}}}\Bigg)
            \end{split}
            }
        \end{split}
    \end{equation*}
    Using the symbolic toolbox in MATLAB, setting $\dTheta=3\pi/N$, results in \eqref{eq_HSidelobes}.
    \end{minipage}
    }
\else
\fi
\par For standard ULA, the gain of the first sidelobe is known to be $2/3\pi$ \cite{van2004optimum}, which implies that the first sidelobe is smaller by a factor of $1-r$ compared to standard ULA.
Specifically, in perfect gain match scenario (i.e., $r\to{}1$) , the sidelobes vanish. 
\subsection{Array directivity}
In the context of this work, the directivity is expressed as
\begin{equation}\label{eq_D}
    \mathcal{D}\rBrace{N,r} = \frac{\Hr_{\Delta\theta=0,\Delta\phi=0,r}}{\frac{1}{2\pi}\int_{0}^{2\pi}\Hr_{\Delta\theta,\Delta\phi=0,r}\ d\Delta\theta} = \frac{2\pi}{\int_{0}^{2\pi}\Hr_{\Delta\theta,\Delta\phi=0,r}\ d\Delta\theta},
\end{equation}
Plugging \eqref{eq_Hdphi0} within \eqref{eq_D} and by numerical evaluation (see App.~\ref{apndx_directivityFit} and the complementary Fig.~\ref{fig_directivity}) we suggest to approximate the directivity with 
\begin{equation}\label{eq_D_result}
    \mathcal{D}\rBrace{N,r} \approx \frac{N-r}{1-r},
\end{equation}
where the standard ULA's known result is obtained for $r=0$.
Also, for $N\geq2$, $\lim_{r\rightarrow 1}\mathcal{D}\rBrace{N,r}=\infty$, implying infinite directivity for the perfectly gain-matched FB. 
\par Finally, expressing the improvement in directivity compared to the standard ULA, assuming large $N$ values, gives rise to
\begin{equation}\label{eq_Dimprovement}
\lim_{N\to\infty}\frac{\mathcal{D}\rBrace{N,r}}{\mathcal{D}\rBrace{N,0}}
=\frac{N/\rBrace{1-r}}{N}=\frac{1}{1-r}.
\end{equation}
\begin{figure}[t]
    \begin{center}
        \begin{overpic}[width=0.75\linewidth, 
        %grid, 
        tics=10,trim=0 0 0 0]{./Media/directivity_expr.eps}
            \put (20, 10){\footnotesize{$N$}}
            \put (72, 6){\footnotesize{$r$}}
            \put (16.5, 70){\footnotesize{Numerical integration}}
            \put (16.5, 65){\footnotesize{$\rBrace{N-r}/\rBrace{1-r}$}}
            \put (-9, 45){\footnotesize{$\mathcal{D}\rBrace{N,r}$}}
        \end{overpic}
    \end{center}
     \caption{Plot of $\mathcal{D}\rBrace{N,r}$, computed using numerical integration (surface), shown to perfectly match the analytic expression (black diamonds) presented in \eqref{eq_D_result}.}
    \label{fig_directivity}
\end{figure}
\subsection{Summary}
To conclude this section, we summarize the feedback integration related performance improvements in Table.~\ref{table_arrayPerformance}.
\begin{table}[h!]
    \caption{Performances of Classical ULA and the Proposed Feedback-Beamforming Architecture, with a Gain Mismatch $r$.}
    \centering
    %\resizebox{1\linewidth}{!}
    {
        \begin{tabular}{||c c c c||}
            \hline
            & ULA & \thead{FEEDBACK\\BEAMFORMING} & IMPROVEMENT \\ [0.5ex] 
            \hline\hline
            HPBW & $ 1.4/N$ & $1.4/\rBrace{f(r)N}$ & Narrower by a factor of $f\rBrace{r}$\\ 
            \thead{FIRST\\SIDELOBE\\GAIN} & $2/3\pi$ & $2\rBrace{1-r}/3\pi$ & \thead{smaller by a factor of $1-r$\\for $N\gg{}1$} \\
            DIRECTIVITY & $N$ & $\rBrace{N-r}/\rBrace{1-r}$ & \thead{$1/\rBrace{1-r}$ times higher \\ for $N\gg{}1$}\\
            [1ex] 
            \hline
         \end{tabular}
     }
    \label{table_arrayPerformance}
\end{table}
Also, in Fig.~\ref{fig_perf} we demonstrate the HPBW and sidelobe attenuation for 3 elements ULA.
As the expressions for HPBW and sidelobe-attenuation are relevant for $\sim{}N>20$, we also simulate 20 elements ULA for in Fig.~\ref{fig_perf20}. 
In Table.~\ref{table_arrayPerfEmp} we show the coherency between the results and the performance related expressions of Table.~\ref{table_arrayPerformance}.
As predicted, the results of Fig.~\ref{fig_perf20} are consistent with the theoretical expressions.
\begin{figure}[t]
    \begin{center}
        \begin{overpic}[width=0.9\linewidth, 
        %grid, 
        tics=10,trim=0 0 0 0]{./Media/fig_performance.png}
            \put (9, 36.5){$dB$}
            \put (2, 20){$\abs{H}^{2}$}
            \put (27, 0){$\theta_{d}/\pi$}
            \put (34.5, 67){$r=0$}
            \put (29, 63){\rotatebox{-90}{\tiny{HPBW = $0.1\pi$}}}
            \put (40, 61){\tiny{$-19.08_{dB}$}}
            \put (77.5, 67){$r=0.3$}
            \put (73.14, 63){\rotatebox{-90}{\tiny{HPBW = $0.073\pi$}}}
            \put (83.5, 58.5){\tiny{$-23.43_{dB}$}}
            \put (34.5, 31.5){$r=0.6$}
            \put (29, 27){\rotatebox{-90}{\tiny{HPBW = $0.043\pi$}}}
            \put (40, 18.75){\tiny{$-31.09_{dB}$}}
            \put (77.5, 31.5){$r=0.8$}
            \put (73.14, 27){\rotatebox{-90}{\tiny{HPBW = $0.02\pi$}}}
            \put (84.25, 13.5){\tiny{$-41.6_{dB}$}}
        \end{overpic}
    \end{center}
     \caption{Simulating 3 elements ULA based feedback beamformer for $r$ values of 0, 0.3, 0.6 and 0.8.
     The HPBW is marked with vertical red dashed line, where an auxiliary horizontal line of $\abs{H}^{2} = 1/2$ is also provided.
     The sidelobe attenuation is cited in each plot in a textual fashion.}
    \label{fig_perf}
\end{figure}
\begin{figure}[t]
    \begin{center}
        \begin{overpic}[width=0.9\linewidth, 
        %grid, 
        tics=10,trim=0 0 0 0]{./Media/fig_performance20.png}
            \put (9, 36.5){$dB$}
            \put (-3, 20){$\abs{H}^{2}$}
            \put (27, 0){$\theta_{d}/\pi$}
            \put (34.5, 67){$r=0$}
            \put (29, 63){\rotatebox{-90}{\tiny{HPBW = $0.0605\pi$}}}
            \put (32.25, 56.5){\tiny{$-26.38_{dB}$}}
            \put (77.5, 67){$r=0.3$}
            \put (73, 63){\rotatebox{-90}{\tiny{HPBW = $0.0454\pi$}}}
            \put (79, 53.75){\tiny{$-32.06_{dB}$}}
            \put (34.5, 31.25){$r=0.6$}
            \put (29, 27){\rotatebox{-90}{\tiny{HPBW = $0.0259\pi$}}}
            \put (40, 14){\tiny{$-41.21_{dB}$}}
            \put (77.5, 31.25){$r=0.8$}
            \put (73, 27){\rotatebox{-90}{\tiny{HPBW = $0.0127\pi$}}}
            \put (84.25, 12.5){\tiny{$-52.55_{dB}$}}
        \end{overpic}
    \end{center}
     \caption{Simulating 20 elements ULA for the sake of HPBW analysis.
     As in Fig.~\ref{fig_perf}, $r$ values of 0, 0.3, 0.6 and 0.8 are simulated.
     The HPBW is marked in the same manner also.}
    \label{fig_perf20}
\end{figure}
\begin{table}[h!]
    \caption{Performances of conventional beamformer $r=0$ and the proposed Feedback-Beamforming Architecture, for several $r$ values.
    Also comparison to expected theoretical expressions is provided.}
    \centering
    %\resizebox{1\linewidth}{!}
    {
        \begin{tabular}{||c | c c c | c c c||}
            \hline
            & \multicolumn{3}{c|}{HPBW [RAD]} & \multicolumn{3}{c||}{SIDELOBE GAIN [dB]} \\ [0.5ex]
            \hline
            Expected & \multicolumn{3}{c|}{$\frac{\rBrace{1-r}\rBrace{-0.4r+1.4}}{1.4}$} & \multicolumn{3}{c||}{$20\log{}\rBrace{\abs{\frac{2\rBrace{1-r}}{3\pi}}^{2}}$} \\ [0.5ex]
            \hline
            & Result & Expected & Error & Result & Expected & Error \\ [0.5ex] 
            \hline\hline
            $r=0$ & 0.0605 & 0.07 & 0.0095 & -26.38 & -26.93 & 0.55 \\ [0.5ex]
            $r=0.3$ & 0.0454 & 0.0448 & 0.006 & -32.06 & -33.12 & 1.06 \\ [0.5ex]
            $r=0.6$ & 0.0259 & 0.0232 & 0.0027 & -41.21 & -42.84 & 1.63 \\ [0.5ex]
            $r=0.8$ & 0.0127 & 0.0108 & 0.0019 & -52.55 & -54.88 & 2.33 \\ [0.5ex]
            \hline
         \end{tabular}
     }
    \label{table_arrayPerfEmp}
\end{table}