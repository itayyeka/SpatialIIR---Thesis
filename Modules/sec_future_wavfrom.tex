In modern RADAR related systems, the waveform of choice is pulse based and modulated by CW, or linear/nonlinear frequency scans.
It holds some obvious advantages over the CW transmission such as
\begin{itemize}
    \item Lower power consumption due to lower transmissions duty cycle. 
    \item The use of wide band signals increases spatial resolution in the radial axis.
\end{itemize}
Another interesting option is using several (possibly many) independent harmonics as super positioned multiple dual frequency beamformers, which simultaneously scan a wide area with multiple resolutions.
By merely selecting harmonic couples and filter coefficients, each harmonics pair is pointed to a specific area.
It follows that with the expense of computational effort (without adding elements to the array), it seems possible that one may have simultaneous scan of the entire arena in any wanted direction/range/resolution.
\par 
As a simple example, consider that the arena consists of $M$ zones of interest, where each zone may be described as a angular bounded radial slice (similar to the visualization of Fig.~\ref{fig_rangeAzimuthSelectivity}) which may be described as $$a_{m}=\cBrace{\vecnot{p}\ \middle\vert\  \theta_{d,m,\text{min}}<\text{DOA}\rBrace{\vecnot{p}}<\theta_{d,m,\text{max}}\ ,\ R_{m,\text{min}}<\norm{\vecnot{p}}<R_{m,\text{max}}}\ \forall\ m=0\dots{}M-1.$$
\par
Each zone will be assigned with matching couple of carrier frequencies ($\omega_{1,2}$) to cover its radial width (where the radial width is equivalent to $R_{\text{rt}}$ of \eqref{eq_DF_phErr}) and coefficient sets ($\vecnot{\alpha}_{m,i},\vecnot{\beta}_{m,i} \ \forall\ m=0\dots{}M-1, i=1,2$), which cover the zone's angular diameter which is equivalent to determining the desired beamwidth via setting the proper $\kappa$ of \eqref{eqn_H_DF_CB}.
\par
Leaving aside the needed computational effort and physical limitations of transmitting multiple narrowband signals, assuming large $M$ values, we have a super-positioned FB which simultaneously scans the entire arena.
