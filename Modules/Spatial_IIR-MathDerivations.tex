As in Sec.~\ref{sec:prlm_FIR_IIR}, an $N$-element ULA with inter-element spacing $d$ is considered.
Its $n$'th sensor is positioned at $\vpi{n},\;\text{for}\; n=0,\ldots,N-1$ and we set $\vpi{0}$ as the axis reference point while the target of interest is positioned at $\vpt$.
We also place a transmitter at $\vpi{0}$, where transmitted signal $s$, is reflected back from the target and re-impinges the array, with a total time delay of $\tau_{\text{pd}}=2R/c$ seconds and $R = \norm{\vpt-\vpi{0}}, c$ are the target's range propagation velocity of the signal in the medium respectively.
Time domain analysis of the proposed feedback based architecture, considering both propagation delay and attenuation, gives rise to
\begin{equation}
    \label{eqn:SingleSensorTemporalEquality}
    % \resizebox{.91\linewidth}{!}{
        \begin{split}
            x_{n}(t) = g\vBrace{s\rBrace{t-\tau_{pd}-\tau_{n}}
            +\sum_{m=0}^{N-1}{\alpha^{*}_{m}x_{m}\rBrace{t-\tau_{pd}-\tau_{n}}}},
        \end{split}
    % }
\end{equation}
where the first term on the right-hand-side (RHS) represents the contribution of the transmitted waveform $s(t)$ to the $n$'th array element and the second term represents the feedback contribution of the re-transmitted array signal to this same element.
Expressing the Fourier transform of \eqref{eqn:SingleSensorTemporalEquality},
\begin{equation}
    \label{eqn_singleSensorFourier}
    % \resizebox{.91\linewidth}{!}{
            X_{n} =
            g\cBrace{\F{s}
            \exp\vBrace{-j\omega\rBrace{\tau_{pd}+\tau_{n}}}
            +\sum_{m=0}^{N-1}
            {
            \alpha^{*}_{m}\F{x}_{m}
            \exp\vBrace{-j\omega\rBrace{\tau_{pd}+\tau_{n}}}
            }},
    % }
\end{equation}
and its vector from,
$$
\F{\vx} = g\rBrace{\F{s}+\vAlphaH \F{\vx}}\vd\exp{\rBrace{-j\omega\tau_{pd}}},
$$
we find that it can be simplified to
$$
\F{\vx} =g\rBrace{I-g\vd\vAlphaH{}e^{-j\omega\tau_{pd}}}^{-1}\vd\F{s}\exp{\rBrace{-j\omega\tau_{pd}}}.
$$
Then, denoting
\[
\phi\triangleq\omega\tau_{pd}
\]
as the round-trip signal propagation related electrical phase.
We use the Sherman-Morrison formula \cite{sherman1950adjustment}, considered to be a result of the Woodbury matrix identity \cite{woodbury1950inverting}, stating that
\begin{equation*}
    \rBrace{I+\vecnot{u}\vecnot{v}^{T}}^{-1}=I-\frac{\vecnot{u}\vecnot{v}^{T}}{1+\vecnot{v}^{T}\vecnot{u}},
\end{equation*}
where $\vecnot{u},\vecnot{v}$ are two $N\times1$ vectors and $I$ is the identity matrix.   
Setting $\vecnot{u}=-g\vecnot{d}\exp{\rBrace{-j\omega\tau}}$ and $\vecnot{v}=\vecnot{\alpha}^{*}$ gives rise to
\begin{equation*}
    \begin{split}
        \rBrace{I-g\vecnot{d}\vecnot{\alpha}^{H}\exp{\rBrace{-j\omega\tau}}}^{-1} 
        &= I+\frac{g\vecnot{d}\vecnot{\alpha}^{H}\exp{\rBrace{-j\omega\tau}}}{1-g\vecnot{\alpha}^{H}\vecnot{d}\exp{\rBrace{-j\omega\tau}}}
        \\
        &= \frac{\rBrace{I-g\vecnot{\alpha}^{H}\vecnot{d}\exp{\vBrace{-j\omega\tau}}+g\vecnot{d}\vecnot{\alpha}^{H}\exp{\vBrace{-j\omega\tau}}}\vecnot{d}}{1-g\vecnot{\alpha}^{H}\vecnot{d}\exp{\rBrace{-j\omega\tau}}}
        \\
        &= \frac{
        \vBrace{I+\rBrace{\vecnot{d}\vecnot{\alpha}^{H}-\vecnot{\alpha}^{H}\vecnot{d}}g\exp{\rBrace{-j\omega\tau}}}\vecnot{d}
        }{
        1-g\vecnot{\alpha}^{H}\vecnot{d}\exp{\rBrace{-j\omega\tau}}
        }
        \\
        &= \frac{
        \vecnot{d}+\rBrace{\vecnot{d}\vecnot{\alpha}^{H}\vecnot{d}-\vecnot{\alpha}^{H}\vecnot{d}\vecnot{d}}g\exp{\rBrace{-j\omega\tau}}
        }{
        1-g\vecnot{\alpha}^{H}\vecnot{d}\exp{\rBrace{-j\omega\tau}}
        }
        \\
        &= \frac{
        \vecnot{d}+\vecnot{\alpha}^{H}\vecnot{d}\rBrace{\vecnot{d}-\vecnot{d}}g\exp{\rBrace{-j\omega\tau}}
        }{
        1-g\vecnot{\alpha}^{H}\vecnot{d}\exp{\rBrace{-j\omega\tau}}
        }
        \\
        &= \frac{
        \vecnot{d}
        }{
        1-g\vecnot{\alpha}^{H}\vecnot{d}\exp{\rBrace{-j\omega\tau}}
        }
    \end{split}
\end{equation*}
leading to
$$
\F{\vx}
=
\frac{    
g\vd\exp{\rBrace{-j\phi}}
}{
1 - g\aHd{}\exp{\rBrace{-j\phi}}
}\F{s}.
$$
Let $z=\vBetaH{}\vecnot{x}+\text{n}$ be the beamformer's output (see Fig.~\ref{fig:Proposed_spatialIIR_ARCH}), with Fourier transform $Z$. Considering the noiseless case $\rBrace{\text{i.e., n}=0}$, the frequency response of the FB is 
\begin{equation}
\label{eqn:GeneralFeedbackTransferFunction}
\Hba
\triangleq
\frac{\F{z}}{\F{s}} 
=
\frac{    
g\bHd{}\exp\rBrace{-j\phi}
}{
1 - g\aHd{}\exp\rBrace{-j\phi}
}.
\end{equation}
\par Note that this architecture achieves a controllable (via setting of $\vBeta$ and $\vAlpha$) and recursive (non-trivial denominator) spatial response.
As will be shown, high directivity and narrow beamwidth are obtainable by a proper selection of the weights. Compared to traditional beamformers (i.e., without feedback), the performance improvement will be expressed in terms of increased aperture, narrower beamwidth and improved sidelobe attenuation.
One may observe that opposed to traditional beamformers, the array response, $\Hba,$ is not only influenced by the impinging signal DOA, since it is also range selective due to its $\phi$ dependency.
As demonstrated in Fig.~\ref{fig_rangeAzimuthSelectivity}, the combination of both angular and range selectivity enables the designer to enhance signals arriving from specific locations (grey area) rather than only specific directions.
\begin{figure}[t!]
    \begin{center}
        \begin{overpic}[width=0.55\linewidth, 
        % grid, 
        tics=10,trim=0 0 0 0]{./Media/azimuthRangSelectivity.png}
            \put (20, 23){\rotatebox{0}{\footnotesize{Angular response}}}
            \put (32, 47){\rotatebox{0}{\footnotesize{Enhanced radial slice}}}
        \end{overpic}
    \end{center}
    \caption{
    % A visualization of the spatial area selectivity concept.
    Combining both radial selectivity and DOA-based selectivity allows to localize the target.
    }
    \label{fig_rangeAzimuthSelectivity}
\end{figure}