% In array signal processing, propagating waves carry signals from the source to the array.
% Therefore, these signals are represented in the time-space domain. The space domain is represented by either the three dimensional Cartesian coordinates $\rBrace{x,y,z}$ or the three dimensional spherical coordinates $\rBrace{r,\phi,\theta}$, where $0 \leq \phi \leq 2\pi$, $0 \leq \theta \leq \pi$ are the azimuth and elevation angles, respectively. The time domain is represented by $t$.
An elementary physical phenomenon is the spatial dynamics of waves, referred to as wave-propagation.
The spatial location is noted by the Cartesian coordinates $\rBrace{x,y,z}$ or the three dimensional spherical coordinates $\rBrace{r,\phi,\theta}$, where $0 \leq \phi \leq 2\pi$, $0 \leq \theta \leq \pi$ are the azimuth and elevation angles, respectively. 
% Finally, let $f\rBrace{t,\vecnot{r}}$ denote the time-space representation of the input signal, where $\vecnot{r}$ is the radius-vector in the three-dimensional system. 
% The relations between the Cartesian and spherical coordinates are given in Fig.~\ref{fig_coordinates}.
Denoting $t$ as the time, the time-space representation of the a signal is $f\rBrace{t,\vecnot{r}}$ or , where the relations between the Cartesian and spherical coordinates are given in Fig.~\ref{fig_coordinates}.
% $f\rBrace{t,x,y,z}$ describes the signal impinging the microphone array. 
In a homogeneous, dispersion free and lossless medium the wave equation is:
\begin{equation}
\label{eq_prlm_waveEq}
\nabla^{2}f\rBrace{t,x,y,z}=\frac{1}{c^2}\frac{\partial^{2}f\rBrace{t,x,y,z}}{\partial{t^{2}}}
\end{equation}
where $\nabla^{2}$ is the Laplacian operator and $c$ represents the wave's velocity in the medium.
\begin{figure}[h!]
    \begin{center}
        \begin{overpic}[width=0.5\linewidth, 
        %grid, 
        tics=10,trim=0 0 0 0]{./Media/fig_coordinates.png}
            % \put (50, 62.5) {\footnotesize{$r=0$}}
        \end{overpic}
    \end{center}
     \caption{A three dimensional coordinate system with Cartesian and spherical coordinates.}
    \label{fig_coordinates}
\end{figure}
A possible solution to \eqref{eq_prlm_waveEq}, $f_p\rBrace{t,x,y,z}$, may be of a complex exponential form:
\begin{equation}
\label{eq_prlm_waveEq_pSol}
f_p\rBrace{t,x,y,z} = A\exp{j\rBrace{\omega{t}-k_{x}x-k_{y}y-k_{z}z}}
\end{equation}
Where $A$ is a complex constant, $\omega$ denotes a constant temporal frequency and $k_{x}x,k_{y},k_{z}$ are real constants. 
Plugging \eqref{eq_prlm_waveEq_pSol} into \eqref{eq_prlm_waveEq} results in the \emph{monochromatic plane wave}:
\begin{equation}
\label{eq_prlm_waveEq_subs}
k_{x}^{2}-k_{y}^{2}-k_{z}^{2} = \frac{\omega^{2}}{c^{2}}.
\end{equation}
Using the plane wave notation emphasizes the fact that for a given time, $t_{0}$, all points on a plane given by $k_{x}x+k_{y}y+k_{z}z = constant$ are with the same wave value and the vector notation of \eqref{eq_prlm_waveEq}'s solution is
\begin{equation}
\label{eq_prlm_waveEq_finSol}
f\rBrace{t,\vecnot{x}}=A\exp{j\rBrace{\omega{t}-\vecnot{k}\vecnot{x}}}.
\end{equation}
where $\vecnot{k}\vecnot{x}=constant$ are planes of constant wave value.
The wave propagation can be described as the traveling of the planes, stating that small steps of both space $\delta{\vecnot{x}}$ and time $\delta{t}$ result in the same wave value i.e. $f\rBrace{t+\delta{t},\vecnot{x}+\delta{\vecnot{x}}} = f\rBrace{t,\vecnot{x}}$, which yields
\begin{equation}
\omega\delta{t}-\vecnot{k}\delta\vecnot{x}=0.
\end{equation}
Assuming $\delta\vecnot{x}$ and $\vecnot{k}$ have the same direction, $\vecnot{k}\delta\vecnot{x} = \abs{\vecnot{k}}\abs{\delta\vecnot{x}}$ and $\frac{\delta{\vecnot{x}}}{\delta{t}} = \frac{\omega}{\abs{\vecnot{k}}}$, where $\frac{\delta{\vecnot{x}}}{\delta{t}}$ can designate the propagation speed of the plane wave. 
Since $\vecnot{k}$ and $\omega$ are related by $\abs{\vecnot{k}}^{2}=\frac{\omega^{2}}{c^{2}}$, we have
\begin{equation}
\label{eq_prlm_waveEq_stepsEq}
\frac{\delta{\vecnot{x}}}{\delta{t}}=c,
\end{equation}
where $c>0$ is the wave velocity in the medium.
The \emph{wavelength} ($\lambda$) denote the distance the plane wave propagated during a single temporal period of $T=\frac{2\pi}{\omega}$.
Let $\vecnot{k}$ denote the \emph{wavelength vector}. 
Its magnitude $\abs{\vecnot{k}}$ expresses the number of cycles in radians per meter of length that the plane wave has exhibited in the propagation direction.
Using \eqref{eq_prlm_waveEq_stepsEq} with $\delta{t} = \frac{2\pi}{\omega}$, we obtain:
\begin{equation}
\lambda=\delta{\vecnot{x}}=\frac{2\pi}{\abs{\vecnot{k}}}.
\end{equation}
Therefore, the wavenumber vector can be considered to represent spatial frequency, similarly to the manner $\omega$ represents temporal frequency.