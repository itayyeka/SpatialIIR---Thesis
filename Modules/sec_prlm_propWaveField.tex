% In array signal processing, propagating waves carry signals from the source to the array.
% Therefore, these signals are represented in the time-space domain. The space domain is represented by either the three dimensional Cartesian coordinates $\rBrace{x,y,z}$ or the three dimensional spherical coordinates $\rBrace{r,\phi,\theta}$, where $0 \leq \phi \leq 2\pi$, $0 \leq \theta \leq \pi$ are the azimuth and elevation angles, respectively. The time domain is represented by $t$.
An elementary physical phenomenon is the spatial and temporal dynamics of waves, referred to as wave-propagation.
The spatial location is noted by the Cartesian coordinates $\rBrace{x,y,z}$ or the three dimensional spherical coordinates $\rBrace{r,\phi,\theta}$, where $0 \leq \phi \leq 2\pi$, $0 \leq \theta \leq \pi$ are the azimuth and elevation angles, respectively.
The relations between the coordinates are given in Fig.~\ref{fig_coordinates}.
% Finally, let $f\rBrace{t,\vecnot{r}}$ denote the time-space representation of the input signal, where $\vecnot{r}$ is the radius-vector in the three-dimensional system. 
% The relations between the Cartesian and spherical coordinates are given in Fig.~\ref{fig_coordinates}.
Denoting $t$ as the time, the time-space representation of the a signal is $f\rBrace{t,\vecnot{r}}$.
% $f\rBrace{t,x,y,z}$ describes the signal impinging the microphone array. 
In a homogeneous, dispersion free and lossless medium the wave equation is:
\begin{equation}
\label{eq_prlm_waveEq}
\nabla^{2}f\rBrace{t,x,y,z}=\frac{1}{c^2}\frac{\partial^{2}f\rBrace{t,x,y,z}}{\partial{t^{2}}}
\end{equation}
where $\nabla^{2}$ is the Laplacian operator and $c$ represents the wave's velocity in the medium.
\begin{figure}[h!]
    \begin{center}
        \begin{overpic}[width=0.5\linewidth, 
        %grid, 
        tics=10,trim=0 0 0 0]{./Media/fig_coordinates.png}
            % \put (50, 62.5) {\footnotesize{$r=0$}}
        \end{overpic}
    \end{center}
     \caption{A three dimensional coordinate system with Cartesian and spherical coordinates.}
    \label{fig_coordinates}
\end{figure}
A possible solution to \eqref{eq_prlm_waveEq}, $f_p\rBrace{t,x,y,z}$, is a complex exponential of the form:
\begin{equation}
\label{eq_prlm_waveEq_pSol}
f_p\rBrace{t,x,y,z} = A\exp\vBrace{j\rBrace{\omega{t}-k_{x}x-k_{y}y-k_{z}z}}
\end{equation}
Where $A$ is a complex constant, $\omega$ denotes temporal radial frequency, $\vecnot{k} = \vBrace{k_{x},k_{y},k_{z}}^{T}$ denote the \emph{wavelength vector} and $k_{x},k_{y},k_{z}$ are real constants. 
Indeed, plugging \eqref{eq_prlm_waveEq_pSol} into \eqref{eq_prlm_waveEq} results in the \emph{monochromatic plane wave} satisfying:
\begin{equation}
\label{eq_prlm_waveEq_subs}
k_{x}^{2}+k_{y}^{2}+k_{z}^{2} = \frac{\omega^{2}}{c^{2}}.
\end{equation}
Using the plane wave notation
\begin{equation}
\label{eq_prlm_waveEq_finSol}
f\rBrace{t,\vecnot{x}}=A\exp{j\rBrace{\omega{t}-\vecnot{k}^{T}\vecnot{x}}},
\end{equation}
where $ \vecnot{x} = \vBrace{x,y,z}^{T} $ emphasizes the fact that for a given time, $t_{0}$, all points on a plane given by $k_{x}x+k_{y}y+k_{z}z = constant$ are with the same phase value where $\vecnot{k}\vecnot{x}=constant$ are planes of constant phase value.
The wave propagation can be described as the traveling of the planes, stating that small steps of both space $\delta{\vecnot{x}}$ and time $\delta{t}$ result in the same wave value i.e. $f\rBrace{t+\delta{t},\vecnot{x}+\delta{\vecnot{x}}} = f\rBrace{t,\vecnot{x}}$, which yields
\begin{equation}
\omega\delta{t}-\vecnot{k}^{T}\delta\vecnot{x}=0.
\end{equation}
Assuming $\delta\vecnot{x}$ and $\vecnot{k}$ have the same direction, $\vecnot{k}^{T}\delta\vecnot{x} = \abs{\vecnot{k}}\abs{\delta\vecnot{x}}$ and $\frac{\abs{\delta{\vecnot{x}}}}{\delta{t}} = \frac{\omega}{\abs{\vecnot{k}}}$, where $\frac{\abs{\delta{\vecnot{x}}}}{\delta{t}}$ can designate the propagation speed of the plane wave. 
Since $\vecnot{k}$ and $\omega$ are related by $\abs{\vecnot{k}}^{2}=\frac{\omega^{2}}{c^{2}}$, we have
\begin{equation}
\label{eq_prlm_waveEq_stepsEq}
\frac{\abs{\delta{\vecnot{x}}}}{\delta{t}}=c.
\end{equation}
The \emph{wavelength} ($\lambda$) denotes the distance the plane wave propagates during a single temporal period of $T=\frac{2\pi}{\omega}$.
Its magnitude $\abs{\vecnot{k}}$ expresses the number of cycles in radians per meter of length that the plane wave has exhibited in the propagation direction.
Using \eqref{eq_prlm_waveEq_stepsEq} with $\delta{t} = \frac{2\pi}{\omega}$, we obtain:
\begin{equation}
\lambda=\abs{\delta{\vecnot{x}}}=\frac{2\pi}{\abs{\vecnot{k}}}.
\end{equation}
Therefore, the wavenumber vector can be considered to represent spatial frequency, similarly to the manner $\omega$ represents temporal frequency.
\par For example, in the context of ULA, assuming far field scenario, the impinging waves are treated as constant phase planes as in \eqref{eq_prlm_waveEq_finSol}.
The spatial diversity of the ULA elements is expressed as a TOA difference of the impinging signal in each sensor which will be shown to provide clues for the signal's DOA as can be seen in Fig.~\ref{fig_ULA_sketch}.
\begin{figure}[h!]
    \begin{center}
        \begin{overpic}[width=0.6\linewidth, 
        % grid, 
        tics=10,trim=0 0 0 0]{./Media/arraySketch.png}
        \put(33.25,2.25){\rotatebox{0}{\tiny{$d$}}}
        \put(16,26){\rotatebox{-36}{\tiny{$c\tau=d\cos{\theta}$}}}
        \put(45.5,15.75){\rotatebox{0}{$\theta$}}
        \put(15,7){\rotatebox{0}{\tiny{0}}}
        \put(28,7){\rotatebox{0}{\tiny{1}}}
        \put(41,7){\rotatebox{0}{\tiny{2}}}
        \put(80,7){\rotatebox{0}{\tiny{$N-1$}}}
        \put(2,24){\rotatebox{-36}{\tiny{$t=0$}}}
        \put(2,34){\rotatebox{-36}{\tiny{$t=\tau$}}}
        \put(2,44){\rotatebox{-36}{\tiny{$t=2\tau$}}}
        \put(40,44){\rotatebox{-36}{\tiny{$t=\rBrace{N-1}\tau$}}}
        \end{overpic}
    \end{center}
     \caption{An illustration of an $N$ element ULA, impinged with a plane wave arriving from a DOA of $\theta$.}
    \label{fig_ULA_sketch}
\end{figure}
When also considering the narrowband scenario, where TOA difference is merely phase shift, it seems obvious that by measuring phases, one should be able to extract the DOA from the spatially sampled data.