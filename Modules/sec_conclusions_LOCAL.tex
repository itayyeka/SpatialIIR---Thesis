In this work we have suggested a novel approach to beamforming and localization, which is based on a generating a continuous feedback between the array and the target of interest.
Integrating feedback into standard beamformers proved to achieve the spatial domain equivalent of the temporal IIR filtering.
It seems that a simple generalization of the conventional-beamformer maximizes (locally) the system's spatial information, thus enabling high localization accuracy.
The feedback-based architecture performance evaluation predicts an unlimited improvement in all criteria, when considering perfect knowledge of the target's range and the channel attenuation.
It turns out that a single frequency waveform based feedback-beamformer is impractical, being too sensitive to even mild target range estimation errors.
Fortunately, using a DF waveform and applying simple frequency domain manipulations to the output and feedback signals, were found to serve as a low frequency (hence low sensitivity) equivalent of the single frequency scheme.
Also, the DF scheme proved to be of low noise sensitivity, featuring high performance even in relatively low signal-to-noise-ratio scenarios.
\par In the following, we overview some interesting leads for future study of the FB concept.