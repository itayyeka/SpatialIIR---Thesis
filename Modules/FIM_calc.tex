Following \eqref{eq_beamPatternFreqDomain_FIM}, we elaborate the steps leading to \eqref{eqn_FIMelements}. First, we express the parial derivatives of $\F{z}$ with respect to $\vEta$, resulting in
\begin{equation*}
    \resizebox{.9\linewidth}{!}{
        \begin{split}
            \frac{1}{\F{s}}\frac{\partial{\F{z}}}{\partial{\thetaD}} &= 
            \frac{
            g\vBetaH{}\vecnot{A}\omegaB\vd\ePhi{-}\rBrace{1-g\aHd\ePhi{-}}+g^{2}\bHd\vAlphaH{}\vecnot{A}\omegaB\vd\ePhi{-2}
            }{
            \rBrace{1-g\aHd\ePhi{-}}^{2}
            }
            \\&=
            \frac{
            g\vBetaH{}\vecnot{A}\omegaB\vd\ePhi{-}-g^{2}\vBetaH{}\rBrace{\vecnot{A}\omegaB\vd\vdT-\vd\vdT{}\vecnot{A}\omegaB}\vAlphaC\ePhi{-2}
            }{
            \rBrace{1-g\aHd\ePhi{-}}^{2}
            }
            \\&=
            \frac{
            g\vBetaH{}\vecnot{A}\omegaB\vd\ePhi{-}+g^{2}\vBetaH{}\vecnot{B}\omegaB\vAlphaC\ePhi{-2}
            }{
            \rBrace{1-g\aHd\ePhi{-}}^{2}
            }
        \end{split}
    }
\end{equation*}
and
\begin{equation*}
    \resizebox{.9\linewidth}{!}{
        \begin{split}
            \frac{1}{\F{s}}\frac{\partial{\F{z}}}{\partial{\phi}} &= 
            \frac{
            -jg\bHd\ePhi{-}\rBrace{1-g\aHd\ePhi{-}}-jg^{2}\bHd\aHd\ePhi{-2}
            }{
            \rBrace{1-g\aHd\ePhi{-}}^{2}
            }
            \\&=
            \frac{
            -jg\bHd\ePhi{-}
            }{
            \rBrace{1-g\aHd\ePhi{-}}^{2}
            }
        \end{split}
    },
\end{equation*}
where we defined $\vecnot{B}\omegaB\triangleq\vd\vdT{}\vecnot{A}\omegaB-\vecnot{A}\omegaB\vd\vdT$. 
The main diagonal elements of the FIM are 
\begin{equation*}
    \resizebox{.9\linewidth}{!}{
        \begin{split}
            J_{\thetaD\thetaD} &= \Re\cBrace{\frac{1}{2\pi\sigma^{2}}\int_{-\omega_{s}/2}^{\omega_{s}/2}\lBrace{\frac{\partial{\F{z}\omegaB}}{\partial{\thetaD}}}^{2}d\omega}
            \\&=
            \frac{1}{2\pi\sigma^{2}}\int_{-\omega_{s}/2}^{\omega_{s}/2}{\frac{
            \lBrace{g\vBetaH{}\vecnot{A}\omegaB\vd-g^{2}\vBetaH{}\vecnot{B}\omegaB\vAlphaC\ePhi{-}}^{2}
            }{
            \lBrace{1-g\aHd\ePhi{-}}^{4}
            }\lBrace{\F{s}\omegaB}^{2}d\omega},
            \\
            J_{\phi\phi} &= \Re\cBrace{\frac{1}{2\pi\sigma^{2}}\int_{-\omega_{s}/2}^{\omega_{s}/2}\lBrace{\frac{\partial{\F{z}\omegaB}}{\partial{\phi}}}^{2}d\omega}
            \\&=
            \frac{1}{2\pi\sigma^{2}}\int_{-\omega_{s}/2}^{\omega_{s}/2}{\frac{
            \lBrace{g\bHd}^{2}
            }{
            \lBrace{1-g\aHd\ePhi{-}}^{4}
            }\lBrace{\F{s}\omegaB}^{2}d\omega}
        \end{split}
    }
\end{equation*}
and the cross terms are
\begin{equation*}
    \resizebox{.85\linewidth}{!}{
        \begin{split}
            &J_{\thetaD\phi} = J_{\phi\thetaD}^{\ast} = 
            \\&= \Re\cBrace{\frac{1}{2\pi\sigma^{2}}\int_{-\omega_{s}/2}^{\omega_{s}/2}
            \rBrace{\frac{\partial{\F{z}\omegaB}}{\partial{\phi}}}^{\ast}
            \frac{\partial{\F{z}\omegaB}}{\partial{\thetaD}}d\omega}
            \\&=
            \Re\cBrace{\frac{1}{2\pi\sigma^{2}}\int_{-\omega_{s}/2}^{\omega_{s}/2}{\frac{
            jg^{2}\vBetaT\vdC\vBetaH{}\rBrace{\vecnot{A}\omegaB\vd+g\vecnot{B}\omegaB\vAlphaC\ePhi{-}}
            }{
            \lBrace{1-g\aHd\ePhi{-}}^{4}
            }\lBrace{\F{s}\omegaB}^{2}d\omega}}.
        \end{split}
    }.
\end{equation*}
Notice that when the weights are proportional to the conjugated steering vector, i.e., $$\vAlpha,\vBeta\propto\vd^{\ast},$$ the $\vBetaT{}B\omegaB\vAlpha$~term vanish.
% Setting the coefficients as $$\vAlpha,\vBeta\propto\vd^{\ast}$$ causes the $\vBetaT{}B\omegaB\vAlpha$~term to vanish. 
Assuming real input waveform $s(t)$, the function
\[
\frac{\lBrace{\F{s}\omegaB}^{2}}{\lBrace{1-g\aHd\ePhi{-}}^{4}}
\]
is even with respect to $\omega$, and $$\vBetaH{}\vecnot{A}\omegaB\vd\propto\vdH{}\vecnot{A}\vd=\sum_{n=0}^{N-1}A_{n,n}\abs{\vdI_{n}}^{2}\propto\omega$$ is odd, hence the cross terms vanish.
