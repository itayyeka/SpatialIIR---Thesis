Revisiting \eqref{eqn:SingleSensorTemporalEquality}, a natural initial generalization would be to allow the target of interest to be dynamic, i.e. introducing a general 2D arbitrary target location $\vecnot{p}_{t}\rBrace{t}$.
The location dynamics may be translated into a dynamic propagation delay, $\tau_{pd}\rBrace{t}$.
It follows that an auxiliary temporal quantity should be defined, which for a given impinging signal, expresses it's transmission time. 
Hence, we denote the moment of transmission $t_{i}\rBrace{t}$ as
\begin{equation}
    % t_{i}\rBrace{t}=\underset{\tau}{\mathrm{argmin}}\cBrace{\tau=\frac{R\rBrace{\tau}}{c}}.
    t_{i}\rBrace{t} = \cBrace{t - 2\tau\ \middle\vert\ t - \tau = \frac{R\rBrace{t-\tau}}{c}}
\end{equation}
With $t_{i}$, we rewrite \eqref{eqn:SingleSensorTemporalEquality} as
\begin{equation}
    \label{eqn:ftr_dyn_temp}
    x_{n}(t) = g\rBrace{s\rBrace{t_{i}\rBrace{t}}
    +\sum_{m=0}^{N-1}{\alpha^{*}_{m}x_{m}\rBrace{t_{i}\rBrace{t}}}}.
\end{equation}
One may consider the following research approaches for further investigation
\begin{itemize}
    \item \textbf{Modeling of target's location}
    \\Considering a constant velocity $\vecnot{v}_{\theta,\phi}$, of the target, where $v_{\theta}$ and $v_{\phi}$ are the radial and the tangent velocity components respectively, will result in a closed form expression for $t_{i}$, $$t_{i}=\frac{R\rBrace{t=0}+\rBrace{c+v_{\theta}}t}{c},$$ where $R\rBrace{t=0}$ is the initial distance to the target of interest and further development may lead to a closed form response like in the stationary case.
    In this manner, other components may be added to the positional equation.
    \item \textbf{Frame based processing}
    \\Considering low velocity of the target with respect to the wave's propagation's speed, one may consider treating the processor as pseudo-stationary and explore its implications for increasing velocities/processing duration. 
\end{itemize}