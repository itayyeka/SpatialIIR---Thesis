Revisiting \eqref{eqn:SingleSensorTemporalEquality}, a natural initial generalization would be to allow the target of interest to be dynamic, i.e. introducing a general 3D arbitrary target location $\vecnot{p}_{t}\rBrace{t}$.
The location dynamics may be represented in terms of dynamic propagation delay, $\tau_{pd}\rBrace{t}$ and Doppler.
% It follows that an auxiliary temporal quantity should be defined, which for a given impinging signal, expresses it's transmission time.
\par
Following the basic steps of temporal analysis, as in Chapter.~\ref{chap:firstchap}, we start with expressing the impinging signal of a single sensor.
As in the analysis of static target scenario, the signal consists of two contributions, i.e. the reflected emitted signal and the reflected feedback signal.
Therefore, for each moment $t$, we should resolve the corresponding moment in which the current impinging signals were emitted by the array.
\par
To this end, we first denote $t_{\text{ref}}\rBrace{t}$ as the moment in which the currently impinging signal was reflected by the target of interest.
Another auxiliary quantity is the reflection moment's corresponding target range $\tau_{\text{ref}}$ where the following relation holds
\begin{equation}
\label{eq_tRef_relation}
    t-\frac{R\rBrace{t_{\text{ref}}\rBrace{t}}}{c} = t_{\text{ref}}\rBrace{t}.
\end{equation}
Assuming $t_{\text{ref}}\rBrace{t}$ is resolved, we denote the propagation delay at the reflection moment $\tau_{\text{ref}}\rBrace{t} = R\rBrace{t_{\text{ref}}\rBrace{t}}/c$ and express the emission moment as
\begin{equation}
\label{eq_tTX}
    t_{\text{\tiny{TX}}}\rBrace{t} 
    = 
    t_{\text{ref}}\rBrace{t} - \tau_{\text{ref}}\rBrace{t} 
    =
    t - 2\tau_{\text{ref}}\rBrace{t}.
\end{equation}
With $t_{\text{\tiny{TX}}}$, we rewrite \eqref{eqn:SingleSensorTemporalEquality} as
\begin{equation}
    \label{eq_fut_dyn_temp}
    x_{n}(t) = g\rBrace{s\rBrace{t_{\text{\tiny{TX}}}\rBrace{t}}
    +\sum_{m=0}^{N-1}{\alpha^{*}_{m}x_{m}\rBrace{t_{\text{\tiny{TX}}}\rBrace{t}}}}.
\end{equation}
\par
Obviously, to have a closed form expression for $x_{n}(t)$, there should be a model for the target's dynamics (i.e. $\vpt{}$).
Such an example, is a target with constant radial (constant DOA) velocity $v$ with respect to the array which is initially positioned with distance of $R_{0}$ from the array.
It follows that
\begin{equation}
    \label{eq_fut_constV_Rt}
    R\rBrace{t} = R_{0} + vt
\end{equation}
which when plugged into \eqref{eq_tRef_relation} gives rise to

\begin{equation}
    \label{eq_fut_constV_ti}
    %\resizebox{.89\linewidth}{!}{
    \begin{split}
        t_{\text{ref}}\rBrace{t} &= \frac{c}{c+v}t-\frac{R_{0}}{c+v} \\
        \tau_{\text{ref}}\rBrace{t} &= \frac{R_{0}+vt}{c+v} \\
        t_{\text{\tiny{TX}}}\rBrace{t} &= t-2\tau_{\text{ref}}\rBrace{t} \\ 
        &= \frac{c-v}{c+v}t-\frac{2R_{0}}{c+v} \\
    \end{split}
\end{equation}

and it follows that
\begin{equation}
    \label{eq_fut_constV_xn}
    x_{n}(t) = g\vBrace{s\rBrace{\frac{c-v}{c+v}t-\frac{2R_{0}}{c+v}}
    +\sum_{m=0}^{N-1}{\alpha^{*}_{m}x_{m}\rBrace{\frac{c-v}{c+v}t-\frac{2R_{0}}{c+v}}}}.
\end{equation}
Then, similar steps as in Sec.~\ref{sec_introduceFeedback} are to be made to get an expression for the system's response $H$.
It is expected that as in RADAR systems, the tracking of the object should now also take Doppler shifts into account.
% One may consider the following research approaches for further investigation
% \begin{itemize}
%     \item \textbf{Modeling of target's location}
%     \\Considering a constant velocity $\vecnot{v}_{\theta,\phi}$, of the target, where $v_{\theta}$ and $v_{\phi}$ are the radial and the tangent velocity components respectively, will result in a closed form expression for $t_{\text{\tiny{TX}}}$, $$t_{\text{\tiny{TX}}}=\frac{R\rBrace{t=0}+\rBrace{c+v_{\theta}}t}{c},$$ where $R\rBrace{t=0}$ is the initial distance to the target of interest and further development may lead to a closed form response like in the stationary case.
%     In this manner, other components may be added to the positional equation.
%     \item \textbf{Frame based processing}
%     \\Considering low velocity of the target with respect to the wave's propagation's speed, one may consider treating the processor as pseudo-stationary and explore its implications for increasing velocities/processing duration. 
% \end{itemize}