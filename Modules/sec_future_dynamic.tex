Revisiting \eqref{eqn:SingleSensorTemporalEquality}, a natural initial generalization would be to allow the target of interest to be dynamic, i.e. introducing a general 3D arbitrary target location $\vecnot{p}_{t}\rBrace{t}$.
The location dynamics may be represented in terms of dynamic propagation delay, $\tau_{pd}\rBrace{t}$ and Doppler.
It follows that an auxiliary temporal quantity should be defined, which for a given impinging signal, expresses it's transmission time. 
Hence, we denote the moment of transmission $t_{i}\rBrace{t}$ as
\begin{equation}
\label{eq_fut_dyn_ti}
    t_{i}\rBrace{t} = \cBrace{t - 2\tau\ \middle\vert\ \tau = \frac{R\rBrace{t-\tau}}{c}}
\end{equation}
With $t_{i}$, we rewrite \eqref{eqn:SingleSensorTemporalEquality} as
\begin{equation}
    \label{eq_fut_dyn_temp}
    x_{n}(t) = g\rBrace{s\rBrace{t_{i}\rBrace{t}}
    +\sum_{m=0}^{N-1}{\alpha^{*}_{m}x_{m}\rBrace{t_{i}\rBrace{t}}}}.
\end{equation}
For example, consider a target with constant, radial only (constant DOA) velocity $v$ with respect to the array which is initially positioned with distance of $R_{0}$ from the array.
It follows that
\begin{equation}
    \label{eq_fut_constV_Rt}
    R\rBrace{t} = R_{0} + vt
\end{equation}
which when plugged into \eqref{eq_fut_dyn_ti} gives rise to
\begin{equation}
    \label{eq_fut_constV_ti}
    t_{i}\rBrace{t}=\frac{c-v}{c+v}t-\frac{2R_{0}}{c+v}
\end{equation}
and it follows that
\begin{equation}
    \label{eq_fut_constV_xn}
    x_{n}(t) = g\rBrace{s\rBrace{\frac{c-v}{c+v}t-\frac{2R_{0}}{c+v}}
    +\sum_{m=0}^{N-1}{\alpha^{*}_{m}x_{m}\rBrace{\frac{c-v}{c+v}t-\frac{2R_{0}}{c+v}}}}.
\end{equation}
Then, similar steps as in Sec.~\ref{sec_introduceFeedback} are to be made to get an expression for the system's response $H$.
It is expected that as in RADAR systems, the tracking of the object should now also take Doppler shifts as the returning signal is frequency shifted.
% One may consider the following research approaches for further investigation
% \begin{itemize}
%     \item \textbf{Modeling of target's location}
%     \\Considering a constant velocity $\vecnot{v}_{\theta,\phi}$, of the target, where $v_{\theta}$ and $v_{\phi}$ are the radial and the tangent velocity components respectively, will result in a closed form expression for $t_{i}$, $$t_{i}=\frac{R\rBrace{t=0}+\rBrace{c+v_{\theta}}t}{c},$$ where $R\rBrace{t=0}$ is the initial distance to the target of interest and further development may lead to a closed form response like in the stationary case.
%     In this manner, other components may be added to the positional equation.
%     \item \textbf{Frame based processing}
%     \\Considering low velocity of the target with respect to the wave's propagation's speed, one may consider treating the processor as pseudo-stationary and explore its implications for increasing velocities/processing duration. 
% \end{itemize}