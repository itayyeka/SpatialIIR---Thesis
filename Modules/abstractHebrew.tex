שיערוך מאפיינים מתוך מדידות הינו מושא לעניין רב עוד מראשית ימי המדע, כשחוקי הפיזיקה נלמדו בעזרת תצפיות ומדידות.
עם התקדמות המדע, רמת הדיוק הנדרשת בשיערוך עלתה ובכדי לעקוף את הגבולות התיאורטיים (פורייה למשל), עלה הרעיון לעיבוד מקבילי של מידע ממספר רגשים.
אחד התחומים בהם עלה משמעותית הצורך בדיוק השיערוך, היה העיבוד המרחבי - קרי שיערוך פרמטרים מרחביים מתוך מדידות. 
עדויות לכך ניתן למצוא עוד בתקופת מלחמת העולם השניה בדמות רדאר ה
\textenglish{Mammut}
שנבנה על ידי הגרמנים לזיהוי מוקדם של מטרות טסות בגבהים עד 8 ק"מ ומרחקים של עד 300 ק"מ.
\par
מאז ועד ימינו, עיבוד מערכי רגשים
\textenglish{(sensors arrays)}
התפתח לכדי תחום מחקר הנותן מענה למגוון רחב של בעיות, ביניהן איכון (מערכות רדאר, סונאר ונגזרותיהן) מטרות על סמך אותות שנפלטים או מוחזרים מהן - הנחשבת לותיקה ביותר בתחום.
עם השנים והתקדמות המדע, נעשה שימוש במערכים לפתרון בעיות נוספות.
\par
במערכות תקשורת, הצורך בסינון מרחבי הוא מן המעלה הראשונה, שכן במציאות מרובת המשדרים והמקלטים, במידה ולא היה שימוש במערכים, הפרעות הדדיות היו מונעות את פענוח האות הנקלט. במקביל, לא רק סינון אלא גם הכוונת אלומת השידור הפכה לאפשרית בזכות השימוש במערכי מופע
\textenglish{(phased array)}
ובכך הושגו גם חסכון באנרגית שידור, מניעת הגעת המידע ליעדים בלתי רצויים, חסינות לרעשים ועוד.
\par
מוקד עניין נוסף, הינו האפליקציות הרפואיות שהתאפשרו בזכות השימוש במערכים. 
בדימות רפואית למשל, כדוגמת 
\textenglish{CT, MRI, EEG}
וכו', רופאים מייצרים באופן בלתי פולשני מודלים של פנים גוף האדם המשמשים לפענוח מצבים רפואיים ותכנון מדויק של הליכים - ובכך מונעים סיכון רב מהמטופלים.
\par
בעשורים האחרונים, עם הגחת הטלפונים הניידים לעולם, עלה הצורך בסינון אותות דיבור מתוך בליל מקורות אקוסטיים, שכן הדובר כבר אינו נמצא בין כותלי ביתו בשקט יחסי.
לשם כך, המכשירים הניידים מצוידים במערכי מיקרופונים אשר מסננים את אות הדובר על פי כיוון הגעתו ומאפייניו הסטטיסטיים.
\par
דוגמא מעניינת נוספת בתחום האסטרונומיה, שם נוכחו המדענים כי תצפיות אופטיות מוגבלות ביכולתן ובכמות המידע שניתן להפיק מהן, היא השימוש במערכי אנטנות, הנפרשים על פני שטחים עצומים בכדי לממש טלסקופים רבי עוצמה ודיוק, העושים שימוש באותות אלקטרומגנטיים באורכי גל שלא בתחום הנראה.
\par
ככלל, בין תחומי מחקר אלו, מתקיימת תרומה הדדית רחבת היקף ורב המסקנות העולות מכל תחום בנפרד משמשות בסיס למחקר ופיתוח בתחומים אחרים ולהפך. 
בנוסף, בבסיס כל מגוון תחומי המחקר, ישנו שימוש בתחום מחקר ותיק ועשיר עוד יותר - הוא השימוש במסננים בכדי להפיק את המידע הרצוי (אות או מאפיין) מהאות הנקלט.  
\par
בבסיס תורת הסינון, שתי אבני בניין מרכזיות - קרי שני סוגי מסננים.
האחד, מסנן בעל תגובה להלם סופית 
\textenglish{FIR}
הינו מישקול וסכימת סט דגימות לכדי מוצא "מסונן".
השני לעומת זאת, משתמש בהיזון חוזר, ובכך מייצר תגובה אינסופית להלם - ועל כך שמו
\textenglish{IIR}
.
לשם תכן מסננים מסופק מפרט התגובה בתדר, בו מפורטים התדרים שאמורים לעבור, אלו שאמורים לחוות דיכוי ומידת הדיכוי הרצויה.
על פי מפרט זה, המתכנן בוחר את סוג המסנן ואת המשקלות שבו.
בעוד המסנן בעל התגובה הסופית להלם מאפשר הימנעות מעיוותים באות (תכונת הפאזה הלינארית - קרי ללא דיספרסיה), המסנן בעל התגובה האינסופית מאפשר חיסכון משמעותי (לעיתים בסדרי גודל) במשאבים לשם עמידה במפרט דומה.
איך לכך, בהתאם לאפליקציה ולמפרט, על המתכנן לבחור במסנן המתאים ביותר תוך הבאה בחשבון של כלל השיקולים - משאבים, עלות וכו'.
\par
עד כה, למיטב ידיעתנו, בכל תחומי המחקר המערבים עיבוד מרובה רגשים אבן הבניין היחידה שבשימוש הינה המסנן בעל התגובה הסופית בתדר, שכן בין אם מעורב שידור במערכת (כמו במכם, תקשורת, דימות רפואית וכו') ובין אם לאו, אין היזון חוזר של האות הנקלט חזרה אל מושא המדידה.
עובדה זו עוררה את תהייתנו ובעטייה חיפשנו את המקביל המרחבי למסנן בעל התגובה האינסופית - בכדי ליהנות מהיתרונות שבו.
במהלך המחקר המקדים, נוכחנו לדעת כי שאלה זו התעוררה גם במוחם של חוקרים אחרים, אך לאחר בחינת פרסומיהם נוכחנו כי פתרונם אינו מממש בצורה מלאה את רעיון ההיזון החוזר אלא מנסה למצוא דרכים עקיפות - בכך חושף עצמו לשגיאות ועיוותים משמעותיים.
\par
למשל, בעבודתו של WEN
\cite{wen2013extending}
עלתה הצעה לשערך את השהיית האות בין רגשים שכנים, וע"י כך לייצר היזון חוזר בצורה מלאכותית.
גישה זו חשופה לשגיאות מדידה וחוטאת למטרת העיבוד המרחבי הטהור - שכן היא מערבת עיבוד זמני.
הצעה נוספת בעבודה הנ"ל 
\cite{wen2013extending}
היתה להתייחס לתת מערכים (מתוך מערך אחד) חופפים כאל הסחות זמניות של מערך אחד קטן יותר ובכך לייצר קירוב סופי לתגובת ההלם האינסופית.
הצעה זו אינה אלא פרשנות שונה לעיבוד מבוסס 
\textenglish{FIR}
שכן בסופו של תהליך העיבוד הינו לינארי ומכיל מספר סופי של דגמים במערכת בכל זמן נתון.
שיטה מעניינת
\cite{Hum2009BeamformingFilters,madanayake2008speed}
, המהווה בסיס למחקר ענף, מטילה את הדגימה המרחבית והדגימה הזמנית אל מישור התדר, בו גל מישורי מיוצג על ידי קו ישר המוטה לפי זוית ההגעה של הגל הפוגע.
בהמשך לכך, מסננים במרחב התדר הדו מימדי מתוכננים על מנת להגביר אותות שמגיעים מכיוונים רצויים.
גישה זו בשימוש רחב בתחום החקר הסיסמי וסובלת בעיקר מחוסר היכולת לתכנן מסננים מדויקים היות והמסננים מחזוריים בתדר, עובדה הגורמת לעיוותים בקצוות מרחב התדר.
עוד על כן, גם שיטה זו אינה סינון מרחבי טהור ולכן אינה מהווה פתרון לשאלת המחקר אותה ניסינו לפתור.
\par
מוצעת איפוא, מערכת מבוססת מערך לינארי בעל מרווחים אחידים בין הרגשים
\textenglish{ULA}
כאשר בראשיתו, בנוסף על הרגש, ישנו משדר.
על פניו, מערכת דומה למכ"ם, הפולט אות ומשערך את מיקומי מטרות על פי האות החוזר.
החידוש בהצעה נעוץ בכך שהאות המשודר אינו רק עותק יחיד הנוצר ע"י המערכת אלא משולב בו האות המסונן הנקלט ע"י המערך, בכך נוצר היזון חוזר בין המערך והמטרה.
ניתוח המערכת מרמז כי ע"י שליטה במשקלות המסנן המייצר את האות החוזר, ניתן להפיק את היתרונות הרבים הטמונים במסנן בעל התגובה האינסופית.
\par
מחקירת המערכת המתקבלת, מצאנו שיפור משמעותי במגוון מאפיינים כגון רוחב אונה, הנחתת אונות צד ודיכוי רעשים איזוטרופים יעיל - הוה אומר כיווניות גבוהה.
לאחר מכן, הדמיות ובחינה דקדקנית של המערכת מגלים כי המערכת בעלת רגישות גבוהה למדי לשגיאת טווח - אבן נגף משמעותית בנסיון למציאת הפתרון.
עם זאת, לשמחתנו, נמצא גם פתרון פשוט וזול יחסית לבעיה, המגדיל סך הכל את המאמץ החישובי ולא את המערך עצמו.
הדמיות וניתוח נוספים מאשררים כי הפתרון קביל ומשיג תוצאות שבתרחישים אידיאלים נוטות לכדי שלמות.
בנוסף על האיכון הזוויתי שחווה שיפור משמעותי בביצועיו, כתופעת לוואי, מתכנן המערך יכול גם לקבוע גבולות מרחק לסינון ובכך לייצר מערכת איכון שמזהה לא רק כיוון אלא מיקום מדויק של מטרות.