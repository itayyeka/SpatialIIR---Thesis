בעבודה זו מוצגת שיטה חדשנית לעיבוד מרחבי של אותות במערכי חיישנים.
בבסיס השיטה נעשה שימוש במשוב מרחבי הניזון מהאותות הנקלטים במערך כך שלא מתבצע עיבוד זמני על האות הנקלט כי אם עיבוד מרחבי בלבד.
מניתוח ביצועי המערך עולה שיפור ניכר ביחס למערכים הידועים.
לשם נוחות, העבודה מוצגת בהקשר של איכון אך ניתנת להרחבה למגוון רחב של אפליקציות המשלבות עיבוד מרחבי של אותות ע"י מערכי חיישנים. 
\par
עיבוד מערכי חיישנים הינו תחום מחקר ענף הנותן מענה למגוון רחב של בעיות ועדויות מוקדמות לשימוש בו ניתן למצוא עוד בתקופת מלחמת העולם השניה בדמות רדאר ה
\textenglish{Mammut}
שנבנה על ידי הגרמנים לזיהוי מוקדם של מטרות טסות בגבהים עד 8 ק"מ ומרחקים של עד 300 ק"מ.
במערכות תקשורת לדוגמא, הצורך בסינון מרחבי הוא מן המעלה הראשונה לאור המציאות מרובת המשדרים והמקלטים. בנוסף, הכוונת אלומת השידור מתאפשרת בזכות השימוש במערכי מופע
\textenglish{(phased array)}
בכך מושגים חסכון באנרגית שידור, מניעת הגעת המידע ליעדים בלתי רצויים, חסינות לרעשים ועוד.
בדימות רפואית כדוגמת 
\textenglish{CT, MRI, EEG}
וכו', רופאים מייצרים מודלים של פנים גוף האדם באופן בלתי פולשני ע"י שימוש בעיבוד מרחבי מבוסס מערכי חיישנים.
בטלפונים הניידים מצויים מערכי מיקרופונים אשר מסננים את אות הדובר על פי כיוון הגעתו ומאפייניו הסטטיסטיים.
בתחום האסטרונומיה, מערכי אנטנות רדיו הנפרשים על פני שטחים עצומים משמשים הלכה למעשה טלסקופים רבי עוצמה ודיוק.
ככלל, בין תחומי מחקר אלו, מתקיימת תרומה הדדית רחבת היקף, כך שמסקנות העולות מתחום כלשהו לעיתים קרובות משמשות בסיס למחקר ופיתוח בתחומים אחרים.
\par
המערך היוניפורמי
\textenglish{(ULA)}
בו החיישנים מפוזרים על קו ישר במרחקים קבועים האחד מהשני נחשב למערך הבסיסי בעיבוד המרחבי בשל פשטות העיבוד והיישום שלו.
בנוסף, מוכרת בספרות ההקבלה בין עיצוב אלומת הסריקה (בהקשר של איכון) של מערך זה לבין תכן מסננים בעלי תגובה סופית להלם
\textenglish{(FIR)}.
הקבלה זו נובעת בין היתר מכך שבהינתן כיוון הגעת חזית הגל, הפרש הזמן בין הגעתה מחיישן אחד לשכנו זהה להפרש זמני ההגעה מהחיישן השכן אל הבא אחריו. תכונה זו מאפשרת הקבלה בין הפרשי זמני ההגעה לבין יחידות ההשהיה המשמשות למימוש המסננים הזמניים.
בנוסף, לאחר ניתוח מתמטי, ניתן לראות הקבלה בין כיוון הגעת חזית הגל בעיבוד המרחבי לבין התדר הזמני בעיבוד הזמני, לכן כיוון הגעת הגל נחשבת לתדר המרחבי.
\par
קיימת ארכיטקטורה נוספת למסננים, המממשת מסנן בעל תגובה אינסופית להלם 
\textenglish{IIR}
ע"י שילוב היזון חוזר.
מסננים אלו מאופיינים ביעילות מוגברת, כלומר עבור דרישות זהות בתגובת התדר, למסנן ה
\textenglish{IIR}
ידרשו פחות משאבים.
יתרון זה מהווה כבסיס לשאלת המחקר בתזה זו שהיא 
``מהו המקביל המרחבי למסנן בעל התגובה האינסופית?``, שכן בההקשר של מערכי חיישנים, חיסכון במשאבים מתבטא בהוזלה משמעותית ומאפשר את מזעור המערכת.
\par
במהלך סקר הספרות, התגלה כי שאלה זו התעוררה גם במוחם של חוקרים אחרים.
בעבודתו של 
\textenglish{WEN}~\cite{wen2013extending}
הוצעו שתי שיטות.
האחת מערבת שיערוך של כיוון הגעת הגל ויצירה מלאכותית של משוב.
שיטה זו רגישה לשגיאות שיערוך ומעבר לכך, מערבת עיבוד זמני ולכן אינה מתאימה כתשובה לשאלת המחקר.
השיטה השניה שהוצעה היתה הגדרת תתי מערכים (חלקים חופפים מהמערך) כך שאל מוצא כל מערך מתייחסים כאל הסחה זמנית של פלט מערך ייחוס כלשהו.
ניתוח מתמטי מוכיח כי בפועל העיבוד המרחבי לא משתפר היות ובפועל מדובר במערך בעל מספר זהה של  דרגות חופש ל
\textenglish{FIR}.
\par
גישה נוספת, אשר מקורה בעיבוד אותו וידאו 
\cite{bruton1985three}
, מתייחסת לכיוון ההגעה כאל תדר נוסף.
במישור הדו תדרי (משמע התדר המרחבי והתדר הזמני) גל מישורי מיוצג ע"י קו ישר המוטה בהתאם לזוית הגעתו אל המערך.
אי לכך, על מנת להגביר אותות מכיוונים רצויים, יש לתכנן מסננים המעבירים פסים ישרים במישור הדו תדרי התואמים לכיווני הגעה אלו.
היות ותגובת המסננים מחזורית בשני התדרים, נוצרים עיוותים בקצוות התגובה התדרית ובנוסף לכך, גם שיטה זו מערבת עיבוד זמני לכן אינה מהווה פתרון מספק לשאלת המחקר.
\par
כאמור, בעבודה זו מוצג לראשונה עקרון המסנן המשלב משוב מרחבי.
בעזרת שילוב משדר במערך, משודרת חזרה לזירה גרסה מעובדת של האות הנקלט.
היות ומניחים שהמטרה מחזירה את האותות המגיעים אליה, השידור בפועל אינו רק של האות הנקלט אלא כולל גם את  הדי האותות שנקלטו בעבר ובכך נוצר חוג סגור בו המידע המרחבי נאסף בהדרגה בכל פעם שהאות חוזר.
מניתוח המערך עולה כי בשילוב המשוב, תגובתו הינה המקבילה המרחבית למסנן בעל התגובה הזמנית האינסופית.
בנוסף, עולה כי התגובה אינה מושפעת רק מכיוון הגעת גל כמו במסננים המוכרים, אלא גם ממרחק המטרה מהמערך. 
\par
בשלב זה, על מנת לקבוע את משקלות המערך, הוגדרו המרחק וכיוון ההגעה כערכים אותם יש לשערך ובהתאם לכך חושבה מטריצת האינפורמציה 
\textenglish{FIM}.
בעזרת מטריצה זו נמצאו המקדמים המביאים את האינפורמציה לכדי שיא - בדיעבד, הרחבה טבעית של מקדמי האיסוף הקוהרנטי הבסיסי המוכר בספרות כמסנן הקונבנציונאלי.
מסנן זה מסיח את הפאזה של האותות המגיעים מחיישנים שונים כך שאות המגיע מכיוון כלשהו יאסף בהתאבכות בונה.
\par
ביצועי המערך מנותחים בהקשר של אפליקצית איכון וכוללים ניתוח של רוחב האונה המרכזית, הנחתת אונות הצד וכיווניות המערך.
לשם ניתוח זה נעשה שימוש בכלים מהספרות הקלאסית על מנת לאפשר השוואה אובייקטיבית אל המערכים המקובלים ללא המשוב.
כידוע, ביצועי המערך קשורים קשר ישיר למפתח הפיזי שלו לכן לשם המחשת התוצאות, השיפור בביצועי המערך מוצג כהגדלה וירטואלית של מפתח המערך.
בנוסף על כך הניתוח מראה כי ניתן ליישם את עקרון המשוב בכל מערך ולא רק ב 
\textenglish{(ULA)}.
\par
סימולציות בשילוב בחינה דקדקנית של המערכת מגלים כי המערכת בעלת רגישות גבוהה למדי לשגיאת טווח.
היות ושגיאה זו נובעת מהתדרים הנושאים, הוצע שימוש בלתי תלוי בשני תדרים קרובים תוך נסיון לחלץ את המידע המרחבי מהקשר בין האותות הנקלטים.
הפתרון פשוט וזול יחסית ומגדיל סך הכל את המאמץ החישובי ולא את המערך עצמו.
ניתוח נוסף מאשרר כי הפתרון קביל ומשיג תוצאות שבתרחישים אידיאלים נוטות לכדי שיפור משמעותי בביצועים המרחביים.