שיערוך מאפייני אותות הינו מושא לעניין רב עוד מראשית ימי המדע, כשחוקי הפיזיקה נלמדו בעזרת תצפיות ומדידות.
עם התקדמות המדע, רמת הדיוק הנדרשת בשיערוך עלתה ובכדי לעקוף את הגבולות התיאורטיים (פורייה למשל), עלה הרעיון לעיבוד מקבילי של מידע ממספר רגשים.
אחד התחומים בהם עלה משמעותית הצורך בדיוק השיערוך, היה העיבוד המרחבי - קרי שיערוך פרמטרים מרחביים מתוך מדידות. 
עדויות לכך ניתן למצוא עוד בתקופת מלחמת העולם השניה בדמות רדאר ה
\textenglish{Mammut}
שנבנה על ידי הגרמנים לזיהוי מוקדם של מטרות טסות בגבהים עד 8 ק"מ ומרחקים של עד 300 ק"מ.
\par
מאז ועד ימינו, עיבוד מערכי חיישנים
\textenglish{(sensors arrays)}
התפתח לכדי תחום מחקר הנותן מענה למגוון רחב של בעיות, ביניהן איכון (מערכות רדאר, סונאר ונגזרותיהן) מטרות על סמך אותות שנפלטים או מוחזרים מהן.
עם השנים והתקדמות המדע, נעשה שימוש במערכים לפתרון בעיות נוספות.
\par
במערכות תקשורת לדוגמא, הצורך בסינון מרחבי הוא מן המעלה הראשונה לאור המציאות מרובת המשדרים והמקלטים. במקביל, לא רק סינון אלא גם הכוונת אלומת השידור התאפשרה בזכות השימוש במערכי מופע
\textenglish{(phased array)}
ובכך הושגו גם חסכון באנרגית שידור, מניעת הגעת המידע ליעדים בלתי רצויים, חסינות לרעשים ועוד.
\par
מוקד עניין נוסף, הינו האפליקציות הרפואיות. 
בדימות רפואית למשל, כדוגמת 
\textenglish{CT, MRI, EEG}
וכו', רופאים מייצרים מודלים של פנים גוף האדם באופן בלתי פולשני.
מודלים אלו משמשים לפענוח מצבים רפואיים ותכנון מדויק של הליכים - ובכך מונעים סיכון רב מהמטופלים.
\par
בעשורים האחרונים, עם הגחת הטלפונים הניידים לעולם, עלה הצורך בסינון אותות דיבור טבולים ברעש, שכן הדובר כבר אינו נמצא בין כותלי ביתו בשקט יחסי.
לשם כך, המכשירים הניידים מצוידים במערכי מיקרופונים אשר מסננים את אות הדובר על פי כיוון הגעתו ומאפייניו הסטטיסטיים.
\par
דוגמא מעניינת נוספת בתחום האסטרונומיה, הינה מערכי אנטנות רדיו הנפרשים על פני שטחים עצומים בכדי לממש טלסקופים רבי עוצמה ודיוק. 
\par
ככלל, בין תחומי מחקר אלו, מתקיימת תרומה הדדית רחבת היקף ורב המסקנות העולות מכל תחום בנפרד משמשות בסיס למחקר ופיתוח בתחומים אחרים ולהפך. 
בנוסף, בבסיס כל מגוון תחומי המחקר, ישנו שימוש במסננים בכדי להפיק את המידע הרצוי (אות או מאפיין) מהאות הנקלט.  
\par
בבסיס תורת הסינון, שתי אבני בניין מרכזיות - קרי שני סוגי מסננים.
האחד, מסנן בעל תגובה להלם סופית 
\textenglish{FIR}
שהינו מישקול וסכימת סט דגימות לכדי מוצא "מסונן".
השני לעומת זאת, משלב היזון חוזר, ובכך מייצר תגובה אינסופית להלם 
\textenglish{IIR}
וידוע כי הוא מאפשר חיסכון משמעותי (לעיתים בסדרי גודל) במשאבים.
\par
עד כה, למיטב ידיעתנו, בכל תחומי המחקר המערבים עיבוד מערכי חיישנים אבן הבניין היחידה שבשימוש הינה המסנן בעל התגובה הסופית בתדר, שכן בין אם מעורב שידור במערכת (כמו במכם, תקשורת, דימות רפואית וכו') ובין אם לאו, אין היזון חוזר של האות הנקלט חזרה אל מושא המדידה.
עובדה זו עוררה את תהייתנו ובעטייה חיפשנו את המקביל המרחבי למסנן בעל התגובה האינסופית.
במהלך המחקר המקדים, נוכחנו לדעת כי שאלה זו התעוררה גם במוחם של חוקרים אחרים, אך הפתרונות שהציעו אינם מממשים רעיון ההיזון.
\par
בעבודה זו אנו מציעים פתרון המשלב, בנוסף על חיישני הקליטה, משדר לשם יצירת ההיזון החוזר.
על פניו, מערכת דומה למכ"ם, הפולט אות ומשערך את מיקומי מטרות על פי האות החוזר.
החידוש בהצעה נעוץ בכך שהאות המשודר אינו רק עותק יחיד הנוצר ע"י המערכת אלא משולבים בו הדי האותות שהוחזרו מן המטרה.
ניתוח המערכת מרמז כי ע"י שליטה במשקלות המסנן המייצר את האות החוזר, ניתן להפיק את היתרונות הרבים הטמונים במסנן בעל התגובה האינסופית.
\par
מחקירת המערכת המתקבלת, מצאנו שיפור משמעותי במגוון מאפיינים כגון רוחב אונה, הנחתת אונות צד ודיכוי רעשים איזוטרופים יעיל - כלומר מערך בעל כיווניות גבוהה.
לאחר מכן, הדמיות ובחינה דקדקנית של המערכת מגלים כי המערכת בעלת רגישות גבוהה למדי לשגיאת טווח - אבן נגף משמעותית בנסיון למציאת הפתרון.
עם זאת, לשמחתנו, נמצא גם פתרון פשוט וזול יחסית לבעיה, המגדיל סך הכל את המאמץ החישובי ולא את המערך עצמו.
הדמיות וניתוח נוספים מאשררים כי הפתרון קביל ומשיג תוצאות שבתרחישים אידיאלים נוטות לכדי שלמות.
