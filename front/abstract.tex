% Just write down your abstract here, no special commands necessary except for the \abstractEnglish{
% before this text is used and a closing } at the end of it

\abstractEnglish{
% State-of-the-art array processing methods, ranging from high-order statistics to adaptive configuration, require costly computing efforts in pursuit for spatial performance improvement.
The general field of array processing has been thoroughly studied throughout several decades.
The array sensors' spatial diversity enables the extraction of spatial information about impinging signals, thus laying the ground for wide range of applications.
The array size and the number of its elements ($N$) has significant influence on the obtained array performance, such as SNR improvement, spatial separation capabilities and its spatial response's degrees of freedom (DOF).
\par 
Considering localization related applications, several processing approaches are to be found. The most basic approach is \emph{beamforming}, i.e steering the array to different parts of the arena in search of objects.
Another approach, which is probably the most commonly used, is the subspace-based processing which tries to identify input signals that fit into the array's manifold using the orthogonal projection concept.
Statistical processing, although costly in terms of computation effort, is also in use, ranging from maximum-likelihood (ML) related schemes to higher-order-statistics (HOS) using cumulants.
Newer approaches which are still in research stage use neural networking and sparsity-based algorithms.
\par
Although much researched, in this work we revisit the \emph{beamforming} approach.
ULA based beamforming and temporal finite impulse response (FIR) filtering are mathematically analogous, where the DOA acts as the spatial version of temporal frequency.
Inspired by this analogy, we search for the spatial counterpart of the temporal infinite impulse response (IIR) filter.
To this end, we arbitrarily choose to use localization related problem formulation and suggest a feedback based approach, featuring low complexity and high spatial performance in the mere excess of integrating a transmitter to the array.
Considering RADAR-like arrays, the transmitter is already in place, hence a mere processing modification suffices.
Assuming the target of interest has a mirror-like behaviour (i.e., reflects its impinging signals), the spatial feedback between the array and the target is created by continuously re-transmitting a synthesized version of the impinging signal (and its reflections) to the target.
Hence a spatial loop is created, which is shown to be equivalent of enlarging the array's aperture in terms of spatial resolution.
Also notable is that the achievement of the exclusive IIR-like beampattern is purely done in the spatial domain while avoiding any temporal processing to the signal.
\par 
Using a traditional beamforming performance analysis, the beamwidth, peak to side-lobe ratio, array directivity and white noise sensitivity are evaluated for the feedback based array.
A significant improvement in all aspects is shown, while thoroughly discussing the conditions for enhanced performance.
Considering ideal scenarios, the feedback beamformer virtually achieves an infinite aperture, increasing the available spatial information about the target and significantly improves the array's spatial performance.
\par
Taking into account the unavoidable estimation errors and uncertainties, we find that the basic feedback integrated beamformer is very sensitive to even mild range errors.
% Also, the system is more sensitive as the carrier frequency increments.
% Considering practical carrier frequencies, the range sensitivity is of wavelength order. 
% Seeking for a solution, bearing in mind that summing multiple harmonics produce also other lower / higher frequencies, we propose a practical and robust implementation of the feedback-based localization concept.
As a solution, we propose a more complex architecture, using two harmonics, which we call \emph{dual-frequency feedback beamformer}.
We also thoroughly analyse the proposed solution and find that it features low and controllable estimation errors sensitivity in the mere expense of doubling the computation effort.

% At this point you write the abstract of your work, in the main language in which it is written (in this template - English). Graduate school regulations require the abstract to constitute an independent whole and be understood to a reader with general knowledge of the field. Use complete sentences and make few or no citations. Do not refer to the main body of the work and do not use uncommon shorthand, symbols and terms unless you have room for explaining them. The English abstract should be between 200 and 500 words long.

% So this should contain a few more paragraphs...

% \lipsum[10-12]

}

% Note that various commands don't work that well in Hebrew. Specifically,
% if you're using hyperref, you'll have trouble with \url, \autoref, \cite
% and friends. iitthesis-extra.sty has a workaround: the \disabledlinksL
% command. See the .sty for details, or:
% http://tex.stackexchange.com/q/32466/5640 for

\abstractHebrew{

% כאן יבוא תקציר מורחב בעברית (כאשר שפת החיבור העיקרית היא אנגלית). היקף התקציר יהיה \textenglish{1000-2000} מילים. התקציר יהווה שלמות בפני עצמו ויהיה מובן לקורא בעל ידיעות כלליות בנושא.

% בית הספר ללימודי מוסמכים מנחה מספר הנחיות לגבי התקציר בעברית:
% \begin{itemize}
% \item על התקציר להיכתב במשפטים מקושרים שלמים.
% \item בדרך-כלל אין לציין בתקציר מקורות ספרותיים וציטוטים.
% \item אין להתייחס למספר של פרק, סעיף, נוסחה, ציור או טבלה שבגוף החיבור, ואין להשתמש בקיצורים, סמלים ומונחים לא מקובלים, אלא אם יש בתקציר די מקום לזיהויים.
% \end{itemize}

% לעתים יש בכל-זאת יש צורך לכלול פקודה הכוללת קישור פנימי או חיצוני בתוך התקציר העברי; במצבים כאלו כדאי דרך-כלל לעטוף את הפקודה היוצרת את הקישור בתוך פקודת \textenglish{\texttt{\textbackslash{}textenglish\{\}}} כדי למנוע כל מיני פורענויות בלתי-רצויות, כגון כישלון בהידור קובץ ה-\textenglish{PDF} או שימוש בגופן העברי באופן אשר עלול שלא להנעים לעין. לדוגמה: נניח שיש לנו צורך לצטט מקור ביבליוגרפי. אם נעשה זאת סתם-כך: \textenglish{\texttt{\textbackslash{}cite\{Hoeffding\}}}, נקבל: \cite{Hoeffding}; אם נעטוף את פקודת הציטוט, כך: \textenglish{\texttt{\textbackslash{}textenglish\{\textbackslash{}cite\{Hoeffding\}\}}}, נקבל \textenglish{\cite{Hoeffding}} (כפי שהציטוטים נראים גם בטקסט באנגלית).

% \subsection*{\texthebrew{תת-חלק בתקציר המורחב}}

% תוכן מקוצר לגבי נושא מסוים. התייחסות ל\emph{מושג} מסוים שהחיבור בוחן. וכולי וכולי.


% \subsection*{\texthebrew{נקודה מעניינת לגבי העמודים בעברית}}

% שימו לב כי העמודים בעברית אמורים להיות מיוצרים בסדר ה''הפוך'', הווה אומר העמוד האחרון בקובץ ה-\textenglish{PDF} הוא הכריכה העברית, לפניו השער העברי, ודפי התקציר צריכים להופיע בסדר הפוך (וכן במספור רומי, לפי נהלי הטכניון). כך אם נתבונן במספר שבתחתית עמוד זה \textenglish{---} אשר צריך להיות העמוד הראשון בתקציר-המורחב מבחינת רצף התוכן, והינו העמוד האחרון מבין עמודי התקציר-המורחב אחרון בקובץ ה-\textenglish{PDF} \textenglish{---} נמצא את המספר \textenglish{i} ...

% \newpage

% ... ואילו עמוד זה של התקציר-המורחב בעברית \textenglish{---} שהינו העמוד השני בתקציר-המורחב מבחינת רצף התוכן, ונמצא ראשון בקובץ ה-\textenglish{PDF} \textenglish{---} ממוספר ב-\textenglish{ii}. המטרה במספור בסדר ה"הפוך" היא, שבעת ההדפסה לא יהיה צורך להפוך דפים, לשנות את סדרם וכולי \textenglish{---} רק להדפיס ולכרוך.

שיערוך מאפיינים מתוך מדידות הינו מושא לעניין רב עוד מראשית ימי המדע, כשחוקי הפיזיקה נלמדו בעזרת תצפיות ומדידות.
עם התקדמות המדע, רמת הדיוק הנדרשת בשיערוך עלתה ובכדי לעקוף את הגבולות התיאורטיים (פורייה למשל), עלה הרעיון לעיבוד מקבילי של מידע ממספר רגשים.
אחד התחומים בהם עלה משמעותית הצורך בדיוק השיערוך, היה העיבוד המרחבי - קרי שיערוך פרמטרים מרחביים מתוך מדידות. 
עדויות לכך ניתן למצוא עוד בתקופת מלחמת העולם השניה בדמות רדאר ה
\textenglish{Mammut}
שנבנה על ידי הגרמנים לזיהוי מוקדם של מטרות טסות בגבהים עד 8 ק"מ ומרחקים של עד 300 ק"מ.
\par
מאז ועד ימינו, עיבוד מערכי רגשים
\textenglish{(sensors arrays)}
התפתח לכדי תחום מחקר הנותן מענה למגוון רחב של בעיות, ביניהן איכון (מערכות רדאר, סונאר ונגזרותיהן) מטרות על סמך אותות שנפלטים או מוחזרים מהן - הנחשבת לותיקה ביותר בתחום.
עם השנים והתקדמות המדע, נעשה שימוש במערכים לפתרון בעיות נוספות.
\par
במערכות תקשורת, הצורך בסינון מרחבי הוא מן המעלה הראשונה, שכן במציאות מרובת המשדרים והמקלטים, במידה ולא היה שימוש במערכים, הפרעות הדדיות היו מונעות את פענוח האות הנקלט. במקביל, לא רק סינון אלא גם הכוונת אלומת השידור הפכה לאפשרית בזכות השימוש במערכי מופע
\textenglish{(phased array)}
ובכך הושגו גם חסכון באנרגית שידור, מניעת הגעת המידע ליעדים בלתי רצויים, חסינות לרעשים ועוד.
\par
מוקד עניין נוסף, הינו האפליקציות הרפואיות שהתאפשרו בזכות השימוש במערכים. 
בדימות רפואית למשל, כדוגמת 
\textenglish{CT, MRI, EEG}
וכו', רופאים מייצרים באופן בלתי פולשני מודלים של פנים גוף האדם המשמשים לפענוח מצבים רפואיים ותכנון מדויק של הליכים - ובכך מונעים סיכון רב מהמטופלים.
\par
בעשורים האחרונים, עם הגחת הטלפונים הניידים לעולם, עלה הצורך בסינון אותות דיבור מתוך בליל מקורות אקוסטיים, שכן הדובר כבר אינו נמצא בין כותלי ביתו בשקט יחסי.
לשם כך, המכשירים הניידים מצוידים במערכי מיקרופונים אשר מסננים את אות הדובר על פי כיוון הגעתו ומאפייניו הסטטיסטיים.
\par
דוגמא מעניינת נוספת בתחום האסטרונומיה, שם נוכחו המדענים כי תצפיות אופטיות מוגבלות ביכולתן ובכמות המידע שניתן להפיק מהן, היא השימוש במערכי אנטנות, הנפרשים על פני שטחים עצומים בכדי לממש טלסקופים רבי עוצמה ודיוק, העושים שימוש באותות אלקטרומגנטיים באורכי גל שלא בתחום הנראה.
\par
ככלל, בין תחומי מחקר אלו, מתקיימת תרומה הדדית רחבת היקף ורב המסקנות העולות מכל תחום בנפרד משמשות בסיס למחקר ופיתוח בתחומים אחרים ולהפך. 
בנוסף, בבסיס כל מגוון תחומי המחקר, ישנו שימוש בתחום מחקר ותיק ועשיר עוד יותר - הוא השימוש במסננים בכדי להפיק את המידע הרצוי (אות או מאפיין) מהאות הנקלט.  
\par
בבסיס תורת הסינון, שתי אבני בניין מרכזיות - קרי שני סוגי מסננים.
האחד, מסנן בעל תגובה להלם סופית 
\textenglish{FIR}
הינו מישקול וסכימת סט דגימות לכדי מוצא "מסונן".
השני לעומת זאת, משתמש בהיזון חוזר, ובכך מייצר תגובה אינסופית להלם - ועל כך שמו
\textenglish{IIR}
.
לשם תכן מסננים מסופק מפרט התגובה בתדר, בו מפורטים התדרים שאמורים לעבור, אלו שאמורים לחוות דיכוי ומידת הדיכוי הרצויה.
על פי מפרט זה, המתכנן בוחר את סוג המסנן ואת המשקלות שבו.
בעוד המסנן בעל התגובה הסופית להלם מאפשר הימנעות מעיוותים באות (תכונת הפאזה הלינארית - קרי ללא דיספרסיה), המסנן בעל התגובה האינסופית מאפשר חיסכון משמעותי (לעיתים בסדרי גודל) במשאבים לשם עמידה במפרט דומה.
איך לכך, בהתאם לאפליקציה ולמפרט, על המתכנן לבחור במסנן המתאים ביותר תוך הבאה בחשבון של כלל השיקולים - משאבים, עלות וכו'.
\par
עד כה, למיטב ידיעתנו, בכל תחומי המחקר המערבים עיבוד מרובה רגשים אבן הבניין היחידה שבשימוש הינה המסנן בעל התגובה הסופית בתדר, שכן בין אם מעורב שידור במערכת (כמו במכם, תקשורת, דימות רפואית וכו') ובין אם לאו, אין היזון חוזר של האות הנקלט חזרה אל מושא המדידה.
עובדה זו עוררה את תהייתנו ובעטייה חיפשנו את המקביל המרחבי למסנן בעל התגובה האינסופית - בכדי ליהנות מהיתרונות שבו.
במהלך המחקר המקדים, נוכחנו לדעת כי שאלה זו התעוררה גם במוחם של חוקרים אחרים, אך לאחר בחינת פרסומיהם נוכחנו כי פתרונם אינו מממש בצורה מלאה את רעיון ההיזון החוזר אלא מנסה למצוא דרכים עקיפות - בכך חושף עצמו לשגיאות ועיוותים משמעותיים.
\par
למשל, עלתה הצעה לשערך את השהיית האות בין רגשים שכנים, וע"י כך לייצר היזון חוזר בצורה מלאכותית.
גישה זו חשופה לשגיאות מדידה וחוטאת למטרת העיבוד המרחבי הטהור - שכן היא מערבת עיבוד זמני.
הצעה נוספת היתה להתייחס לתת מערכים (מתוך מערך אחד) חופפים כאל הסחות זמניות של מערך אחד קטן יותר ובכך לייצר קירוב סופי לתגובת ההלם האינסופית.
הצעה זו אינה אלא פרשנות שונה לעיבוד מבוסס 
\textenglish{FIR}
שכן בסופו של תהליך העיבוד הינו לינארי ומכיל מספר סופי של דגמים במערכת בכל זמן נתון.
שיטה מעניינת, המהווה בסיס למחקר ענף, מטילה את הדגימה המרחבית והדגימה הזמנית אל מישור התדר, בו גל מישורי מיוצג על ידי קו ישר המוטה לפי זוית ההגעה של הגל הפוגע.
בהמשך לכך, מסננים במרחב התדר הדו מימדי מתוכננים על מנת להגביר אותות שמגיעים מכיוונים רצויים.
גישה זו בשימוש רחב בתחום החקר הסיסמי וסובלת בעיקר מחוסר היכולת לתכנן מסננים מדויקים היות והמסננים מחזוריים בתדר, עובדה הגורמת לעיוותים בקצוות מרחב התדר.
עוד על כן, גם שיטה זו אינה סינון מרחבי טהור ולכן אינה מהווה פתרון לשאלת המחקר אותה ניסינו לפתור.
\par
מוצעת איפוא, מערכת מבוססת מערך לינארי בעל מרווחים אחידים בין הרגשים
\textenglish{ULA}
כאשר בראשיתו, בנוסף על הרגש, ישנו משדר.
על פניו, מערכת דומה למכ"ם, הפולט אות ומשערך את מיקומי מטרות על פי האות החוזר.
החידוש בהצעה נעוץ בכך שהאות המשודר אינו רק אות מלאכותי הנוצר ע"י המערכת אלא משולב בו האות המסונן הנקלט ע"י המערך, בכך נוצר היזון חוזר בין מהמערך והמטרה.
ניתוח המערכת מרמז כי ע"י שליטה במשקלות המסנן המייצר את האות החוזר, ניתן להפיק את היתרונות הרבים הטמונים במסנן בעל התגובה האינסופית.
\par
לאחר מכן, הדמיות ובחינה דקדקנית של המערכת מגלים כי המערכת בעלת רגישות גבוהה למדי - אבן נגף משמעותית בנסיון למציאת הפתרון.
עם זאת, לשמחתנו, נמצא גם פתרון פשוט וזול יחסית לבעיה, המגדיל סך הכל את המאמץ החישובי ולא את המערך עצמו.
הדמיות וניתוח נוספים מאשררים כי הפתרון קביל ומשיג תוצאות שבתרחישים אידיאלים נוטות לכדי שלמות.
בנוסף על האיכון הזוויתי שחווה שיפור משמעותי בביצועיו, כתופעת לוואי, מתכנן המערך יכול גם לקבוע גבולות מרחק לסינון ובכך לייצר מערכת איכון שמזהה לא רק כיוון אלא מיקום מדויק של מטרות.


}