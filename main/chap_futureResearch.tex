In the following chapter, some of the many possibilities for further investigation of the feedback beamformer are suggested.
As mentioned in Sec.~\ref{sec_introduceFeedback}, we assumed a single stationary target scenario where the target's location is fixed.
Naturally, the first suggested subjects for future research are dynamic (Sec.\ref{sec_future_dynamic}) and multiple (Sec.\ref{sec_future_multiTarget}) target scenarios.
Another setting which may be an interesting subject for future research is the waveform characteristics (Sec.~\ref{sec_future_wavfrom}) and it's affect on the overall spatial performance.
We choose to conclude with another final suggestion (Sec.~\ref{sec_future_coef}) - i.e. to revisit Sec.\ref{sec_FIM}'s FIM related considerations to the choice of array settings. 
\subsection{Dynamic target}
\label{sec_future_dynamic}
Revisiting \eqref{eqn:SingleSensorTemporalEquality}, a natural initial generalization would be to allow the target of interest to be dynamic, i.e. introducing a general 3D arbitrary target location $\vecnot{p}_{t}\rBrace{t}$.
The location dynamics may be represented in terms of dynamic propagation delay, $\tau_{pd}\rBrace{t}$ and Doppler.
% It follows that an auxiliary temporal quantity should be defined, which for a given impinging signal, expresses it's transmission time.
\par
Following the basic steps of temporal analysis, as in Chapter.~\ref{chap:firstchap}, starts with expressing the impinging signal of a single sensor.
As in the analysis of static target scenario, the signal consists of two contributions, i.e. the reflected emitted signal and the reflected feedback signal.
Therefore, for each moment $t$, we should resolve the corresponding moment in which the current impinging signals were emitted by the array.
\par
To this end, we first denote $t_{\text{ref}}\rBrace{t}$ the moment in which the currently impinging signal was reflected by the target of interest.
Another auxiliary quantity is reflection moment's corresponding target range $\tau_{\text{ref}}$ where the following relation holds
\begin{equation}
\label{eq_tRef_relation}
    t-\frac{R\rBrace{t_{\text{ref}}\rBrace{t}}}{c} = t_{\text{ref}}\rBrace{t}.
\end{equation}
Assuming $t_{\text{ref}}\rBrace{t}$ is resolved, we denote the propagation delay at the reflection moment $\tau_{\text{ref}}\rBrace{t} = R\rBrace{t_{\text{ref}}\rBrace{t}}/c$ and express the emission moment as
\begin{equation}
\label{eq_tTX}
    t_{\text{\tiny{TX}}}\rBrace{t} 
    = 
    t_{\text{ref}}\rBrace{t} - \tau_{\text{ref}}\rBrace{t} 
    =
    t - 2\tau_{\text{ref}}\rBrace{t}.
\end{equation}
With $t_{\text{\tiny{TX}}}$, we rewrite \eqref{eqn:SingleSensorTemporalEquality} as
\begin{equation}
    \label{eq_fut_dyn_temp}
    x_{n}(t) = g\rBrace{s\rBrace{t_{\text{\tiny{TX}}}\rBrace{t}}
    +\sum_{m=0}^{N-1}{\alpha^{*}_{m}x_{m}\rBrace{t_{\text{\tiny{TX}}}\rBrace{t}}}}.
\end{equation}
\par
Obviously, to have a closed form expression for $x_{n}(t)$, there should be a model for the target's dynamics (i.e. $\vpt{}$).
Such an example, is a target with constant radial (constant DOA) velocity $v$ with respect to the array which is initially positioned with distance of $R_{0}$ from the array.
It follows that
\begin{equation}
    \label{eq_fut_constV_Rt}
    R\rBrace{t} = R_{0} + vt
\end{equation}
which when plugged into \eqref{eq_tRef_relation} gives rise to
\begin{equation}
    \label{eq_fut_constV_ti}
    t_{\text{\tiny{TX}}}\rBrace{t}=\frac{c-v}{c+v}t-\frac{2R_{0}}{c+v}
\end{equation}
and it follows that
\begin{equation}
    \label{eq_fut_constV_xn}
    x_{n}(t) = g\vBrace{s\rBrace{\frac{c-v}{c+v}t-\frac{2R_{0}}{c+v}}
    +\sum_{m=0}^{N-1}{\alpha^{*}_{m}x_{m}\rBrace{\frac{c-v}{c+v}t-\frac{2R_{0}}{c+v}}}}.
\end{equation}
Then, similar steps as in Sec.~\ref{sec_introduceFeedback} are to be made to get an expression for the system's response $H$.
It is expected that as in RADAR systems, the tracking of the object should now also take Doppler shifts into account.
% One may consider the following research approaches for further investigation
% \begin{itemize}
%     \item \textbf{Modeling of target's location}
%     \\Considering a constant velocity $\vecnot{v}_{\theta,\phi}$, of the target, where $v_{\theta}$ and $v_{\phi}$ are the radial and the tangent velocity components respectively, will result in a closed form expression for $t_{\text{\tiny{TX}}}$, $$t_{\text{\tiny{TX}}}=\frac{R\rBrace{t=0}+\rBrace{c+v_{\theta}}t}{c},$$ where $R\rBrace{t=0}$ is the initial distance to the target of interest and further development may lead to a closed form response like in the stationary case.
%     In this manner, other components may be added to the positional equation.
%     \item \textbf{Frame based processing}
%     \\Considering low velocity of the target with respect to the wave's propagation's speed, one may consider treating the processor as pseudo-stationary and explore its implications for increasing velocities/processing duration. 
% \end{itemize}
\subsection{Multi-target scenario}
\label{sec_future_multiTarget}
Considering practical scenarios of multiple ($P$) targets, positioned in $p_{t_,i}\ \forall i\in0\dots{}P-1$, the obvious expansion of \eqref{eqn:SingleSensorTemporalEquality} is 
\begin{equation}
    \label{eqn:ftr_dyn_temp}
    x_{n}(t) = \sum_{i=0}^{i=P-1}{\rBrace{g\rBrace{s\rBrace{t-\tau_{pd,i}-\tau_{n,i}}
    +\sum_{m=0}^{N-1}{\alpha^{*}_{m}x_{m}\rBrace{t-\tau_{pd,i}-\tau_{n,i}}}}}},
\end{equation}
where $\tau_{pd,i}$ is the delay of the signal which reflected from $p_{t,i}$.
Interesting directions of research may include
\begin{itemize}
    \item The affects of closely positioned targets
    \item Localization resolution
    \item Maximal number of simultaneously detected targets with respect to N
    \item Independent beamforming per target with multiple transmitters steering the feedback signal. 
\end{itemize}
\subsection{Waveform modifications}
\label{sec_future_wavfrom}
In practical RADAR related systems, the waveform of choice is pulse (chirp) based.
It holds some obvious advantages over the CW transmission such as
\begin{itemize}
    \item Lower power consumption
    \item Inherent nulling in the radial axis (localization)
    \item Possible match filters should allow the decision of weather to re-transmit or not
    \item In multi target scenario, there is a smaller chance of interference.
    \item The most interesting research option is the use of several (possibly many) independent harmonics as super positioned multiple dual frequency beamformers, which simultaneously scan a wide area with multiple resolutions, by merely selecting harmonic couples and filter coefficients, where each harmonic/coefficient set it pointed to a specific area.
    In the expense of computational effort (without adding elements to the array), it seems possible that one may have simultaneous scan of the entire arena in any wanted direction/range/resolution.
\end{itemize}
\subsection{Coefficients quality criteria}
\label{sec_future_coef}
As mentioned in Sec.~\ref{sec_introduceFeedback}, the reasoning for the selection of $\vecnot{\alpha}, \vecnot{\beta}$ as the conventional beamformer generalization, was the FIM related information considerations.
Obviously, there are many more possibilities/considerations to choosing the coefficient.
Some examples for such coefficient choosing schemes may include
\begin{itemize}
    \item Finite word affects
    \item Nulling of specific DOAs
    \item Setting all poles in a single DOA may lead to enhanced directivity
    \item General radiation patterns of the sensors
\end{itemize}
and many more.
% \section{Apply to other applications}
% \label{sec_future_applications}
% \input{Modules/sec_future_applications}