\chapter{Preliminaries}
\label{chap:prelims}

In this chapter, we overview the theoretical basis of the thesis. 
In section 1...
In section 2...
In section 3...

\begin{definition}
The \emph{von Neumann model} of a computer, also known as the \emph{Princeton architecture} is an architecture for digital computers, which consists of a processing units, containing an ALU and processing registers; a control unit consisting of an instruction register and a program counter; a memory unit which stores both data and instructions; and input-and-output mechanisms.
\end{definition}

\section{Propagating Wave Fields}
\label{sec:prlm_propWaveField}
% In array signal processing, propagating waves carry signals from the source to the array.
% Therefore, these signals are represented in the time-space domain. The space domain is represented by either the three dimensional Cartesian coordinates $\rBrace{x,y,z}$ or the three dimensional spherical coordinates $\rBrace{r,\phi,\theta}$, where $0 \leq \phi \leq 2\pi$, $0 \leq \theta \leq \pi$ are the azimuth and elevation angles, respectively. The time domain is represented by $t$.
An elementary physical phenomenon is the spatial dynamics of waves, referred to as wave-propagation.
The spatial location is noted by the Cartesian coordinates $\rBrace{x,y,z}$ or the three dimensional spherical coordinates $\rBrace{r,\phi,\theta}$, where $0 \leq \phi \leq 2\pi$, $0 \leq \theta \leq \pi$ are the azimuth and elevation angles, respectively. 
% Finally, let $f\rBrace{t,\vecnot{r}}$ denote the time-space representation of the input signal, where $\vecnot{r}$ is the radius-vector in the three-dimensional system. 
% The relations between the Cartesian and spherical coordinates are given in Fig.~\ref{fig_coordinates}.
Denoting $t$ as the time, the time-space representation of the a signal is $f\rBrace{t,\vecnot{r}}$ or , where the relations between the Cartesian and spherical coordinates are given in Fig.~\ref{fig_coordinates}.
% $f\rBrace{t,x,y,z}$ describes the signal impinging the microphone array. 
In a homogeneous, dispersion free and lossless medium the wave equation is:
\begin{equation}
\label{eq_prlm_waveEq}
\nabla^{2}f\rBrace{t,x,y,z}=\frac{1}{c^2}\frac{\partial^{2}f\rBrace{t,x,y,z}}{\partial{t^{2}}}
\end{equation}
where $\nabla^{2}$ is the Laplacian operator and $c$ represents the wave's velocity in the medium.
\begin{figure}[h!]
    \begin{center}
        \begin{overpic}[width=0.5\linewidth, 
        %grid, 
        tics=10,trim=0 0 0 0]{./Media/fig_coordinates.png}
            % \put (50, 62.5) {\footnotesize{$r=0$}}
        \end{overpic}
    \end{center}
     \caption{A three dimensional coordinate system with Cartesian and spherical coordinates.}
    \label{fig_coordinates}
\end{figure}
A possible solution to \eqref{eq_prlm_waveEq}, $f_p\rBrace{t,x,y,z}$, may be of a complex exponential form:
\begin{equation}
\label{eq_prlm_waveEq_pSol}
f_p\rBrace{t,x,y,z} = A\exp{j\rBrace{\omega{t}-k_{x}x-k_{y}y-k_{z}z}}
\end{equation}
Where $A$ is a complex constant, $\omega$ denotes a constant temporal frequency and $k_{x}x,k_{y},k_{z}$ are real constants. 
Plugging \eqref{eq_prlm_waveEq_pSol} into \eqref{eq_prlm_waveEq} results in the \emph{monochromatic plane wave}:
\begin{equation}
\label{eq_prlm_waveEq_subs}
k_{x}^{2}-k_{y}^{2}-k_{z}^{2} = \frac{\omega^{2}}{c^{2}}.
\end{equation}
Using the plane wave notation emphasizes the fact that for a given time, $t_{0}$, all points on a plane given by $k_{x}x+k_{y}y+k_{z}z = constant$ are with the same wave value and the vector notation of \eqref{eq_prlm_waveEq}'s solution is
\begin{equation}
\label{eq_prlm_waveEq_finSol}
f\rBrace{t,\vecnot{x}}=A\exp{j\rBrace{\omega{t}-\vecnot{k}\vecnot{x}}}.
\end{equation}
where $\vecnot{k}\vecnot{x}=constant$ are planes of constant wave value.
The wave propagation can be described as the traveling of the planes, stating that small steps of both space $\delta{\vecnot{x}}$ and time $\delta{t}$ result in the same wave value i.e. $f\rBrace{t+\delta{t},\vecnot{x}+\delta{\vecnot{x}}} = f\rBrace{t,\vecnot{x}}$, which yields
\begin{equation}
\omega\delta{t}-\vecnot{k}\delta\vecnot{x}=0.
\end{equation}
Assuming $\delta\vecnot{x}$ and $\vecnot{k}$ have the same direction, $\vecnot{k}\delta\vecnot{x} = \abs{\vecnot{k}}\abs{\delta\vecnot{x}}$ and $\frac{\delta{\vecnot{x}}}{\delta{t}} = \frac{\omega}{\abs{\vecnot{k}}}$, where $\frac{\delta{\vecnot{x}}}{\delta{t}}$ can designate the propagation speed of the plane wave. 
Since $\vecnot{k}$ and $\omega$ are related by $\abs{\vecnot{k}}^{2}=\frac{\omega^{2}}{c^{2}}$, we have
\begin{equation}
\label{eq_prlm_waveEq_stepsEq}
\frac{\delta{\vecnot{x}}}{\delta{t}}=c,
\end{equation}
where $c>0$ is the wave velocity in the medium.
The \emph{wavelength} ($\lambda$) denote the distance the plane wave propagated during a single temporal period of $T=\frac{2\pi}{\omega}$.
Let $\vecnot{k}$ denote the \emph{wavelength vector}. 
Its magnitude $\abs{\vecnot{k}}$ expresses the number of cycles in radians per meter of length that the plane wave has exhibited in the propagation direction.
Using \eqref{eq_prlm_waveEq_stepsEq} with $\delta{t} = \frac{2\pi}{\omega}$, we obtain:
\begin{equation}
\lambda=\delta{\vecnot{x}}=\frac{2\pi}{\abs{\vecnot{k}}}.
\end{equation}
Therefore, the wavenumber vector can be considered to represent spatial frequency, similarly to the manner $\omega$ represents temporal frequency.
\section{Sensor arrays}
\label{sec:prlm_sensorArrays}
Sensor arrays are sets of independently positioned sensors, where each sensor is sampling temporal snapshots of impinging signals.
The samples from the entire array are then fused together in order to extract the underlying data.
The spatial diversity of the sampled data allows the extraction of spatial features.
Sensor arrays are used in numerous applications, ranging from source localization, communication, medical applications, astronomy etc.
\par 
When designing a sensor array, multiple consideration are to be taken into account, ranging from physical characteristics e.g. area, weight and carrying platform, through performance related considerations such as accuracy, spatial selectivity, signal to noise ratio and even the financial aspect.
The most fundamental array specification is its geometry, for it dictates the spatial relations between simultaneous samples measured by the array's sensors, which also greatly influences the processing methods of the raw-data.
A basic example of a sensor array is the ULA (see Fig.~\ref{fig_ULA}), where its elements are uniformly spaced on a straight line with distance $d$ between each pair of sensors.
Considering an array of $N$ elements, the sensors are positioned at
\begin{equation}
p_{n}=nd,\ n=0,\dots,N-1
\end{equation}
where $n$ is the sensors index; $p_{n}$ denotes the position of the $n$'th sensor, $p_{0} = 0$ corresponds to the left-hand side of the array as illustrated in Fig.~\ref{fig_ULA}.
\begin{figure}[h!]
    \begin{center}
        \begin{overpic}[width=0.5\linewidth, 
        %grid, 
        tics=10,trim=0 0 0 0]{./Media/arrayBasic.png}
        \put (4, 4.5) {\tiny{$N$}}
        \put (86, 4.5) {\tiny{$N-1$}}
        \put (83.25, 28.5) {\tiny{$\rBrace{N-1}\cdot{}d$}}
        \end{overpic}
    \end{center}
     \caption{A uniform linear array of size $N$ and spacing $d$.}
    \label{fig_ULA}
\end{figure}
\par To properly formulate the discussed additional spatial information when using sensor arrays, a quick overview on wave propagation is due and presented in the following section (Sec.~\ref{sec:prlm_propWaveField}).

\section{Time-Space Signals}
\label{sec:prlm_timeSpaceSig}
As mentioned before, the measurements of the sensor array's elements are fused together, where spatial information resides in the relations between samples taken the same time $t$.
A basic application of the sensor array is the beamformer, designed to enhance signals from certain directions while suppressing other signals from unwanted directions, forming a beam directed to the enhanced direction.  
\begin{figure}[ht!]
    \begin{center}
        \begin{overpic}[width=0.6\linewidth, 
        %grid, 
        tics=10,trim=0 0 0 0]{./Media/arrayProc.png}
        \put(20,65){\rotatebox{-50}{\tiny{Impinging wavefront}}}
        \put(49,33.25){\rotatebox{0}{$\theta$}}
        \put(15,25){\rotatebox{0}{\tiny{0}}}
        \put(28,25){\rotatebox{0}{\tiny{1}}}
        \put(41,25){\rotatebox{0}{\tiny{2}}}
        \put(80,25){\rotatebox{0}{\tiny{$N-1$}}}
        \put(16,20){\rotatebox{0}{\tiny{$\omega_{0}$}}}
        \put(29.5,20){\rotatebox{0}{\tiny{$\omega_{1}$}}}
        \put(42,20){\rotatebox{0}{\tiny{$\omega_{2}$}}}
        \put(81,20){\rotatebox{0}{\tiny{$\omega_{N-1}$}}}
        \put(50.5,3.5){\rotatebox{0}{\tiny{$z\rBrace{t}$}}}
        \end{overpic}
    \end{center}
     \caption{ULA based coherent processing of simultaneous spatially sampled signal, measuring a wavefront impinging from DOA $\theta$}
    \label{fig_ULA_imping}
\end{figure}
In complimentary to Fig.~\ref{fig_ULA_sketch}, Fig.~\ref{fig_ULA_imping} adds the processor layer of the beamformer.
\par A temporal snapshot measured by the array at time $t$ is
\begin{equation}
\vecnot{f}\rBrace{t,\vecnot{p}} = \vBrace{f\rBrace{t,p_{0}},\dots,f\rBrace{t,p_{N-1}}}^{T},
\end{equation}
where $p_{n}$ denotes the $n$-th sensor position, $t$ denotes the continuous-time variable and $^{T}$ denotes the transpose operation. 
Considering the \emph{conventional beamformer} \cite{van2004optimum}, the output ($z\rBrace{t}$) is expressed by:
\begin{equation}
z\rBrace{t} = \sum_{n=0}^{N-1}{f\rBrace{t,p_{n}}\omega_{n}^{*}},
\end{equation}
where $\omega_{n}$ is a complex weight used for the $n$'th sensor's output, as shown in Fig.~\ref{fig_ULA_imping} and $^{*}$ denotes complex conjugation. 
Consider $s\rBrace{t} = e^{j\omega{t}}$ as a plane wave propagating at temporal frequency $\omega$, and $\theta \in \vBrace{-\frac{\pi}{2},\frac{\pi}{2}}$ as the direction of arrival (DOA) angle measured with respect to the broadside of the linear array, as shown in Fig.~\ref{fig_ULA_imping}.
The wave signal, spatially sampled by the sensor array inputs is
\begin{equation}
\label{eq_prlm_timeSpace_outputVec}
\vecnot{f}\rBrace{t,x} = \vBrace{s\rBrace{t},s\rBrace{t-\tau}\dots, s\rBrace{t-\rBrace{N-1}\tau}}^{T},
\end{equation}
where $\tau = \frac{d\cdot{}\cos{\theta}}{c}$ is the propagation delay between consecutive sensors and $c$ is the wave's velocity in the medium.
Considering narrowband signals, the time delay $\tau$ translates to a mere phase shift and the \emph{steering vector} takes the form of
\begin{equation}
\vecnot{d}\rBrace{\theta}=\vBrace{1,e^{-j\omega\tau},\dots,e^{-j\omega\tau\rBrace{N-1}}}^{T}.
\end{equation}
As every impinging signal will cause the excitation of the array elements in the form of a specific steering vector (up to a common additive phase), we define the \emph{array manifold} to be the set of all steering vectors of $\theta\in\left[0,2\pi\right)$, hence it also spans the received signal sub space in the noiseless scenario.
\section{Sensor array performance measure}
\label{sec:prlm_sensorArrayPerf}
Numerous performance measures are utilized for evaluating the microphone array capabilities. 
Each of the measures aims to quantify a significant aspect of either the response of an array to the signal environment or of the sensitivity to an array design error.
First, we review the concept of beampatterns. 
Beampatterns are the main tool in assessing an array performance as they express the beamformer response for various frequencies and signal angles. 
Let $P\rBrace{\omega,\theta}$ denote the frequency and DOA dependent beamformer response. It is given by:
\begin{equation}
P\rBrace{\omega,\theta}=\sum_{m=0}^{M-1}e^{-j\omega\tau_{m}w_{m}^{*}}=\vecnot{w}^{H}\vecnot{v}\rBrace{\omega,\theta}    
\end{equation}
where $\vecnot{w}=\vBrace{w_{0},\dots,w_{M-1}}^{T}$ and $^{H}$ denotes Hermitian transpose operation, and let $D\rBrace{\omega,\theta}$ denote the beampattern of a given beamformer. 
It is expressed by:
\begin{equation}
D\rBrace{\omega,\theta}=20\log_{10}\abs{P\rBrace{\omega,\theta}},
\end{equation}
where $\abs{\cdot}$ denotes the absolute value. 
For a given angular frequency $\omega$, the beampattern $D\rBrace{\omega,\theta}$ is a function of the angle $\theta$ and the beamwidth is measured in terms of $\theta$.
In this work, the beamwidth is measured between the two lowest values at both sides of the main lobe.
\par
A major challenge in practical beamformer applications is the potential sensitivity to mismatches between the actual array attributes and the model used to derive the desired beamformer. 
In practical applications, mismatches can occur either by array location perturbations, production faults or filter perturbations. 
The sensitivity function often used as a criterion for assessing the affect of mismatches on the array response is defined in \cite{van2004optimum} by:
\begin{equation}
T_{se}=A^{-1}_{w}=\abs{\abs{\vecnot{w}}}^{2},
\end{equation}
where $A_{w}^{-1}$ is the inverse expression of the white noise gain given by $A_{w}=\text{SNR}_{out}\rBrace{k}/\text{SNR}_{in}\rBrace{k}$ and $\vecnot{w}$ is the weight vector corresponding to all of the FIR filter channels in the $k$-th frequency bin. Therefore, as the white noise gain increases, the sensitivity decreases and the array would be more robust to mismatch.
\section{Finite Impulse Response Based Spatial Filtering}
\label{sec:prlm_FIR}
Delay-and-sum beamformers, as shown in Fig.~\ref{XXX}, utilize a single weight for each microphone. 
This makes them ineffective when dealing with wideband signals, the signals of interest in speech and audio processing, as each of the signals is composed of various frequency components. 
In order to design a beamformer for wideband signals, the weight values in \eqref{XXX} must be altered for different frequencies in order to obtain the desired beamformer output. 
That is, the weights should be frequency dependent, i.e., in the form of $\vecnot{w}\rBrace{\omega}=\vBrace{w_{0}\rBrace{\omega},\dots,w_{M-1}\rBrace{\omega}}^{T}$. 
This can be acheived by introducing discrete FIR filters \cite{yan2005design,liu2010wideband}.
FIR filters perform temporal filtering in order to compensate for the phase differences of the input wideband signals’ various frequency components. 
Fig.\ref{XXX} shows the realization of frequency dependent weights by FIR filters connected to the microphone array.
Let $y\rBrace{t}$ denote the output of the FIR-based beamformer. 
It is expressed by:
\begin{equation}
y\rBrace{t}=\sum_{m=0}^{M-1}\sum_{n=0}^{N-1}f\rBrace{t-nT_{s},x_{m}}\cdot{}w_{m,n}^{*}
\end{equation}
where $N$ denotes the number of FIR filter coefficients connected to each of the $M$ sensors, $w_{m,n}^{*}$ is the $n$-th coefficient of the FIR filter connected to the microphone indexed by $m$, and $T_{s}$ denotes the delay between adjacent filter elements. 
Given a complex plane wave signal, the beamformer output is given by:
\begin{equation}
y\rBrace{t}=e^{j\omega{}t}P\rBrace{\omega,\theta}=e^{j\omega{}t}\sum_{m=0}^{M-1}\sum_{n=0}^{N-1}e^{-j\omega\rBrace{\tau_{m}+nT_{s}}}\cdot{}w_{m,n}^{*}
\end{equation}
Let $\vecnot{v}_{s}\rBrace{\omega,\theta}$ denote a \emph{stacked array manifold vector} of dimension $M\cdot{}N$, where each subvector of dimension $M$ epresents the array manifold vector associated with a specific FIR filter coefficient in \eqref{XXX}, i.e. the first subvector, $\vBrace{e^{-j\omega\tau_{0}},\dots,e^{-j\omega\tau_{M-1}}}^{T}$, is associated with the coefficient indexed by $n=0$ and all of the array microphones, indexed by $m=0,\dots,M-1$.
Thus, it is denoted by $\vecnot{v}_{0}\rBrace{\omega,\theta}$.
The second subvector, $\vBrace{e^{=j\omega\rBrace{\tau_{0}+T_{s}}},\dots,e^{=j\omega\rBrace{\tau_{M-1}+T_{s}}}}^{T}$, is associated with the coefficient indexed by $n=1$ and all of the array microphones.
Thus, it is denoted by $\vecnot{v}_{1}\rBrace{\omega,\theta}$, and so on.
This form of expression is called vector stacking, and $\vecnot{v}_{s}\rBrace{\omega,\theta}$ is given by:
\begin{align}
\vecnot{v}_{s}\rBrace{\omega,\theta} &= 
\begin{bmatrix}
   \vecnot{v}_{0}\rBrace{\omega,\theta} \\
   \vecnot{v}_{1}\rBrace{\omega,\theta} \\
   \vdots \\
   \vecnot{v}_{N-1}\rBrace{\omega,\theta}
\end{bmatrix}.
\end{align}
Given a ULA of $M$ sensors, sensor spacing $d$, and $M$ FIR filters, each composed of $N$ coefficients, and connected to a respective microphone. 
Then, from \eqref{XXX} and
\eqref{XXX}, the beamformer response can be expressed as:
\begin{equation}
    P\rBrace{\omega,\theta}=    \sum_{m=0}^{M-1}e^{-j\omega\tau_{m}}\sum_{n=0}^{N-1}e^{jn\omega{}T_{s}}\cdot{}w_{m,n}^{*}=\vecnot{w}_{s}^{H}\vecnot{v}_{s}\rBrace{\omega,\theta},
\end{equation}
where $\vecnot{w}_{s}$ denotes the composite stacked weight vector of dimension $M\cdot{}N$ created by vector stacking, having $\vecnot{w}_{0}=\vBrace{w_{0,0},w_{1,0},\dots,w_{M-1,0}}^{T}$ as its first subvector, $\vecnot{w}_{1}=\vBrace{w_{0,1},w_{1,1},\dots,w_{M-1,1}}^{T}$ as its second subvector, and so on.
The design specification of the FIR filters will be addressed in Chapter \ref{XXX}. Fig.\ref{XXX} illustrates an FIR beamformer architecture.
\section{Infinite Impulse Response Filters}
\label{sec:prlm_IIR}
\section{Fisher Information Matrix}
\label{sec:prlm_FIM}

Let's define some concept we'll be using throughout the thesis.



