\chapter{Preliminaries}
\label{chap:prelims}

As a preliminary discussion to the presentation of out contribution, we overview some the key theoretical basis of the thesis.
In Sec.~\ref{sec:prlm_propWaveField}, the wave propagation, being the most basic concept for array processing, is discussed.
It is then followed by a short overview of array-processing in Sec.~\ref{sec:prlm_sensorArrays}.
To accommodate the discussion for spatial filtering, we also mention the time-space relation of spatially sampled signals in Sec.~\ref{sec:prlm_timeSpaceSig}.
As an additional theoretical concept, arriving from the information theory, which will be used in the main chapter (Ch.~\ref{chap:firstchap}), is the Fisher Information Matrix, covered in Sec.~\ref{sec:prlm_FIM}.
\par 
For an easier reading treat each section as independent, where notations and symbols are each defined in the specific section and are not mutual, nor used in the main chapter. 

% \begin{definition}
% The \emph{von Neumann model} of a computer, also known as the \emph{Princeton architecture} is an architecture for digital computers, which consists of a processing units, containing an ALU and processing registers; a control unit consisting of an instruction register and a program counter; a memory unit which stores both data and instructions; and input-and-output mechanisms.
% \end{definition}

\section{Propagating Wave Fields}
\label{sec:prlm_propWaveField}
% In array signal processing, propagating waves carry signals from the source to the array.
% Therefore, these signals are represented in the time-space domain. The space domain is represented by either the three dimensional Cartesian coordinates $\rBrace{x,y,z}$ or the three dimensional spherical coordinates $\rBrace{r,\phi,\theta}$, where $0 \leq \phi \leq 2\pi$, $0 \leq \theta \leq \pi$ are the azimuth and elevation angles, respectively. The time domain is represented by $t$.
An elementary physical phenomenon is the spatial dynamics of waves, referred to as wave-propagation.
The spatial location is noted by the Cartesian coordinates $\rBrace{x,y,z}$ or the three dimensional spherical coordinates $\rBrace{r,\phi,\theta}$, where $0 \leq \phi \leq 2\pi$, $0 \leq \theta \leq \pi$ are the azimuth and elevation angles, respectively. 
% Finally, let $f\rBrace{t,\vecnot{r}}$ denote the time-space representation of the input signal, where $\vecnot{r}$ is the radius-vector in the three-dimensional system. 
% The relations between the Cartesian and spherical coordinates are given in Fig.~\ref{fig_coordinates}.
Denoting $t$ as the time, the time-space representation of the a signal is $f\rBrace{t,\vecnot{r}}$ or , where the relations between the Cartesian and spherical coordinates are given in Fig.~\ref{fig_coordinates}.
% $f\rBrace{t,x,y,z}$ describes the signal impinging the microphone array. 
In a homogeneous, dispersion free and lossless medium the wave equation is:
\begin{equation}
\label{eq_prlm_waveEq}
\nabla^{2}f\rBrace{t,x,y,z}=\frac{1}{c^2}\frac{\partial^{2}f\rBrace{t,x,y,z}}{\partial{t^{2}}}
\end{equation}
where $\nabla^{2}$ is the Laplacian operator and $c$ represents the wave's velocity in the medium.
\begin{figure}[h!]
    \begin{center}
        \begin{overpic}[width=0.5\linewidth, 
        %grid, 
        tics=10,trim=0 0 0 0]{./Media/fig_coordinates.png}
            % \put (50, 62.5) {\footnotesize{$r=0$}}
        \end{overpic}
    \end{center}
     \caption{A three dimensional coordinate system with Cartesian and spherical coordinates.}
    \label{fig_coordinates}
\end{figure}
A possible solution to \eqref{eq_prlm_waveEq}, $f_p\rBrace{t,x,y,z}$, may be of a complex exponential form:
\begin{equation}
\label{eq_prlm_waveEq_pSol}
f_p\rBrace{t,x,y,z} = A\exp{j\rBrace{\omega{t}-k_{x}x-k_{y}y-k_{z}z}}
\end{equation}
Where $A$ is a complex constant, $\omega$ denotes temporal radial frequency and $k_{x}x,k_{y},k_{z}$ are real constants. 
Plugging \eqref{eq_prlm_waveEq_pSol} into \eqref{eq_prlm_waveEq} results in the \emph{monochromatic plane wave}:
\begin{equation}
\label{eq_prlm_waveEq_subs}
k_{x}^{2}-k_{y}^{2}-k_{z}^{2} = \frac{\omega^{2}}{c^{2}}.
\end{equation}
Using the plane wave notation emphasizes the fact that for a given time, $t_{0}$, all points on a plane given by $k_{x}x+k_{y}y+k_{z}z = constant$ are with the same wave value and the vector notation of \eqref{eq_prlm_waveEq}'s solution is
\begin{equation}
\label{eq_prlm_waveEq_finSol}
f\rBrace{t,\vecnot{x}}=A\exp{j\rBrace{\omega{t}-\vecnot{k}\vecnot{x}}}.
\end{equation}
where $\vecnot{k}\vecnot{x}=constant$ are planes of constant wave value.
The wave propagation can be described as the traveling of the planes, stating that small steps of both space $\delta{\vecnot{x}}$ and time $\delta{t}$ result in the same wave value i.e. $f\rBrace{t+\delta{t},\vecnot{x}+\delta{\vecnot{x}}} = f\rBrace{t,\vecnot{x}}$, which yields
\begin{equation}
\omega\delta{t}-\vecnot{k}\delta\vecnot{x}=0.
\end{equation}
Assuming $\delta\vecnot{x}$ and $\vecnot{k}$ have the same direction, $\vecnot{k}\delta\vecnot{x} = \abs{\vecnot{k}}\abs{\delta\vecnot{x}}$ and $\frac{\delta{\vecnot{x}}}{\delta{t}} = \frac{\omega}{\abs{\vecnot{k}}}$, where $\frac{\delta{\vecnot{x}}}{\delta{t}}$ can designate the propagation speed of the plane wave. 
Since $\vecnot{k}$ and $\omega$ are related by $\abs{\vecnot{k}}^{2}=\frac{\omega^{2}}{c^{2}}$, we have
\begin{equation}
\label{eq_prlm_waveEq_stepsEq}
\frac{\delta{\vecnot{x}}}{\delta{t}}=c,
\end{equation}
where $c>0$ is the wave velocity in the medium.
The \emph{wavelength} ($\lambda$) denote the distance the plane wave propagated during a single temporal period of $T=\frac{2\pi}{\omega}$.
Let $\vecnot{k}$ denote the \emph{wavelength vector}. 
Its magnitude $\abs{\vecnot{k}}$ expresses the number of cycles in radians per meter of length that the plane wave has exhibited in the propagation direction.
Using \eqref{eq_prlm_waveEq_stepsEq} with $\delta{t} = \frac{2\pi}{\omega}$, we obtain:
\begin{equation}
\lambda=\delta{\vecnot{x}}=\frac{2\pi}{\abs{\vecnot{k}}}.
\end{equation}
Therefore, the wavenumber vector can be considered to represent spatial frequency, similarly to the manner $\omega$ represents temporal frequency.
\par For example, in the context of ULA, assuming far field scenario, the impinging waves are treated as constant phase planes as in \eqref{eq_prlm_waveEq_finSol}.
The spatial diversity of the ULA elements is expressed as a TOA difference of the impinging signal in each sensor which is will be shown to provide clues for the signal's DOA as can be seen in Fig.~\ref{fig_ULA_sketch}.
\begin{figure}[h!]
    \begin{center}
        \begin{overpic}[width=0.6\linewidth, 
        %grid, 
        tics=10,trim=0 0 0 0]{./Media/arraySketch.png}
        \put(20,45){\rotatebox{-50}{\tiny{Impinging wavefront}}}
        \put(33.25,2.25){\rotatebox{0}{\tiny{$d$}}}
        \put(52.25,26){\rotatebox{38}{\tiny{$c\tau=d\cos{\theta}$}}}
        \put(49,15.75){\rotatebox{0}{$\theta$}}
        \put(15,7){\rotatebox{0}{\tiny{0}}}
        \put(28,7){\rotatebox{0}{\tiny{1}}}
        \put(41,7){\rotatebox{0}{\tiny{2}}}
        \put(80,7){\rotatebox{0}{\tiny{$N-1$}}}
        \put(44,36){\rotatebox{38}{\tiny{$t=\rBrace{N-1}\tau$}}}
        \put(57,36){\rotatebox{38}{\tiny{$t=\rBrace{N-2}\tau$}}}
        \put(70,36){\rotatebox{38}{\tiny{$t=\rBrace{N-3}\tau$}}}
        \put(92,25){\rotatebox{38}{\tiny{$t=0$}}}
        \end{overpic}
    \end{center}
     \caption{An illustration of an $N$ element ULA, impinged with a plane wave arriving from a DOA of $\theta$.}
    \label{fig_ULA_sketch}
\end{figure}
When also considering the narrowband scenario, where TOA difference is merely phase shift, it seems obvious that by phase measuring, one should be able to extract the DOA from the spatially sampled data.
% \section{Digital Filter Design - Finite/Infinite Response Filters}
% \label{sec:prlm_FILTERS}
% A basic building block of any signal processing system is the digital filter, which manipulates input signals according to some predetermined specifications. 
Nominally, digital filters are applied to temporal-domain sampled signals, while designed to meet certain frequency-domain specification.
The most  
\section{Sensor arrays}
\label{sec:prlm_sensorArrays}
Microphone arrays consist of sets of microphones positioned in a way to function as a directional acoustic antenna. 
Microphone arrays are utilized for filtering signals in a space-time field. 
Filtering is enabled by exploitation of the incoming signals spatial characteristics. 
A desirable spatial filtering, i.e., beamforming, should result in enhancement of signals of interest, originated in a specific direction, while forcing suppression of undesired signals originated in other directions.
Microphone arrays are utilized for solving many signal processing problems: dereverberation, localization of a single source, noise reduction and source separation \cite{cohen2004multichannel,habets2006dual,pavlidi2013real}.
\par 
Numerous factors must be taken into consideration when designing a microphone array configuration.
Initially, the geometry of the microphone array plays an important role in the formulation of the processing algorithms, as it forces fundamental constraints on the array’s operation. 
In most cases, the array geometry is the first consideration in array design due to practical and physical constraints of the design. 
Therefore, the degree of freedom in choosing the array geometry is limited. 
Nevertheless, in some other crucial problems such as noise reduction or source separation, the geometry of the array may have little importance. 
For instance, Uniform Linear Arrays (ULA) can only handle one source direction at a time resulting in uncertainties and possibly direction ambiguities.
Therefore, ULA configurations must be ruled out when handling several signal sources.
Other array geometries such as non-uniform linear arrays and circular arrays, have been studied in the field \cite{liu2008design,van2004optimum}.
This work is based on ULA and therefore we elaborate on this geometry.
A conventional beamformer \cite{van2004optimum} is composed of a ULA of microphones. 
That is, the microphones are positioned on an axis with a uniform spacing between the microphones.
The general expression of the microphone positioning in a ULA is given by:
\begin{equation}
x_{m}=m\cdot{d}, m=0, 1, \dots, M-1
\end{equation}
where $m$ denotes the microphone index; $x_{m}$ denotes the position of the microphone having index $m$, where $x_{m} = 0$ corresponds to the left-hand side of the array, i.e., the location where the microphone indexed by $m = 0$ is positioned; $d$ denotes the spacing between microphones, and $M$ is the array size, i.e., the number of microphones in the array, as demonstrated in Fig.~\ref{fig_ULA}.
\begin{figure}[h!]
    \begin{center}
        \begin{overpic}[width=0.5\linewidth, 
        %grid, 
        tics=10,trim=0 0 0 0]{./Media/fig_ULA.png}
            % \put (50, 62.5) {\footnotesize{$r=0$}}
        \end{overpic}
    \end{center}
     \caption{A uniform linear array of size M and spacing d.}
    \label{fig_ULA}
\end{figure}
\section{Time-Space Signals}
\label{sec:prlm_timeSpaceSig}
The incoming signals are spatially sampled by the array microphones, then the samples are processed to attenuate signals from undesired directions and extract the signal from a desired direction. 
A spatial response of the microphone array is obtained via a beam (main-lobe) directed to the desired signal while nulls are directed to the undesired signals.
\begin{figure}[ht!]
    \begin{center}
        \begin{overpic}[width=0.5\linewidth, 
        %grid, 
        tics=10,trim=0 0 0 0]{./Media/fig_ULA_imping.png}
            % \put (50, 62.5) {\footnotesize{$r=0$}}
        \end{overpic}
    \end{center}
     \caption{A simple beamformer design.}
    \label{fig_ULA_imping}
\end{figure}
Fig.~\ref{fig_ULA_imping} illustrates a beamformer design based on a ULA of Fig.~\ref{fig_ULA}.
Let $\vecnot{f}\rBrace{t,\vecnot{x}}$ denote the set of signals sampled by the microphone array at time $t$, expressed by
\begin{equation}
\vecnot{f}\rBrace{t,\vecnot{x}} = \vBrace{f\rBrace{t,x_{0}},\dots,f\rBrace{t,x_{M-1}}}^{T},
\end{equation}
where $x_{m}$ denotes the microphone position, $t$ denotes the continuous-time variable and $^{T}$ denotes the transpose operation. 
The beamformer output $y\rBrace{t}$ is expressed by:
\begin{equation}
y\rBrace{t} = \sum_{m=0}^{M-1}{f\rBrace{t,x_{m}}\omega_{m}^{*}},
\end{equation}
where $\omega_{m}$ is a complex weight of the microphone having index $m$, as shown in Fig.~\ref{fig_ULA_imping} and $^{*}$ denotes complex conjugation. 
The simplest beamformer design architecture has uniform weights, $\omega_{m}=\frac{1}{M}, m=0,\dots,M-1$.
Let $f_{\omega}\rBrace{t,x} = e^{j\omega{t}}$ denote a plane wave propagating at angular frequency $\omega$, and let $\theta \in \vBrace{-\frac{\pi}{2},\frac{\pi}{2}}$ denote the direction of arrival (DOA) angle measured with respect to the broadside of the linear array, as shown in Fig.~\ref{fig_ULA_imping}.
The wave signals spatially sampled by the microphone array inputs are given by:
\begin{equation}
\label{eq_prlm_timeSpace_outputVec}
\vecnot{f}\rBrace{t,x} = \vBrace{f_{\omega}\rBrace{t-\tau_{0}}, \dots, f_{\omega}\rBrace{t-\tau_{M-1}}}^{T}, \tau_{m} = \frac{\sin{\theta}\cdot{}x_{m}}{c},
\end{equation}
where $\tau_{m}$ is the propagation delay for the incoming signal and $c$ is the wave's velocity in the medium.
A value of $\tau_{m}=0, \forall{m}$ implies a DOA of $\theta = 0$, i.e. a plane wave parallel to the array, propagating perpendicularly to the array.
Let $\kappa=\frac{\omega}{c}\sin{\theta}=\frac{2\pi}{\lambda}\sin{\theta}$ denote the wavenumber for plane waves in a locally homogeneous medium, where λ denotes the wavelength corresponding to the angular frequency $\vecnot{v}\rBrace{\kappa}$ denote the \emph{array manifold vector} \cite{van2004optimum}, featuring all of the spatial characteristics of the microphone array.
Based on \eqref{eq_prlm_timeSpace_outputVec}, and the definition of κ above, the manifold vector can be expressed as
\begin{equation}
\vecnot{v}\rBrace{\kappa}=\vBrace{e^{-j\kappa{}x_{0}},\dots,e^{-j\kappa{}x_{M-1}}}^{T}.
\end{equation}
% \section{Sensor array performance measure}
% \label{sec:prlm_sensorArrayPerf}
% Numerous performance measures are utilized for evaluating the microphone array capabilities. 
Each of the measures aims to quantify a significant aspect of either the response of an array to the signal environment or of the sensitivity to an array design error.
First, we review the concept of beampatterns. 
Beampatterns are the main tool in assessing an array performance as they express the beamformer response for various frequencies and signal angles. 
Let $P\rBrace{\omega,\theta}$ denote the frequency and DOA dependent beamformer response. It is given by:
\begin{equation}
P\rBrace{\omega,\theta}=\sum_{m=0}^{M-1}e^{-j\omega\tau_{m}w_{m}^{*}}=\vecnot{w}^{H}\vecnot{v}\rBrace{\omega,\theta}    
\end{equation}
where $\vecnot{w}=\vBrace{w_{0},\dots,w_{M-1}}^{T}$ and $^{H}$ denotes Hermitian transpose operation, and let $D\rBrace{\omega,\theta}$ denote the beampattern of a given beamformer. 
It is expressed by:
\begin{equation}
D\rBrace{\omega,\theta}=20\log_{10}\abs{P\rBrace{\omega,\theta}},
\end{equation}
where $\abs{\cdot}$ denotes the absolute value. 
For a given angular frequency $\omega$, the beampattern $D\rBrace{\omega,\theta}$ is a function of the angle $\theta$ and the beamwidth is measured in terms of $\theta$.
In this work, the beamwidth is measured between the two lowest values at both sides of the main lobe.
\par
A major challenge in practical beamformer applications is the potential sensitivity to mismatches between the actual array attributes and the model used to derive the desired beamformer. 
In practical applications, mismatches can occur either by array location perturbations, production faults or filter perturbations. 
The sensitivity function often used as a criterion for assessing the affect of mismatches on the array response is defined in \cite{van2004optimum} by:
\begin{equation}
T_{se}=A^{-1}_{w}=\abs{\abs{\vecnot{w}}}^{2},
\end{equation}
where $A_{w}^{-1}$ is the inverse expression of the white noise gain given by $A_{w}=\text{SNR}_{out}\rBrace{k}/\text{SNR}_{in}\rBrace{k}$ and $\vecnot{w}$ is the weight vector corresponding to all of the FIR filter channels in the $k$-th frequency bin. Therefore, as the white noise gain increases, the sensitivity decreases and the array would be more robust to mismatch.
\section{Fisher Information Matrix}
\label{sec:prlm_FIM}
In the parameter estimation problems, we obtain information about the parameter from asample of data coming from the underlying probability distribution. 
A natural question is:how much information can a sample of data provide about the unknown parameter? 
This section introduces such a measure for information, and we can also see that this informationmeasure can be used to find bounds on the variance of estimators, and it can be used toapproximate the sampling distribution of an estimator obtained from a large sample, andfurther be used to obtain an approximate confidence interval in case of large sample.
In this section, we consider a random variable $X$ for which the pdf or pmf is $f\rBrace{x|\theta}$, wehre $\theta$ is an unknown parameter and $\theta\in\Theta$, with $\Theta$ is the parameter space.
Intuitively, if an event has small probability, then the occurrence of this eventbrings us much information. 
For a random variable $X \sim f\rBrace{x|\theta}$, if $\theta$ were the true value ofthe parameter, the likelihood function should take a big value, or equivalently, the derivativelog-likelihood function should be close to zero, and this is the basic principle of maximumlikelihood estimation.
We define $l\rBrace{x|\theta} = \log{f\rBrace{x|\theta}}$ as the log-likelihood function, and
\begin{equation*}
    l^{\prime}\rBrace{x|\theta} = \frac{\partial}{\partial\theta}\log{f\rBrace{x|\theta}}=\frac{f^{\prime}\rBrace{x|\theta}}{f\rBrace{x|\theta}}
\end{equation*}
where $f^{\prime}\rBrace{x|\theta}$ is the derivative of $f\rBrace{x|\theta}$ with respect to $\theta$.
Similarly, we denote the second order derivative of $f\rBrace{x|\theta}$ with respect to $\theta$ as $f^{\prime\prime}\rBrace{x|\theta}$.
According to the above analysis, if $l^{\prime}\rBrace{X|\theta}$ is close to zero, then it is expected, thus the random variable does not provide much information about $\theta$;
on the other hand, if $\abs{l^{\prime}\rBrace{X|\theta}}$ or $\abs{l^{\prime}\rBrace{X|\theta}}^{2}$ is large, the random variable provides much information about $\theta$.
Thus, we can use $\abs{l^{\prime}\rBrace{X|\theta}}^{2}$ to measure the amount of information provided by $X$.
however, since $X$ is a random variable, we should consider the average case. Thus, we introduce the following definition:
\begin{definition}
\emph{Fisher information} (for $\theta$) contained in the random variable $X$ is defined as:
\begin{equation}
\label{eq_prlm_FI_def}
    I\rBrace{\theta}=E_{\theta}\cBrace{\abs{l^{\prime}\rBrace{X|\theta}}^{2}}=\int{}\abs{l^{\prime}\rBrace{X|\theta}}^{2}f\rBrace{x|\theta}dx.
\end{equation}
\end{definition}
We assume that we can exchange the order of differentiation and integration, then
\begin{equation*}
    \int{}f^{\prime}\rBrace{x|\theta}dx=\frac{\partial}{\partial\theta}\int{}f\rBrace{x|\theta}dx=0.
\end{equation*}
Similarly,
\begin{equation*}
    \int{}f^{\prime\prime}\rBrace{x|\theta}dx=\frac{\partial^{2}}{\partial\theta^{2}}\int{}f\rBrace{x|\theta}dx=0.
\end{equation*}
It is easy to see that
\begin{equation*}
    E_{\theta}\cBrace{l^{\prime}\rBrace{X|\theta}}=\int{}l^{\prime}\rBrace{x|\theta}f\rBrace{x|\theta}dx=\int{}\frac{f^{\prime}\rBrace{x|\theta}}{f\rBrace{x|\theta}}f\rBrace{x|\theta}dx=\int{}f^{\prime}\rBrace{x|\theta}=0
\end{equation*}
Therefore, the definition of Fisher information \eqref{eq_prlm_FI_def} can be rewritten as
\begin{equation}
\label{eq_prlm_IeqVar}
    I\rBrace{\theta}=\text{Var}_{\theta}\cBrace{l^{\prime}\rBrace{X|\theta}}
\end{equation}.
Also, notice that
\begin{equation*}
    l^{\prime\prime}\rBrace{x|\theta}=\frac{\partial}{\partial\theta}\vBrace{\frac{f^{\prime}\rBrace{x|\theta}}{f\rBrace{x|\theta}}}=\frac{f^{\prime\prime}\rBrace{x|\theta}f\rBrace{x|\theta}-\abs{f^{\prime}\rBrace{x|\theta}}^{2}}{\abs{f\rBrace{x|\theta}}^{2}}=\frac{\abs{f^{\prime\prime}\rBrace{x|\theta}}^{2}}{\abs{f\rBrace{x|\theta}}^{2}}-\abs{l^{\prime}\rBrace{x|\theta}}^{2}.
\end{equation*}
Therefore,
\begin{equation*}
    E_{\theta}\cBrace{l^{\prime\prime}\rBrace{x|\theta}}=\int{\vBrace{\frac{\abs{f^{\prime\prime}\rBrace{x|\theta}}^{2}}{\abs{f\rBrace{x|\theta}}^{2}}-\abs{l^{\prime}\rBrace{x|\theta}}^{2}}f\rBrace{x|\theta}}=\int{f^{\prime\prime}\rBrace{x|\theta}dx}-E_{\theta}\cBrace{l^{\prime}\rBrace{X|\theta}}=-I\rBrace{\theta}.
\end{equation*}
Finally, we have another formula to calculate Fisher information:
\begin{equation}
\label{eq_prlm_Ifin}
    I\rBrace{\theta} = -E_{\theta}\cBrace{l^{\prime\prime}\rBrace{x|\theta}}=-\int{\vBrace{\frac{\partial^{2}}{\partial\theta^{2}}\log{f\rBrace{x|\theta}}f\rBrace{x|\theta}}dx}
\end{equation}.
To summarize, we have three methods to calculate Fisher information: equations \eqref{eq_prlm_FI_def}, \eqref{eq_prlm_IeqVar} and \eqref{eq_prlm_Ifin}.
In any problems, using \eqref{eq_prlm_Ifin} is the most convenient choice.
\subsection{Cram\'er-Rao Lower Bound and Asymptotic Distribution of Maximum Likelihood Estimators}
Suppose that we have a random sample $X_{1},\dots,X_{n}$ coming from a distribution for which the pdf or pmf is $f\rBrace{x|\theta},$ where the value of the parameter $\theta$ is unknown. 
We will show how to used Fisher information to determine the lower bound for the variance of an estimator of the parameter $\theta$.
\par
Let $\hat{\theta}=r\rBrace{X_{1},\dots,X_{n}}=r\rBrace{\vecnot{X}}$ be an arbitrary estimator of $\theta$.
Assume $E_{\theta}\rBrace{\hat{\theta}}=m\rBrace{\theta}$ and the variance of $\hat{\theta}$ is finite.
Let us consider the random variable $l^{\prime}\rBrace{\vecnot{X}|\theta}$, it can be shown that $E_{\theta}\cBrace{l^{\prime}\rBrace{\vecnot{X}|\theta}}=0$ and that
\begin{equation}
\label{eq_prlm_covThetaLTag}
    \text{Cov}_{\theta}\cBrace{\hat{\theta},l^{\prime}\rBrace{\vecnot{X}|\theta}}=m^{\prime}\rBrace{\theta}.
\end{equation}.
By Cauchy-Schwartz inequality and the definition of $I_{n}\rBrace{\theta},$
\begin{equation*}
    \vBrace{\text{Cov}_{\theta}\cBrace{\hat{\theta},l^{\prime}\rBrace{\vecnot{X}|\theta}}}^{2}\leq{}\text{Var}_{\theta}\cBrace{\hat{\theta}}\text{Var}_{\theta}\cBrace{l^{\prime}\rBrace{\vecnot{X}|\theta}}=\text{Var}_{\theta}\cBrace{\hat{\theta}}I_{n}\rBrace{\theta}
\end{equation*}
i.e.
\begin{equation*}
    \vBrace{m^{\prime}\rBrace{\theta}}^{2}\leq{}\text{Var}_{\theta}\cBrace{\hat{\theta}}I_{n}\rBrace{\theta}=nI\rBrace{\theta}=\text{Var}_{\theta}{\cBrace{\hat{\theta}}}
\end{equation*}
Finally, we get the lower bound of variance of an arbitrary estimator $\hat{\theta}$ as
\begin{equation}
\label{eq_prlm_CramerRao_lim}
    \text{Var}_{\theta}{\cBrace{\hat{\theta}}}\geq{}\frac{\vBrace{m^{\prime}\rBrace{\theta}}^{2}}{nI\rBrace{\theta}}
\end{equation}
The inequality \eqref{eq_prlm_CramerRao_lim} is called the \emph{information inequality}, and also known as the \emph{Cram\'er-Rao inequality} in honor of the Sweden statistician H. Cram\'er Indian statistician C. R. Rao who independently developed this inequality during the 1940s. 
The information inequality shows that as $I_{\theta}$ increases, the variance of the estimator decreases, therefore, the quality of the estimator increases, that is why the quantity is called “information”.
\par
If $\hat{\theta}$ is an unbiased estimator, then $m\rBrace{\theta}=E_{\theta}\rBrace{\hat{\theta}}=\theta$, $m^{\prime}\rBrace{\theta}=1$.
Hence, by the information inequality, for unbiased estimator $\hat{\theta}$,
\begin{equation*}
    \text{Var}_{\theta}\cBrace{\hat{\theta}}\geq\frac{1}{nI\rBrace{\theta}}.
\end{equation*}
The right hand side is always called the Cram\'er-Rao lower bound (CRLB): under certain conditions, no other unbiased estimator of the parameter $\theta$ based on an i.i.d sample of size $n$ can have a variance smaller than CRLB.
\subsection{The Multiple Parameter Case}
Suppose now there are more than one parameters in the distribution model, that is, the random variable $X\sim{}f\rBrace{x|\vecnot{\theta}}$ with $\vecnot{\theta}=\rBrace{\theta_{0},\dot,\theta_{k-1}}^{T}$.
We denote the log-likelihood function as
\begin{equation*}
    l\rBrace{\vecnot{\theta}}=\log{f\rBrace{x|\vecnot{\theta}}},
\end{equation*}
and its first order derivative with respect to $\vecnot{\theta}$ is a $k$-dimensional vector, which is
\begin{equation*}
    \frac{\partial{}l\rBrace{\vecnot{\theta}}}{\partial{}\vecnot{\theta}}=\rBrace{\frac{\partial{}l\rBrace{\vecnot{\theta}}}{\partial{}\theta_{0}},\dots,\frac{\partial{}l\rBrace{\vecnot{\theta}}}{\partial{}\theta_{k-1}}}^{T},
\end{equation*}
The second order derivative of $l\rBrace{\vecnot{\theta}}$ with respect to $\vecnot{\theta}$ is a $k\times{}k$ matrix, which is
\begin{equation*}
    \frac{\partial^{2}l\rBrace{\vecnot{\theta}}}{\partial{}\vecnot{\theta}^{2}}=\cBrace{\frac{\partial^{2}l\rBrace{\vecnot{\theta}}}{\partial{}\theta_{i}\partial{}\theta_{j}}}_{i,j\in\vBrace{0,\dots,k-1}}.
\end{equation*}
We define the \emph{Fisher information matrix} as
\begin{equation*}
    I\rBrace{\vecnot{\theta}}=E\cBrace{\frac{\partial{}l\rBrace{\vecnot{\theta}}}{\partial\vecnot{\theta}}\rBrace{\frac{\partial{}l\rBrace{\vecnot{\theta}}}{\partial\vecnot{\theta}}}^{T}}=\text{Cov}\cBrace{\frac{\partial{}l\rBrace{\vecnot{\theta}}}{\partial\vecnot{\theta}}}=-E\cBrace{\frac{\partial^{2}l\rBrace{\vecnot{\theta}}}{\partial{}\vecnot{\theta}^{2}}}.
\end{equation*}
Since the covariance matrix is symmetric and semi-positive definite, these properties hold for the Fisher information matrix as well.
\subsection{FIM applications}
Considering unbiased estimators, the CRLB for the multi-parameter case can be shown to be
\begin{equation}
\label{eq_prlm_multiVar_CRLB}
    \test{Cov}_{\vecnot{\theta}}\cBrace{\hat{\vecnot{\theta}}\rBrace{\vecnot{X}}}\geq{}I^{-1}\rBrace{\vecnot{\theta}}
\end{equation}
where the matrix inequality $A\geq{}B$ means that $A-B$ is positive semi-definite.
From \eqref{eq_prlm_multiVar_CRLB}, it is obvious that when $I$'th determinant is increased, the CRLB decreases which implicates that the measuring is more informative and though not necessarily obtainable, an optimal estimator will achieve higher accuracy in such scenario.