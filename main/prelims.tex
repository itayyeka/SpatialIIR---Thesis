\chapter{Preliminaries}
\label{chap:prelims}

As a preliminary discussion to the presentation of out contribution, we overview some the key theoretical basis of the thesis.
In Sec.~\ref{sec:prlm_propWaveField}, the wave propagation, being the most basic concept for array processing, is discussed.
It is then followed by a short overview of array-processing in Sec.~\ref{sec:prlm_sensorArrays}.
To accommodate the discussion for spatial filtering, we also mention the time-space relation of spatially sampled signals in Sec.~\ref{sec:prlm_timeSpaceSig}.
As an additional theoretical concept, arriving from the information theory, which will be used in the main chapter (Ch.~\ref{chap:firstchap}), is the Fisher Information Matrix, covered in Sec.~\ref{sec:prlm_FIM}.
\par 
For an easier reading treat each section as independent, where notations and symbols are each defined in the specific section and are not mutual, nor used in the main chapter. 

% \begin{definition}
% The \emph{von Neumann model} of a computer, also known as the \emph{Princeton architecture} is an architecture for digital computers, which consists of a processing units, containing an ALU and processing registers; a control unit consisting of an instruction register and a program counter; a memory unit which stores both data and instructions; and input-and-output mechanisms.
% \end{definition}

% \section{Propagating Wave Fields}
% \label{sec:prlm_propWaveField}
% % In array signal processing, propagating waves carry signals from the source to the array.
% Therefore, these signals are represented in the time-space domain. The space domain is represented by either the three dimensional Cartesian coordinates $\rBrace{x,y,z}$ or the three dimensional spherical coordinates $\rBrace{r,\phi,\theta}$, where $0 \leq \phi \leq 2\pi$, $0 \leq \theta \leq \pi$ are the azimuth and elevation angles, respectively. The time domain is represented by $t$.
An elementary physical phenomenon is the spatial dynamics of waves, referred to as wave-propagation.
The spatial location is noted by the Cartesian coordinates $\rBrace{x,y,z}$ or the three dimensional spherical coordinates $\rBrace{r,\phi,\theta}$, where $0 \leq \phi \leq 2\pi$, $0 \leq \theta \leq \pi$ are the azimuth and elevation angles, respectively. 
% Finally, let $f\rBrace{t,\vecnot{r}}$ denote the time-space representation of the input signal, where $\vecnot{r}$ is the radius-vector in the three-dimensional system. 
% The relations between the Cartesian and spherical coordinates are given in Fig.~\ref{fig_coordinates}.
Denoting $t$ as the time, the time-space representation of the a signal is $f\rBrace{t,\vecnot{r}}$ or , where the relations between the Cartesian and spherical coordinates are given in Fig.~\ref{fig_coordinates}.
% $f\rBrace{t,x,y,z}$ describes the signal impinging the microphone array. 
In a homogeneous, dispersion free and lossless medium the wave equation is:
\begin{equation}
\label{eq_prlm_waveEq}
\nabla^{2}f\rBrace{t,x,y,z}=\frac{1}{c^2}\frac{\partial^{2}f\rBrace{t,x,y,z}}{\partial{t^{2}}}
\end{equation}
where $\nabla^{2}$ is the Laplacian operator and $c$ represents the wave's velocity in the medium.
\begin{figure}[h!]
    \begin{center}
        \begin{overpic}[width=0.5\linewidth, 
        %grid, 
        tics=10,trim=0 0 0 0]{./Media/fig_coordinates.png}
            % \put (50, 62.5) {\footnotesize{$r=0$}}
        \end{overpic}
    \end{center}
     \caption{A three dimensional coordinate system with Cartesian and spherical coordinates.}
    \label{fig_coordinates}
\end{figure}
A possible solution to \eqref{eq_prlm_waveEq}, $f_p\rBrace{t,x,y,z}$, may be of a complex exponential form:
\begin{equation}
\label{eq_prlm_waveEq_pSol}
f_p\rBrace{t,x,y,z} = A\exp{j\rBrace{\omega{t}-k_{x}x-k_{y}y-k_{z}z}}
\end{equation}
Where $A$ is a complex constant, $\omega$ denotes a constant temporal frequency and $k_{x}x,k_{y},k_{z}$ are real constants. 
Plugging \eqref{eq_prlm_waveEq_pSol} into \eqref{eq_prlm_waveEq} results in the \emph{monochromatic plane wave}:
\begin{equation}
\label{eq_prlm_waveEq_subs}
k_{x}^{2}-k_{y}^{2}-k_{z}^{2} = \frac{\omega^{2}}{c^{2}}.
\end{equation}
Using the plane wave notation emphasizes the fact that for a given time, $t_{0}$, all points on a plane given by $k_{x}x+k_{y}y+k_{z}z = constant$ are with the same wave value and the vector notation of \eqref{eq_prlm_waveEq}'s solution is
\begin{equation}
\label{eq_prlm_waveEq_finSol}
f\rBrace{t,\vecnot{x}}=A\exp{j\rBrace{\omega{t}-\vecnot{k}\vecnot{x}}}.
\end{equation}
where $\vecnot{k}\vecnot{x}=constant$ are planes of constant wave value.
The wave propagation can be described as the traveling of the planes, stating that small steps of both space $\delta{\vecnot{x}}$ and time $\delta{t}$ result in the same wave value i.e. $f\rBrace{t+\delta{t},\vecnot{x}+\delta{\vecnot{x}}} = f\rBrace{t,\vecnot{x}}$, which yields
\begin{equation}
\omega\delta{t}-\vecnot{k}\delta\vecnot{x}=0.
\end{equation}
Assuming $\delta\vecnot{x}$ and $\vecnot{k}$ have the same direction, $\vecnot{k}\delta\vecnot{x} = \abs{\vecnot{k}}\abs{\delta\vecnot{x}}$ and $\frac{\delta{\vecnot{x}}}{\delta{t}} = \frac{\omega}{\abs{\vecnot{k}}}$, where $\frac{\delta{\vecnot{x}}}{\delta{t}}$ can designate the propagation speed of the plane wave. 
Since $\vecnot{k}$ and $\omega$ are related by $\abs{\vecnot{k}}^{2}=\frac{\omega^{2}}{c^{2}}$, we have
\begin{equation}
\label{eq_prlm_waveEq_stepsEq}
\frac{\delta{\vecnot{x}}}{\delta{t}}=c,
\end{equation}
where $c>0$ is the wave velocity in the medium.
The \emph{wavelength} ($\lambda$) denote the distance the plane wave propagated during a single temporal period of $T=\frac{2\pi}{\omega}$.
Let $\vecnot{k}$ denote the \emph{wavelength vector}. 
Its magnitude $\abs{\vecnot{k}}$ expresses the number of cycles in radians per meter of length that the plane wave has exhibited in the propagation direction.
Using \eqref{eq_prlm_waveEq_stepsEq} with $\delta{t} = \frac{2\pi}{\omega}$, we obtain:
\begin{equation}
\lambda=\delta{\vecnot{x}}=\frac{2\pi}{\abs{\vecnot{k}}}.
\end{equation}
Therefore, the wavenumber vector can be considered to represent spatial frequency, similarly to the manner $\omega$ represents temporal frequency.
% \section{Digital Filter Design - Finite/Infinite Response Filters}
% \label{sec:prlm_FILTERS}
% A basic building block of any signal processing system is the digital filter, which manipulates input signals according to some predetermined specifications. 
Nominally, digital filters are applied to temporal-domain sampled signals, while designed to meet certain frequency-domain specification.
The most  
\section{Sensor arrays}
\label{sec:prlm_sensorArrays}
Sensor arrays are sets of independently positioned sensors, where each sensor is sampling temporal snapshots of impinging signals.
The samples from the entire array are then fused together in order to extract the underlying data.
The spatial diversity of the sampled data allows the extraction of spatial features.
Sensor arrays are used in numerous applications, ranging from source localization, communication, medical applications, astronomy etc.
\par 
When designing a sensor array, multiple consideration are to be taken into account, ranging from physical characteristics e.g. area, weight and carrying platform, through performance related considerations such as accuracy, spatial selectivity, signal to noise ratio and even the financial aspect.
The most fundamental array specification is its geometry, for it dictates the spatial relations between simultaneous samples measured by the array's sensors, which also greatly influences the processing methods of the raw-data.
A basic example of a sensor array is the ULA (see Fig.~\ref{fig_ULA}), where its elements are uniformly spaced on a straight line with distance $d$ between each pair of sensors.
Considering an array of $N$ elements, the sensors are positioned at
\begin{equation}
p_{n}=nd,\ n=0,\dots,N-1
\end{equation}
where $n$ is the sensors index; $p_{n}$ denotes the position of the $n$'th sensor, $p_{0} = 0$ corresponds to the left-hand side of the array as illustrated in Fig.~\ref{fig_ULA}.
\begin{figure}[h!]
    \begin{center}
        \begin{overpic}[width=0.5\linewidth, 
        %grid, 
        tics=10,trim=0 0 0 0]{./Media/arrayBasic.png}
        \put (4, 4.5) {\tiny{$N$}}
        \put (86, 4.5) {\tiny{$N-1$}}
        \put (83.25, 28.5) {\tiny{$\rBrace{N-1}\cdot{}d$}}
        \end{overpic}
    \end{center}
     \caption{A uniform linear array of size $N$ and spacing $d$.}
    \label{fig_ULA}
\end{figure}
\par To properly formulate the discussed additional spatial information when using sensor arrays, a quick overview on wave propagation is due and presented in the following section (Sec.~\ref{sec:prlm_propWaveField}).

\section{Time-Space Signals}
\label{sec:prlm_timeSpaceSig}
As mentioned before, the measurements of the sensor array's elements are fused together, where spatial information resides in the relations between samples taken the same time $t$.
A basic application of the sensor array is the beamformer, designed to enhance signals from certain directions while suppressing other signals from unwanted directions, forming a beam directed to the enhanced direction.  
\begin{figure}[ht!]
    \begin{center}
        \begin{overpic}[width=0.6\linewidth, 
        %grid, 
        tics=10,trim=0 0 0 0]{./Media/arrayProc.png}
        \put(20,65){\rotatebox{-50}{\tiny{Impinging wavefront}}}
        \put(49,33.25){\rotatebox{0}{$\theta$}}
        \put(15,25){\rotatebox{0}{\tiny{0}}}
        \put(28,25){\rotatebox{0}{\tiny{1}}}
        \put(41,25){\rotatebox{0}{\tiny{2}}}
        \put(80,25){\rotatebox{0}{\tiny{$N-1$}}}
        \put(16,20){\rotatebox{0}{\tiny{$\omega_{0}$}}}
        \put(29.5,20){\rotatebox{0}{\tiny{$\omega_{1}$}}}
        \put(42,20){\rotatebox{0}{\tiny{$\omega_{2}$}}}
        \put(81,20){\rotatebox{0}{\tiny{$\omega_{N-1}$}}}
        \put(50.5,3.5){\rotatebox{0}{\tiny{$z\rBrace{t}$}}}
        \end{overpic}
    \end{center}
     \caption{ULA based coherent processing of simultaneous spatially sampled signal, measuring a wavefront impinging from DOA $\theta$}
    \label{fig_ULA_imping}
\end{figure}
In complimentary to Fig.~\ref{fig_ULA_sketch}, Fig.~\ref{fig_ULA_imping} adds the processor layer of the beamformer.
\par A temporal snapshot measured by the array at time $t$ is
\begin{equation}
\vecnot{f}\rBrace{t,\vecnot{p}} = \vBrace{f\rBrace{t,p_{0}},\dots,f\rBrace{t,p_{N-1}}}^{T},
\end{equation}
where $p_{n}$ denotes the $n$-th sensor position, $t$ denotes the continuous-time variable and $^{T}$ denotes the transpose operation. 
Considering the \emph{conventional beamformer} \cite{van2004optimum}, the output ($z\rBrace{t}$) is expressed by:
\begin{equation}
z\rBrace{t} = \sum_{n=0}^{N-1}{f\rBrace{t,p_{n}}\omega_{n}^{*}},
\end{equation}
where $\omega_{n}$ is a complex weight used for the $n$'th sensor's output, as shown in Fig.~\ref{fig_ULA_imping} and $^{*}$ denotes complex conjugation. 
Consider $s\rBrace{t} = e^{j\omega{t}}$ as a plane wave propagating at temporal frequency $\omega$, and $\theta \in \vBrace{-\frac{\pi}{2},\frac{\pi}{2}}$ as the direction of arrival (DOA) angle measured with respect to the broadside of the linear array, as shown in Fig.~\ref{fig_ULA_imping}.
The wave signal, spatially sampled by the sensor array inputs is
\begin{equation}
\label{eq_prlm_timeSpace_outputVec}
\vecnot{f}\rBrace{t,x} = \vBrace{s\rBrace{t},s\rBrace{t-\tau}\dots, s\rBrace{t-\rBrace{N-1}\tau}}^{T},
\end{equation}
where $\tau = \frac{d\cdot{}\cos{\theta}}{c}$ is the propagation delay between consecutive sensors and $c$ is the wave's velocity in the medium.
Considering narrowband signals, the time delay $\tau$ translates to a mere phase shift and the \emph{steering vector} takes the form of
\begin{equation}
\vecnot{d}\rBrace{\theta}=\vBrace{1,e^{-j\omega\tau},\dots,e^{-j\omega\tau\rBrace{N-1}}}^{T}.
\end{equation}
As every impinging signal will cause the excitation of the array elements in the form of a specific steering vector (up to a common additive phase), we define the \emph{array manifold} to be the set of all steering vectors of $\theta\in\left[0,2\pi\right)$, hence it also spans the received signal sub space in the noiseless scenario.
% \section{Localization}
% \label{sec:prlm_localization}
% In the wide and very active research field of parametric estimation, one issue of relevance to this thesis is localization.
\par 
In this work, we focus on localization of signal sources in the far-field case, where free space signal propagation is assumed.
Following the classical methods overview in \cite{krim1996two}, localization estimators may be divided to two main groups - \textit{spectral-based} and \textit{parametric}.
The spectral based estimators steer the array (mechanically or by beamforming) to a set of DOAs, searching for peaks in the received energy, where each peak is treated as a resolved emitter when greater than a certain threshold.
This approach of spatial filtering, is considered to suffer from fundamental
limitations, namely ``its performance, is directly dependent upon the physical size of the array (the aperture), regardless of the available data collection time and signal-to-noise ratio`` \cite{krim1996two}.
Although trials were made to increase the resolution of an array given a certain aperture, this limitation remained intact.
In this contribution we tackle this issue as will be shown in Chapter.~\ref{chap:firstchap} by introducing a spatial feedback.
\par
The later approach, of parameter estimation, assumes a physical model of the experimental scenario.
% Then, spatio-temporal estimators take advantage of not only the spatial diversity but also from the temporal information in order to increase the overall SNR.
Such methods, not only enabled the increase of spatial filtering resolution, but also allowed the implementation of data processing algorithms, which are more sophisticated than a mere search with a steered array.
In the following, a basic overview of some classical algorithms is presented.
\par
The earliest localization algorithms, which date back to world war II, are beamforming based, e.g. the conventional Bartlet \cite{van2004optimum} and Capon's \cite{capon1969high} beamformers.
Early examples of parametric estimators, are the Maximum-Likelihood \cite{macdonald1969optimum,schweppe1968sensor} and Maximum-Entropy \cite{ables1974maximum} which assume a known PDF to the received signal and estimate the desired parameters according to sampled data and fitting the closest matching PDF.
Until the mid- 1970's, direction finding techniques required knowledge of the array directional sensitivity pattern in analytical form, and the task of the antenna designer was to build an array of antennas with a prespecified sensitivity pattern.
\par
Trying to relax the need for such accuracy, also serving as the origin of the subspace based approach, was the Multiple SIgnal Classification (MUSIC) \cite{schmidt1986multiple} algorithm.
MUSIC essentially relieved the designer from designing accurate radiation patterns by introducing the concept of array calibration.
Although MUSIC did not mitigate the computational complexity of solution to the DOA estimation problem, it did extend the applicability of high-resolution DOA estimation to arbitrary arrays of sensors.
An important breakthrough, was the introduction of the orthogonal array manifold noise spaces, thus allowing the use of orthogonal projection in order to mitigate the noise effects.
\par
Another important parametric approach related algorithm, is the ESPRIT (Estimation of Signal Parameters via Rotational Invariance Techniques) \cite{ESPRIT}, serving as a milestone in the path to state-of-the-art algorithms.
In addition to using the rotational invariance of the signal subspace eigenvectors, it also reduced computation and storage costs (especially in the multi-dimensional estimation case) by replacing the covariance matrix calculation and eigendecomposition by a relaxed partial singular value decomposition (SVD) which is employed on the data itself without squaring it - thus mitigating numerical problems associated with ill-conditioned matrices.
\par
Following state-of-the-art developments \cite{tuncer2009classical}, the subspace based algorithms are still modified and adjusted to specific scenarios as in \cite{LpNorm_MUSIC} but also new and interesting approaches emerged such as
\begin{itemize}
    \item High Order Statistics (HOS), which extracts more information from the samples' higher order moments, is a very active field of research due to some fundamental issues which are inherently resolved - i.e. the Gaussian noise vanishes in the 4th order statistics and the ability to resolve more DOAs than array elements \cite{chevalier2006high}.
    This approach, being costly in computation effort, became popular probably due to the recently available low-cost powerful computation platforms.
    \item Inspired by the vast research field of sparse representations, some algorithms \cite{nadiri2014localization} use $L^{p}$ norms (where $p<2$), in order to improve estimation resolution under high reverberation acoustic scenarios.
    \item A very wide and active research field is the concept of cooperative localization related to mobile networks is growing at a very high rate due to the never-ending need for high-bandwidth and low-power communication of the mobile networks.
    \item Naturally, also many trials of harnessing the promising concept of neural networks are being done, see for example \cite{shareef2008localization}.
\end{itemize}
In this work, we actually revisit the most basic approach - i.e. beamforming which resides under the spectral based algorithms.
% \section{Sensor array performance measure}
% \label{sec:prlm_sensorArrayPerf}
% Numerous performance measures are utilized for evaluating the microphone array capabilities. 
Each of the measures aims to quantify a significant aspect of either the response of an array to the signal environment or of the sensitivity to an array design error.
First, we review the concept of beampatterns. 
Beampatterns are the main tool in assessing an array performance as they express the beamformer response for various frequencies and signal angles. 
Let $P\rBrace{\omega,\theta}$ denote the frequency and DOA dependent beamformer response. It is given by:
\begin{equation}
P\rBrace{\omega,\theta}=\sum_{m=0}^{M-1}e^{-j\omega\tau_{m}w_{m}^{*}}=\vecnot{w}^{H}\vecnot{v}\rBrace{\omega,\theta}    
\end{equation}
where $\vecnot{w}=\vBrace{w_{0},\dots,w_{M-1}}^{T}$ and $^{H}$ denotes Hermitian transpose operation, and let $D\rBrace{\omega,\theta}$ denote the beampattern of a given beamformer. 
It is expressed by:
\begin{equation}
D\rBrace{\omega,\theta}=20\log_{10}\abs{P\rBrace{\omega,\theta}},
\end{equation}
where $\abs{\cdot}$ denotes the absolute value. 
For a given angular frequency $\omega$, the beampattern $D\rBrace{\omega,\theta}$ is a function of the angle $\theta$ and the beamwidth is measured in terms of $\theta$.
In this work, the beamwidth is measured between the two lowest values at both sides of the main lobe.
\par
A major challenge in practical beamformer applications is the potential sensitivity to mismatches between the actual array attributes and the model used to derive the desired beamformer. 
In practical applications, mismatches can occur either by array location perturbations, production faults or filter perturbations. 
The sensitivity function often used as a criterion for assessing the affect of mismatches on the array response is defined in \cite{van2004optimum} by:
\begin{equation}
T_{se}=A^{-1}_{w}=\abs{\abs{\vecnot{w}}}^{2},
\end{equation}
where $A_{w}^{-1}$ is the inverse expression of the white noise gain given by $A_{w}=\text{SNR}_{out}\rBrace{k}/\text{SNR}_{in}\rBrace{k}$ and $\vecnot{w}$ is the weight vector corresponding to all of the FIR filter channels in the $k$-th frequency bin. Therefore, as the white noise gain increases, the sensitivity decreases and the array would be more robust to mismatch.