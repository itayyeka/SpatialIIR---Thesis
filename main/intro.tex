\chapter{Introduction}
\label{chap:intro}
\section{Background and Motivation}
The general field of array processing has been thoroughly studied throughout several decades.
The array sensors' spatial diversity enables the extraction of spatial information about impinging signals, thus laying the ground for wide range of applications, such as localizing a transmitting source \cite{skolnik2008radar,long2019}, blindly separating mixtures of impinging signals \cite{comon1994independent}, improving SNR \cite{verdu1998multiuser}, and many more. 
\par The ULA has always been a point of interest, due to its simplicity of analysis \cite{van2004optimum,benesty2018}. 
The array size and the number of its elements ($N$) have significant influence on the obtained array performance, such as SNR improvement, spatial separation capabilities and its spatial response's degrees of freedom (DOF).
\par Early algorithms for direction of arrival (DOA) estimation were based on the beamforming concept \cite{krim1996two}, where the array's reception beam is steered (mechanically or electronically) to multiple directions in search of received energy peaks which are interpreted as valid emitters.
One notable improvement to the convectional beamformer \cite{van2004optimum} was Capon's beamformer  (aka minimum variance distortionless response (MVDR) in acoustics related literature) which attempts to minimize the power contributed by noise and any signals coming from other directions than the DOA of interest, while maintaining a fixed gain
in the desired direction.
\par The most practically used approach in DOA estimators, referred as the subspace based estimation, involves invariant subspace analysis of observed covariance matrices.
Early research includes PCA and errors-in-variables time series analysis \cite{krim1996two}.
However, the tremendous interest in the subspace approach is mainly due to the introduction of the
MUSIC (Multiple SIgnal Classification) algorithm \cite{schmidt1986multiple} which decomposes the estimated covariance matrix to its signal and noise eigenvectors, using orthogonal projection exploiting the fact that the noise sub space is orthogonal to the signal subspace.
It is interesting to note that while earlier works were mostly derived in the context of time series analysis and only later applied to the sensor array problems, MUSIC was originally presented as a DOA estimator.
\par A less practical, but optimal in the sense of root mean square error (RMSE) \cite{krim1996two}, is the maximum likelihood (ML) approach which tries to find the most probable DOA assuming a known PDF.
Though this approach served as a basis for extensive research, leading to more advanced concepts such as the independent component analysis (ICA) \cite{hyvarinen1999survey}, in the context of practical DOA estimation it's computational effort is high.
Adding the fact that sub-optimal methods such as MUSIC were proven to achieve the Cram\'er Rao bound (CRB) under some practical assumptions (e.g. non-coherent signals) \cite{stoica1989music}, these sub optimal estimators are substantially more common in practice. 
\par In pursuit of spatial performance improvement, namely higher spatial separation and selectivity of arriving signals, many approaches were suggested.  
One approach, commonly referenced as ``virtual arrays'' \cite{pal2010nested,chevalier2005virtual,dogan1995applications} deals with the extraction of samples originated in sensors that do not really exist, by using high (higher than 2) order statistics and manipulating multiple statistical cross-terms in order to estimate statistical characteristics of signals impinging in missing sensors.
\par Using a similar approach, the $2q$-MUSIC algorithm \cite{chevalier2006high}, enables the use of $N^{2q}$ ``virtual elements'', by calculating the $q$'th order statistics.
Another approach, involving different array geometries, examined minimum redundancy arrays \cite{moffet1968minimum,pillai1985new,pillai1987statistical,Kupershtein2013}, aiming to reduce the spatial ambiguity. The basic concept was minimization of the inter-element spacing redundancy in order to increase the overall resolution.
Although the $q$-th order statistics based approach promises substantial improvement, both noise sensitivity and impractical computation costs are dominant drawbacks, limiting the usage of such methods in practical applications.   
\par Adaptive processing schemes \cite{frost1972algorithm,manolakis2000statistical}, being a wide and active research area, were also suggested trying to adaptively estimate and suppress the noise component in impinging signals by minimization of the receiver's output energy with some constraints.
In \cite{manolakis2000statistical}, two main approaches to adaptive processing are discussed. The block adaptive scheme (which is also called \textit{sample matrix inversion}) and the sample-by-sample method.
In both methods, after a certain amount of time, the system refreshes the spatial filter coefficients, trying to better suppress the received noise.
As in many adaptive processing applications, the two methods rely on steepest descend optimization, which exposes the adaptive processing approach to errors related to optimization parameters choices such step size etc.
\par In \cite{van1988beamforming}, ULA based beamforming and temporal FIR filtering are shown to be mathematically analogous, where the DOA acts as the spatial version of temporal frequency.
Inspired by this analogy, which is thoroughly discussed in Sec.~\ref{sec:prlm_FIR_IIR}, we were motivated to find the spatial counterpart of the temporal IIR filter, both from academic curiousness and due to the known advantages of the IIR filter's performance.
Therefore we formulated a question, ``what are the equivalent spatial domain processing methods which will be analogous to temporal IIR filtering?'' which served as a guide throughout our research.
\par Achieving spatial IIR response has also motivated other works.
In \cite{wen2013extending} two methods were considered.
The first one (see Fig.~\ref{fig_intro_wen1}) was to estimate the time of arrival (TOA) difference between two consecutive sensors and to synthetically generate the recursive part of the IIR filter, entirely in the time-domain.
\begin{figure}[ht!]
    \begin{center}
        \begin{overpic}[width=0.5\linewidth, 
        %grid, 
        tics=10,trim=0 0 0 0]{./Media/WEN_method1.png}
            % \put (50, 62.5) {\footnotesize{$r=0$}}
        \end{overpic}
    \end{center}
     \caption{Synthetic delay emulated for the recursive part of the spatial IIR filter.}
    \label{fig_intro_wen1}
\end{figure}
The second approach suggested to consider overlapping subsets of one large ULA as finite approximation to an infinite array (see Fig.~\ref{fig_intro_wen2}).
\begin{figure}[ht!]
    \begin{center}
        \begin{overpic}[width=0.5\linewidth, 
        %grid, 
        tics=10,trim=0 0 0 0]{./Media/WEN_method2.png}
            % \put (50, 62.5) {\footnotesize{$r=0$}}
        \end{overpic}
    \end{center}
     \caption{In \cite{wen2013extending} subsections on a single large ULA are used to approximate an IIR response}
    \label{fig_intro_wen2}
\end{figure}
Obviously, the former approach is very sensitive to the synthetic delay accuracy and involves temporal domain processing (which we try to avoid in this work in order to achieve purely spatial processing).  
Also, close inspection of the latter approach reveals that even though the sub-arrays are each weighted separately, each sensor's output is eventually linearly weighted when taking into account all related sub-arrays.
Clearly, this is just another possible implementation of basic FIR like ULA spatial processor, so in both cases, there is no true spatial feedback between the array and the source of interest.
\par Other interesting works \cite{Hum2009BeamformingFilters,madanayake2008speed} in the field of ultra-wideband (UWB) signals treat the spatial and temporal diversity, which reside in multiple consecutive snapshots of the array output, as if the signal is sampled in two independent temporal frequencies.
Inspired by \cite{bruton1985three}, the authors show that impinging plane waves are represented as straight lines tilted according to their DOA and suggest methods to design line filters in that spatio-temporal plane.
In \cite{Hum2009BeamformingFilters}, the authors also address the fact that ideal straight line filter is mathematically unachievable due to spatio-temporal frequencies periodicity.
This issue manifests as a ``bend`` close to the sampling frequencies (see Fig.~\ref{fig_intro_UWB_bend}).
\begin{figure}[ht!]
    \begin{center}
        \begin{overpic}[width=0.5\linewidth, 
        %grid, 
        tics=10,trim=0 0 0 0]{./Media/UWB_bend.png}
            % \put (50, 62.5) {\footnotesize{$r=0$}}
        \end{overpic}
    \end{center}
     \caption{2D spatio-temporal frequency response (dB) of practical UWB filter. The black line is angled according to the desired DOA.}
    \label{fig_intro_UWB_bend}
\end{figure}


\section{Overview of the Thesis}
In this thesis, we propose a new feedback based beamformer architecture which acts as an actual spatial IIR filter while avoiding any temporal processing to the signal.
The reader may observe that the presented research stages lead one to the other as each analysis of the proposed methods uncovered issues that are addressed in the following.
\par
First, as in the early stages of the research, the analogy between a spatial (the ULA) and temporal (the FIR) processors, led us to search for the spatial counterpart of another temporal (the IIR) processor, aiming to translate its advantages from the temporal domain to the spatial domain.
Also, classic spatial analysis tools have been gathered to serve as objective tests to the proposed solutions.
\par
In the second part of the research we present the basic spatial architecture, called FB, introducing the concept of spatial feedback.
The array configuration is rigorously justified, e.g. the setting of the coefficients is done with the use of the Fisher Information Matrix (FIM).
Analysis of the spatial structure proves that the desired recursive spatial response is achievable and we show that the advantages of the proposed solution may be expressed with the classic analysis framework.
As we aim to present a practical architecture, estimation errors were taken into consideration and unveiled some sensitivities that should have been addressed.
\par
After close inspection, we conclude in the third part that the carrier frequencies cause this sensitivity and search for ways to mitigate it.
To this end, we present a more sophisticated design, based on the basic architecture, achieving a practical low sensitivity spatial processor which successfully extracts spatial information from two independent close frequency FB units.
We then simulate the proposed solution and find that in addition to its low sensitivity, it also outperforms the classic beamformers, such as the conventional beamformer (CB) in low SNR. 
% In this thesis, we propose a new feedback based beamformer architecture which acts as an actual spatial IIR filter while avoiding any temporal processing to the signal.
% We aim to guide the reader thorough the key stages of the research where in each stage, a sufficient background is provided followed by the solution composition.
% \par As this research does not closely follows any former work, some of analysis must be rigorously justified, hence we turn to classic literature as guidance.
% As an example, for the configuration of the array, we use the Fisher Information Matrix (FIM) aiming to maximize the residing spatial information in the system.
% \par After presenting the basic approach with it's complementary adjusted classic spatial analysis, we find that it is inherently sensitive to range estimation errors.
% We then suggest a more robust solution which is based on our conclusions from previous analysis.
% With the robust solution in hand, we present simulations and compare it to the basic conventional beamformer (CB) and the initial more sensitive approach.
% As expected, we find that it outperforms them, proving to perform well also in low SNR.
\section{Main Contributions}
\par In this contribution, we present a low-complexity sensor array processing approach which achieves the desired spatial domain exclusive IIR-like beampattern, while avoiding any temporal processing of the signal.
To this end, we arbitrarily choose to formulate the problem in the context of localization, hence our goal is to estimate the direction and the range of some target of interest. 
\par The novelty, compared to traditional array processing, is by incorporating spatial feedback, which we prove to be the spatial domain equivalent of temporal domain IIR filtering.
Assuming the target of interest has a mirror-like behaviour (i.e., reflects its impinging signals), the spatial feedback between the array and the target is created by continuously re-transmitting a synthesized version of the impinging signal (and its reflections) to the target.
Note that the initial stimulus can be generated by the target or the array itself. In the text to follow, we assume this is the latter. 
Furthermore, as opposed to the passive target case (i.e., a target which merely reflects the impinging signal), one may consider a cooperative target, which receives, enhances and re-transmits the signal back to the array.
\section{Thesis Organization}
\par This thesis is organized as follows.
In Chapter~\ref{chap:prelims}, we cover basic concepts of array processing, followed by some emphasize on the localization field of research and finalize with the basic concepts which drive the motivation for this research.
\par In Chapter.~\ref{chap:firstchap} the basic FB is presented and thoroughly analysed, followed by the presentation of the more robust architecture, which uses the former as a building block.
Simulations are also provided, supporting the analytical analysis.
\par In Chapter.~\ref{chap:conclusion} we discuss the results and conclude our findings. We finalize by also proposing some leads to future research.
\par In this work, we use the following conventions; vectors and matrices are denoted by lower and upper case bold italic letters respectively. 
$v_{i}$ and $A_{ij}$ are the $i$-th element of the vector $\vecnot{v}$ and the $i,j$-th element of the matrix $\vecnot{A}$.
Also, $\vecnot{A}^{T}$, $\vecnot{A}^{*}$ and $\vecnot{A}^{H}$ are the transpose, conjugate and Hermitian transpose of the matrix $\vecnot{A}$ respectively.