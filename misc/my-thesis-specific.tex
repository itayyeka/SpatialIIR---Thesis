% This file should contain your own definitions specific
% only to this thesis;

% What it contains below is used 
% simply for generating dummy text in the sample 
% content provided with the template (see mainchap1.tex);
% so you can safely delete this when creating your own
% thesis
%general:
%Box and color definitions:
%--------------------------
%General definitions:
%-------------------
\newcommand{\etal}{{\em {et al.}}}
% \newcommand{\B}[1]{\mathbf{#1}}
\newcommand{\df}{\triangleq}
\newcommand{\norm}[1]{\left\Vert#1\right\Vert}
\newcommand{\abs}[1]{\left\vert#1\right\vert}
\newcommand{\RE}{\operatorname{Re}}
\newcommand{\IM}{\operatorname{Im}}
\newcommand{\sgma}[3]{\sum\limits_{{#1}={#2}}^{#3}}
\newcommand{\Brace}[1]{\left\{{#1}\right\}} %Braces
\newcommand{\Brack}[1]{\left({#1}\right)} %Brackets
\newcommand{\sBrack}[1]{\left[{#1}\right]} %square Brackets

%\newcommand{\ip}[2]{{\langle{#1},{#2}\rangle}} %inner-product
\newcommand{\ipLW}[3]{{\langle{#1},{#2}\rangle}_{{#3}}} %weighted inner-product

\newcommand{\Tr}[1]{Tr\Brack{#1}}
\newcommand{\Mtr}[2] %short notation for 2x1 Matrix.
{\begin{bmatrix}
  #1 \\
  #2
\end{bmatrix}}
\newcommand{\cMtr}[2] %short notation for 2x1 Matrix with curves.
{\left(
\begin{array}{c}
    {#1} \\
    {#2} \\
\end{array}
\right)}
\newcommand{\Mtrs}[2] %short notation for 2x1 Matrix star (adjoint)
{\begin{bmatrix}
  #1 &
  #2
\end{bmatrix}}
\newcommand{\Mtrt}[3] %short notation for 3x1 Matrix.
{\begin{bmatrix}
  #1 \\
  #2 \\
  #3
\end{bmatrix}}

\newcommand{\Cases}[4]{
\left\{
\begin{tabular}{lcl}
    $#1$ & $=$ & $#2$\\
    $#3$     & $=$ & $#4$
\end{tabular}
\right. }

\newcommand{\und}{\underline} %How lazy can I get?
\newcommand{\ovr}{\overline}
\newcommand{\conj}[1]{{#1}^\ast} %Conjugation


\newcommand{\er}[1]{{(\ref{#1})}} %equation reference

\newtheorem{Lemma}{Lemma}{}
\newtheorem{Prop}{Proposition}{}
% \newtheorem{theorem}{Theorem}{}


\newenvironment{alg}[5]
{
\begin{figure}[htbp]
\begin{center}
\fbox{
  \begin{ColorBoxedminipage}{13cm}
%    \leftline{\color{Black}\bf {#1}}
    {#4}
   \end{ColorBoxedminipage}
   }
\end{center}
  \bcaptionff{#1}{#2}{}{#3}
  \label{#5}
\end{figure}
}{}

%Just body, caption and label.
\newenvironment{algo}[3]
{
\begin{figure}[htbp]
\begin{center}
\fbox{
  \begin{ColorBoxedminipage}{7.5cm}
%    \leftline{\color{Black}\bf {#1}}
    {#1}
   \end{ColorBoxedminipage}
   }
\end{center}
  \caption{#2}
  \label{#3}
\end{figure}
}{}

\newenvironment{BOX}[1]
{
\begin{center}
\fbox{
  \begin{ColorBoxedminipage}{16cm}
%    \leftline{\color{Black}\bf {#1}}
    {#1}
   \end{ColorBoxedminipage}
   }
\end{center}
}{}

\newcommand\vecnot[1]{\boldsymbol{#1}}
\newcommand\optvecnot[1]{\vecnot{#1}_{opt}}

\usepackage{amsmath}
\DeclareMathOperator*{\argmax}{arg\,max}
\DeclareMathOperator*{\argmin}{arg\,min}
%%%%%%%%%%%%%%%%%%%%%%%%%%%%%%%%%%%%%%%%%%%%%%%%%%%%%%
%%%%%%%%%%%%%%      Document flags     %%%%%%%%%%%%%%%
%%%%%%%%%%%%%%%%%%%%%%%%%%%%%%%%%%%%%%%%%%%%%%%%%%%%%%
\def\DEFIncludeAttenuation{}
\def\DEFInclueApplication{}
%%%%%%%%%%%%  Document Code starts here %%%%%%%%%%%%%%

%%%%%%%%%%%%%%    aliases    %%%%%%%%%%%%%%

% \newcommand{\Brace}[1]{\left\{{#1}\right\}}
\newcommand{\dTheta}{\Delta\theta}
\newcommand{\dPhi}{\Delta\phi}
\newcommand{\dThetaO}{\Delta\theta\rBrace{\omega}}
\newcommand{\dPhiO}{\Delta\phi\rBrace{\omega}}
\newcommand{\dThetaHPBW}{\Delta\theta_{\text{HPBW}}}
\newcommand{\dOmega}{\Delta\omega}
\newcommand{\dR}{\Delta{R}}
\newcommand{\dTau}{CHANGE_TO_DPHI}
% \newcommand{\D}[2]{\mathcal{D}\rBrace{#1,#2}}
% \newcommand{\Dp}[2]{\mathcal{D}^{#2}\rBrace{#1}}
\newcommand{\D}[2]{\text{D}\rBrace{#1,#2}}
\newcommand{\Dp}[2]{\text{D}^{#2}\rBrace{#1}}
% \newcommand{\coefSetName}{\text{CB}}
\newcommand{\coefSetName}{\text{CB}}
\newcommand{\vd}{\vecnot{d}}
\newcommand{\vdO}{\vecnot{d}\rBrace{\omega}}
\newcommand{\vdC}{\vecnot{d}^{\ast}}
\newcommand{\vx}{\vecnot{x}}
\newcommand{\vAlpha}{\vecnot{\alpha}}
\newcommand{\vBeta}{\vecnot{\beta}}
\newcommand{\vdI}[1]{\vecnot{d}_{#1}}
\newcommand{\vdHatI}[1]{\hat{\vecnot{d}}_{#1}}
\newcommand{\vdHatIC}[1]{\hat{\vecnot{d}}^{\ast}_{#1}}
\newcommand{\vdTI}[1]{\vecnot{d}^{T}_{#1}}
\newcommand{\gI}[1]{g_{#1}}
\newcommand{\gIHat}[1]{\hat{g}_{#1}}
\newcommand{\phiI}[1]{\phi_{#1}}
\newcommand{\phiIHat}[1]{\hat{\phi}_{#1}}
\newcommand{\vAlphaTI}[1]{\vecnot{\alpha}^{T}_{#1}}
\newcommand{\vAlphaHI}[1]{\vecnot{\alpha}^{H}_{#1}}
\newcommand{\vBetaTI}[1]{\vecnot{\beta}^{T}_{#1}}
\newcommand{\vBetaHI}[1]{\vecnot{\beta}^{H}_{#1}}
\newcommand{\vAlphaI}[1]{\vecnot{\alpha}_{#1}}
\newcommand{\vAlphaCI}[1]{\vecnot{\alpha}^{*}_{#1}}
\newcommand{\vBetaI}[1]{\vecnot{\beta}_{#1}}
\newcommand{\vBetaCI}[1]{\vecnot{\beta}^{*}_{#1}}
\newcommand{\vAlphaIW}[1]{\vecnot{\alpha}_{#1}}
\newcommand{\vBetaIW}[1]{\vecnot{\beta}_{#1}}
\newcommand{\vAlphaC}{\vecnot{\alpha}^{\ast}}
\newcommand{\vBetaC}{\vecnot{\beta}^{\ast}}
\newcommand{\vBetaFB}{\vecnot{\beta}_{\coefSetName}}
\newcommand{\vBetaCFB}{\vecnot{\beta}^{*}_{\coefSetName}}
\newcommand{\vAlphaFB}{\vecnot{\alpha}_{\coefSetName}}
\newcommand{\vAlphaCFB}{\vecnot{\alpha}^{*}_{\coefSetName}}
\newcommand{\vdT}{\vecnot{d}^{T}}
\newcommand{\vxT}{\vecnot{x}^{T}}
\newcommand{\vAlphaT}{\vecnot{\alpha}^{T}}
\newcommand{\vBetaT}{\vecnot{\beta}^{T}}
\newcommand{\vdH}{\vecnot{d}^{H}}
\newcommand{\vdHat}{\hat{\vecnot{d}}}
\newcommand{\vdHatC}{\hat{\vecnot{d}}^{\ast}}
\newcommand{\vxH}{\vecnot{x}^{H}}
\newcommand{\vAlphaH}{\vecnot{\alpha}^{H}}
\newcommand{\vBetaH}{\vecnot{\beta}^{H}}
\newcommand{\vEta}{\vecnot{\eta}}
\newcommand{\vEtaT}{\vEta^{T}}
\newcommand{\vEtaH}{\vEta^{H}}
%\newcommand{\F}[1]{#1^{\mathcal{F}}}
\newcommand{\F}[1]{\text{\MakeUppercase{#1}}}
\newcommand{\ePhi}[1]{\exp{\rBrace{#1j\phi}}}
\newcommand{\thetaD}{\theta_{\text{d}}}
\newcommand{\Hba}{H_{\vBeta,\vAlpha}}
\newcommand{\HbaFB}{H_{\vecnot{\beta}_{\coefSetName},\vecnot{\alpha}_{\coefSetName}}}
\newcommand{\vp}{\vecnot{p}}
\newcommand{\vpi}[1]{\vp_{#1}}
\newcommand{\vpt}{\vp_{\text{t}}}
\newcommand{\Steer}[1]{\vd_{#1}} 
\newcommand{\aTd}{\vAlphaT\Steer{}} 
\newcommand{\aHd}{\vAlphaH\Steer{}} 
\newcommand{\bTd}{\vBetaT\Steer{}}
\newcommand{\bHd}{\vBetaH\Steer{}}
\newcommand{\Hr}{\mathcal{H}}
\newcommand{\HrTPr}{\mathcal{H}_{\Delta\theta,\Delta\phi,r}}
\newcommand{\myTodo}[2]{\ifdefined\showTodo{\todo[#1]{#2}}\fi}
\newcommand{\myTodoNew}[2]{\ifdefined\showTodoNew{\todo[#1]{#2}}\fi}
\newcommand{\Tx}{\text{T}_{\text{x}}} 
\usepackage{xparse}
\usepackage{latexsym}
\usepackage{amsfonts}
\usepackage{graphicx}
\usepackage{txfonts}
\usepackage{wasysym}
\usepackage{enumitem}
\usepackage{adjustbox}
\usepackage{ragged2e}
\usepackage{tabularx}
\usepackage{changepage}
\usepackage{setspace}
\usepackage{hhline}
\usepackage{multicol}
\usepackage{float}
\usepackage{multirow}
\usepackage{makecell}
\usepackage{fancyhdr}
\usepackage[toc,page]{appendix}
\usepackage[T1]{fontenc}
\usepackage{hyperref}
\usepackage{esvect}
\hypersetup{
    colorlinks=true,
    linkcolor=blue,
    filecolor=magenta,      
    urlcolor=cyan,
}
\usepackage{isomath}
\usepackage{fixmath}
\usepackage{tikz}
\usepackage{textcomp}
\usepackage{epstopdf} %converting to PDF
\usepackage{upgreek}
\usepackage{mathtools}
\usepackage{amsmath}
\usepackage{xfrac}
\usepackage{lipsum}
\usepackage[colorinlistoftodos]{todonotes}
\usepackage[percent]{overpic}
\usepackage{stfloats}
\usepackage{subcaption}
\usepackage{longtable}

\DeclareTextCommandDefault{\nobreakspace}{\leavevmode\nobreak\ }

% \usepackage{listings}
% \usepackage{color} %red, green, blue, yellow, cyan, magenta, black, white
% \definecolor{mygreen}{RGB}{28,172,0} % color values Red, Green, Blue
% \definecolor{mylilas}{RGB}{170,55,241}

% \lstset{language=Matlab,%
%     %basicstyle=\color{red},
%     breaklines=true,%
%     morekeywords={matlab2tikz},
%     keywordstyle=\color{blue},%
%     morekeywords=[2]{1}, keywordstyle=[2]{\color{black}},
%     identifierstyle=\color{black},%
%     stringstyle=\color{mylilas},
%     commentstyle=\color{mygreen},%
%     showstringspaces=false,%without this there will be a symbol in the places where there is a space
%     numbers=left,%
%     numberstyle={\tiny \color{black}},% size of the numbers
%     numbersep=9pt, % this defines how far the numbers are from the text
%     emph=[1]{for,end,break},emphstyle=[1]\color{red}, %some words to emphasise
%     %emph=[2]{word1,word2}, emphstyle=[2]{style},    
% }

\usepackage[framed,numbered,autolinebreaks,useliterate]{misc/mcode}